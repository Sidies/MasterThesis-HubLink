
\chapter{Evaluation}
\label{ch:experimentation}

This chapter presents our evaluation of HubLink, which constitutes contribution \hyperref[enum:c1]{\textbf{C1}} of this thesis. We conducted the evaluation following the \gls{gqm} plan that we outlined in Section~\ref{sec:evaluation_goals_and_metrics}. To establish a comparative baseline, we incorporated five \gls{kgqa} approaches sourced from existing literature for which we detail the selection process in Section~\ref{sec:implementation_baselines}. All \gls{kgqa} approaches, including the baselines, were executed using their final configurations, which we determined through the parameter selection process described in \autoref{ch:parameter_selection_process}. Particularly for our HubLink approach, the chosen configuration along with three additional variations have been evaluated:

\begin{enumerate}
    \item \emph{HubLink (T)}: This configuration has been selected through our parameter selection process (see \autoref{tab:hublink_final_config}) and utilizes the \emph{graph traversal} strategy of HubLink.
    \item \emph{HubLink (D)}: This variant employs the same configuration parameters as HubLink (T) but instead uses the \emph{direct retrieval} strategy.
    \item \emph{HubLink (F)}: We designed this configuration for reduced runtime, employing the \emph{direct retrieval} strategy, and limiting the \emph{number of hubs} to 10 per question.
    \item \emph{HubLink (O)}: This variant shares its parameters with HubLink (T) but utilizes the \emph{mxbai-embed-large} embedding model and the \emph{Qwen2.5-14B} \gls{llm}.
\end{enumerate}

The first variant HubLink (T) is expected to achieve the best performance across the variants, as the parameters have been chosen by the parameter selection process. We introduced the other variants to better understand the performance characteristics of HubLink under different operational conditions. Specifically, HubLink (D) allows us to compare the efficacy of the \emph{direct retrieval} strategy against the \emph{graph traversal} strategy used in HubLink (T), which are explained in Section~\ref{sec:hublink_overview_retrieval_generation}. HubLink (F) aims to provide an economical version optimized for rapid execution, which is a crucial factor for practical usability. Finally, HubLink (O) enables us to assess the performance of our approach when using open-source models. However, the open-source variant is not expected to yield competitive performance compared to the OpenAI models. This is because we were only able to run smaller models due to hardware constraints. Still, including this configuration makes it possible to understand the difference in performance when using smaller models.

Furthermore, our evaluation incorporated four distinct graph variants, which we introduce in Section~\ref{sec:contribution_templates}. However, because of the high cost involved, we were unable to test all the experiments for each of the variants. Consequently, we chose the graph variant \hyperref[enum:gv1]{\textbf{GV1}} for the experiments in which the use of multiple variants was not necessary for the reasons provided in Section~\ref{sec:selection_planning_graph_variant}.

In the sections that follow, we first discuss the evaluation results related to retrieval performance (\hyperref[sec:evaluation_goals_and_metrics]{\textbf{ReT1}}) in Section~\ref{sec:evaluating_relevance_and_robustness_of_retrieved_contexts}. Subsequently, Section~\ref{sec:evaluating_answer_alignment} analyzes the evaluation results concerning answer generation performance (\hyperref[sec:evaluation_goals_and_metrics]{\textbf{GeT1}} and \hyperref[sec:evaluation_goals_and_metrics]{\textbf{GeT2}}). We then conclude our evaluation with a comprehensive discussion of the findings in Section~\ref{sec:discussion_on_evaluation_results}, before finally addressing threats to validity in Section~\ref{sec:general_threats_to_validity}.




\section{Evaluating Retrieval Quality}
\label{sec:evaluating_relevance_and_robustness_of_retrieved_contexts}

In this section, we present our evaluation results addressing retrieval target \hyperref[sec:evaluation_goals_and_metrics]{\textbf{ReT1}}. We begin by analyzing the retrieval performance of HubLink in comparison to the baseline approaches, with a focus on the accuracy and relevance of the retrieved contexts. Subsequently, we examine the impact that the chosen retrieval operation has on overall retrieval performance. Following this, we investigate the applicability and performance of HubLink in various scholarly literature search use cases. We then analyze the influence of type information, when present in the input question, on retrieval outcomes. Next, we assess the robustness of HubLink to structural and lexical variations within the graph. Furthermore, we analyze the relationship between retrieval performance metrics, runtime, and \gls{llm} token efficiency. Finally, we evaluate the environmental impact of HubLink relative to the baseline \gls{kgqa} approaches.

\subsection{Improvement of Retrieval Accuracy and Relevance}
\label{sec:results_retrieval_relevance_and_accuracy}

To effectively assist researchers in scholarly literature searches, a \gls{kgqa} approach must be capable of extracting a broad range of relevant information from the \gls{rkg} that matches a given question. In the following, we analyze the performance of our proposed HubLink approach in comparison to established \gls{kgqa} approaches, specifically focusing on the accuracy and relevance of the retrieved contexts. Here, accuracy refers to the extent to which the approach is capable of fetching the desired triples from the graph, while relevance pertains to the degree to which the retrieved triples correspond to the golden triples. 

The evaluation results in \autoref{tab:q11:relevance_and_accuracy} demonstrate that HubLink substantially improves the accuracy and relevance of retrieved triples compared to established \gls{kgqa} baseline methods. In the following, we first discuss the results regarding the different HubLink variants before we focus on the comparison of the baseline approaches.

\begin{table}[t]
\centering
% \resizebox{\textwidth}{!}{%
\begin{tabular}{@{}llllllll@{}}
\toprule
Approach & Recall & Precision & F1 & Hits@10 & MAP@10 & MRR@10 & EM@10 \\ 
\midrule
HubLink (T) & \textbf{0.754} & 0.246 & 0.328 & \textbf{0.512} & \textbf{0.299} & 0.502 & \textbf{0.298}  \\ 
HubLink (D) & 0.709 & 0.221 & 0.277 & 0.436 & 0.259 & 0.486 & 0.273 \\ 
HubLink (F) & 0.649 & \textbf{0.278} & \textbf{0.344} & 0.451 & 0.267 & 0.473 & 0.290 \\ 
HubLink (O) & 0.559 & 0.144 & 0.188 & 0.408 & 0.272 & \textbf{0.526} & 0.222 \\ 
DiFaR & 0.352 & 0.011 & 0.022 & 0.208 & 0.151 & 0.297 & 0.104 \\ 
Mindmap & 0.119 & 0.030 & 0.045 & 0.015 & 0.002 & 0.013 & 0.007 \\ 
FiDeLiS & 0.093 & 0.053 & 0.063 & 0.093 & 0.063 & 0.103 & 0.053 \\ 
\bottomrule
\end{tabular}%
% }
\caption[Comparative Retrieval Performance]{Comparative overall retrieval performance of HubLink and baseline \gls{kgqa} approaches on graph variant \hyperref[enum:gv1]{\textbf{GV1}} of the \gls{orkg}. All presented metrics are macro-averaged.}
\label{tab:q11:relevance_and_accuracy}
\end{table}

\subsubsection{Analyzing HubLink Variants}

The HubLink variant (T), which employs the graph traversal strategy, has achieved the highest scores among all variants. With regard to variant (D), which uses the direct retrieval strategy, we observe a decrease in performance. Specifically, the F1 score for HubLink (D) decreases by approximately 15.55\% (from 0.328 to 0.277), and MAP@10 shows a reduction of approximately 13.38\% (from 0.299 to 0.259). The MRR@10 metric experiences a smaller decrease of approximately 3.19\% (from 0.502 to 0.486), while EM@10 drops by approximately 8.39\% (from 0.298 to 0.273). In particular, the Hits@10 score sees a substantial decrease of approximately 14.84\% (from 0.512 to 0.436) for HubLink (D) compared to HubLink (T). These results suggest that, for our data, the graph traversal strategy is more effective. Although the direct retrieval strategy (HubLink (D)) offers the operational advantage of not requiring a predefined topic entity, this convenience comes at the cost of reduced retrieval efficacy. While we hypothesize that the size of the tested \gls{rkg} may not fully represent the differences between the strategies, the observed performance degradation with direct retrieval could already indicate potential challenges that could be amplified on larger and more complex graph structures. This aspect warrants further investigation in future work.

The HubLink (F) variant, designed for reduced runtime, presents a notable trade-off. Compared to HubLink (T), it shows a lower Recall (0.649 versus 0.754) and Hits@10 (0.451 versus 0.512). However, HubLink (F) achieves the highest Precision (0.278) and F1 score (0.344) among all evaluated variants, surpassing even HubLink (T) (Precision 0.246, F1 0.328). In our opinion, the improved Precision score is attributed to the reduced number of triples that this variant retrieves. This is because, by retrieving fewer triples overall, the proportion of truly relevant triples among those retrieved can be higher, thus increasing Precision. The results also underscore the significant role that the number of considered hubs plays in retrieval performance.

The open-source variant HubLink (O) generally shows lower performance in most metrics compared to the other HubLink configurations, as it records the lowest metric scores across all variants. This outcome is consistent with previous observations from the parameter selection process (see Section~\ref{tab:hublink_parameter_selection_part_1}). This suggests that the capability of the applied \gls{llm} has a substantial impact on the retrieval performance. An interesting exception is MRR@10, where HubLink (O) achieves the highest score of all variants. This indicates that when HubLink (O) does identify a correct triple within the top 10 results, that triple is often ranked high. However, the lower Hits@10 implies that the model is less frequently successful in placing a correct triple within the top 10.

% Possible challenges of direct strategy on larger graphs
% Number of Hubs have a significant role in retrieval performance. Increasing Recall but decreasing Precision and ranking quality.
% The LLM and Embedding models have a significant impact on retrieval performance

\subsubsection{Analyzing against Baselines}

HubLink (T) achieved a Recall of 0.754, representing a 114\% improvement over the next best \gls{kgqa} approach (\gls{difar}), which reached a Recall of 0.352. In contrast, Mindmap and FiDeLiS retrieved only approximately 10\% of the expected triples from the graph. This result indicates that the baseline approaches struggled significantly with the task. In particular, both \gls{difar} and HubLink utilize dense vector retrieval mechanisms, while Mindmap and FiDeLiS rely on subgraph construction and stepwise reasoning, respectively. These findings suggest that in the context of our evaluation, embedding-based approaches offer a clear advantage in retrieving a larger proportion of relevant triples.

Although Precision values were generally lower across all models, HubLink (T) again outperformed baselines with a Precision of 0.246, highlighting its relatively greater ability to return correct triples. However, the absolute score is rather low, which indicates that a large number of irrelevant triples is also retrieved. However, as discussed in Section~\ref{sec:selecting_tuning_metric}, these triples may still contribute positively to answer generation.

In evaluating the effectiveness of the ranking, HubLink (T) achieved a Hits@10 score of 0.512, which is more than twice as high as the next best baseline. However, this metric also reveals that the most relevant triples did not always appear at the top, suggesting weaknesses in overall ranking performance. The other ranking metrics further clarify this pattern. The MAP@10 score of 0.299 and the MRR@10 score of 0.502 indicate that while HubLink generally ranks relevant triples higher than baselines, its ability to consistently prioritize them in the topmost positions remains limited. Nevertheless, these scores are significantly higher than those of other methods, confirming a substantial advancement in contextual relevance and ranking quality over baseline methods.

% Significantly better retrieval performance than baseline approaches
% Baselines seem to generally struggle. Embedding-based methods seem to perform the best
% Absolute Precision and ranking performance of HubLink suggests that many irrelevant  triples are included. The retriever shows limited performance in ranking.


\subsubsection{Discussion on Overall Retrieval Performance}

Our analysis of HubLink variants and the comparison to baseline methods reveals significant advancements in retrieval accuracy and relevance for scholarly literature search. We determined that the graph traversal strategy yields superior performance over the direct retrieval strategy. Furthermore, we observed that increasing the number of hubs during retrieval enhances Recall. However, this enhancement corresponds to a reduction in Precision and ranking performance. These outcomes suggest that, while more hubs allow for a broader retrieval scope, thereby capturing a larger set of potentially relevant items, the specificity to the query context diminishes. Moreover, the results for HubLink (O) affirm the critical role that the choice of the \gls{llm} and the underlying embedding model plays in the retrieval performance of HubLink. Additionally, the absolute Precision and ranking scores indicate a current limitation in the ability of HubLink to accurately assess the relevance of triples.

When we contrast HubLink with established baseline approaches, the advantages of our method become particularly clear. All HubLink variants demonstrate a marked improvement across all evaluated metrics. We generally observe low performance from the baseline approaches, suggesting their limited applicability to scholarly literature searches. Interestingly, the data indicate that embedding-based methods are superior for this task.


% Q1.1 To what extent does the HubLink retrieval algorithm improve context relevance and accuracy compared to baseline KGQA methods in scholarly literature search?
\begin{enumerate}[label={}]
    \item \textbf{Answer to \hyperref[sec:evaluation_gqm_plan]{Q1}:} \textit{Our HubLink retrieval approach significantly improves retrieval performance with respect to relevance and Precision compared to established baseline \gls{kgqa} methods in the scholarly literature search setting. The approach more than doubles the Recall of the next best baseline, indicating a substantially better retrieval of relevant triples. Precision is also markedly higher, suggesting an improved assessment of relevance during retrieval. However, the absolute Precision and ranking performance of HubLink highlights a weakness and a potential need for improvement. Nevertheless, these results collectively demonstrate that HubLink offers a notable improvement in the retrieval of contextually relevant triples.}
\end{enumerate}

\subsection{Impact of Operation Complexity}

\begin{table}[hp]
\centering
\resizebox{\textwidth}{!}{%
\begin{tabular}{@{}llllllll@{}}
\toprule
Retrieval Operation & Recall & Precision & F1 & Hits@10 & MAP@10 & MRR@10 & EM@10 \\ 
\midrule
\multicolumn{8}{c}{HubLink (T)} \\
\midrule
basic & \textbf{0.917} & \textbf{0.382} & \textbf{0.480} & \textbf{0.917} & \textbf{0.445} & 0.490 & \textbf{0.389} \\
aggregation & 0.810 & 0.209 & 0.285 & 0.497 & 0.225 & 0.347 & 0.240 \\
counting & 0.840 & 0.275 & 0.372 & 0.644 & 0.357 & 0.526 & 0.340 \\
ranking & 0.817 & 0.321 & 0.414 & 0.561 & 0.360 & \textbf{0.576} & 0.363 \\
comparative & 0.742 & 0.262 & 0.366 & 0.456 & 0.320 & 0.560 & 0.296 \\
relationship & 0.628 & 0.254 & 0.314 & 0.410 & 0.298 & 0.528 & 0.331 \\
negation & 0.584 & 0.072 & 0.122 & 0.244 & 0.125 & 0.419 & 0.144 \\
superlative & 0.656 & 0.129 & 0.193 & 0.319 & 0.207 & 0.540 & 0.237 \\
\midrule
\multicolumn{8}{c}{HubLink (D)} \\
\midrule
basic & \textbf{0.861} & 0.217 & 0.276 & \textbf{0.611} & 0.297 & 0.332 & 0.228 \\
aggregation & 0.730 & 0.166 & 0.217 & 0.388 & 0.188 & 0.365 & 0.200 \\
counting & 0.723 & 0.218 & 0.293 & 0.481 & 0.287 & 0.410 & 0.253 \\
ranking & 0.659 & 0.221 & 0.278 & 0.428 & 0.278 & 0.494 & 0.269 \\
comparative & 0.701 & 0.314 & \textbf{0.376} & 0.444 & 0.287 & 0.537 & 0.339 \\
relationship & 0.689 & \textbf{0.347} & \textbf{0.376} & 0.456 & \textbf{0.314} & 0.627 & \textbf{0.411} \\
negation & 0.639 & 0.065 & 0.118 & 0.325 & 0.169 & 0.534 & 0.200 \\
superlative & 0.690 & 0.133 & 0.204 & 0.332 & 0.229 & \textbf{0.635} & 0.244 \\
\midrule
\multicolumn{8}{c}{HubLink (F)} \\
\midrule
basic & \textbf{0.806} & 0.279 & 0.338 & \textbf{0.694} & \textbf{0.364} & 0.392 & 0.287 \\
aggregation & 0.652 & 0.215 & 0.277 & 0.427 & 0.172 & 0.267 & 0.235 \\
counting & 0.779 & 0.273 & 0.376 & 0.477 & 0.301 & 0.561 & 0.273 \\
ranking & 0.630 & 0.236 & 0.320 & 0.404 & 0.224 & 0.458 & 0.235 \\
comparative & 0.617 & 0.366 & 0.428 & 0.504 & 0.355 & 0.549 & 0.366 \\
relationship & 0.610 & \textbf{0.420} & \textbf{0.443} & 0.456 & 0.310 & 0.541 & \textbf{0.428} \\
negation & 0.509 & 0.148 & 0.216 & 0.311 & 0.177 & 0.487 & 0.211 \\
superlative & 0.575 & 0.251 & 0.315 & 0.325 & 0.232 & \textbf{0.572} & 0.263 \\
\midrule
\multicolumn{8}{c}{HubLink (O)} \\
\midrule
basic & \textbf{0.806} & \textbf{0.280} & \textbf{0.345} & \textbf{0.806} & \textbf{0.590} & 0.617 & 0.297 \\
aggregation & 0.638 & 0.123 & 0.164 & 0.426 & 0.250 & 0.391 & 0.204 \\
counting & 0.710 & 0.170 & 0.249 & 0.600 & 0.404 & \textbf{0.714} & 0.273 \\
ranking & 0.611 & 0.128 & 0.188 & 0.394 & 0.256 & 0.514 & 0.234 \\
comparative & 0.387 & 0.075 & 0.118 & 0.285 & 0.179 & 0.443 & 0.150 \\
relationship & 0.465 & 0.265 & 0.274 & 0.328 & 0.259 & 0.653 & \textbf{0.328} \\
negation & 0.499 & 0.039 & 0.067 & 0.232 & 0.121 & 0.475 & 0.144 \\
superlative & 0.322 & 0.031 & 0.055 & 0.149 & 0.084 & 0.382 & 0.106 \\
\bottomrule
\end{tabular}%
}
\caption[HubLink Performance by Operation Complexity]{Impact of the retrieval operation on the performance of the HubLink approach. The results are based on graph variant \hyperref[enum:gv1]{\textbf{GV1}} and all metrics have been macro-averaged. The results for the baseline approaches are provided in Appendix~\ref{sec:appendix:additional_evaluation_results_operation_complexity}.}
\label{tab:q12:retrieval_operation}
\end{table}

To effectively address a wide array of scholarly questions characterized by varying semantic and logical complexity, a \gls{kgqa} approach must be able to handle diverse reasoning operations. In the following, we analyze the performance of HubLink across different operations, which are provided by our \gls{kgqa} retrieval taxonomy (see \autoref{ch:question_catalog}). 

The results of the experiment are presented in \autoref{tab:q12:retrieval_operation}, which shows the retrieval performance of each of the four HubLink variants for eight distinct operations.

For \emph{Basic} operations, which entail the retrieval of a single triple without further processing, all HubLink variants demonstrate their highest Recall and Hits@10 scores. These findings suggest that the approach shows the best performance in retrieving triples for simple, fact-based questions solvable through single triple lookups. 

Furthermore, although HubLink (T) and HubLink (O) have achieved the highest F1 scores with the \emph{Basic} operation questions, the same is not true for the other variants. HubLink (D) and HubLink (F) achieved the highest F1 scores with \emph{Comparative} and \emph{Relationship} questions. Moreover, we observe that Precision and F1 scores are generally lower than Recall across all operations, suggesting inherent difficulties for the retriever in precisely identifying relevant information. This is particularly evident in the case of \emph{Negation} and \emph{Superlative} operations, which consistently yield lower Precision and F1 scores across all variants when compared to other operations. This pattern indicates that the retriever faces challenges in accurately identifying relevant information when these logical constructs are involved. The Hits@10 metric further confirms this trend, as it also shows lower scores for these operations.

An examination of the MAP@10, MRR@10, and EM@10 metrics does not reveal a clear, overarching trend in performance across the different operations that is consistent for all HubLink variants. However, we observe a minor trend where \emph{negation} and \emph{superlative} operations tend to yield lower scores.

% Basic operation has the best performance, but is also the least complex
% Operations other than basic generally emit a lower performance.
% Negation and superlative operations are the most difficult for the retriever

% Q1.2 How does retrieval performance vary with the logical complexity of operations required by different scholarly questions?
\begin{enumerate}[label={}]
    \item \textbf{Answer to \hyperref[sec:evaluation_gqm_plan]{Q2}:} \textit{The results indicate that the highest Recall is achieved with basic operation questions, with a noticeable performance drop for questions demanding more complex reasoning operations. Furthermore, the results suggest a general limitation in assessing relevance, as the retriever consistently retrieves more contexts than asked for and struggles to differentiate effectively between relevant and irrelevant contexts. This difficulty is particularly pronounced for negation and superlative operations.}
\end{enumerate}


\subsection{Applicability to Different Scholarly Literature Search Use Cases}


\begin{table}[t]
\centering
% \resizebox{\textwidth}{!}{%
\begin{tabular}{@{}llllllll@{}}
\toprule
Use Case & Recall & Precision & F1 & Hits@10 & MAP@10 & MRR@10 & EM@10 \\ 
\midrule
\multicolumn{8}{c}{HubLink (T)} \\
\midrule
1 & 0.800 & \textbf{0.507} & \textbf{0.575} & \textbf{0.767} & \textbf{0.557} & \textbf{0.644} & \textbf{0.552} \\
2 & \textbf{0.848} & 0.252 & 0.364 & 0.729 & 0.301 & 0.341 & 0.281 \\
3 & 0.768 & 0.252 & 0.343 & 0.507 & 0.268 & 0.543 & 0.287 \\
4 & 0.663 & 0.198 & 0.277 & 0.395 & 0.266 & 0.561 & 0.255 \\
5 & 0.702 & 0.122 & 0.186 & 0.350 & 0.184 & 0.408 & 0.213 \\
6 & 0.779 & 0.206 & 0.286 & 0.428 & 0.278 & 0.512 & 0.257 \\
\midrule
\multicolumn{8}{c}{HubLink (D)} \\
\midrule
1 & \textbf{0.791} & \textbf{0.450} & \textbf{0.510} & \textbf{0.745} & \textbf{0.495} & 0.556 & \textbf{0.489} \\
2 & 0.715 & 0.073 & 0.127 & 0.410 & 0.155 & 0.281 & 0.111 \\
3 & 0.675 & 0.265 & 0.332 & 0.444 & 0.234 & 0.459 & 0.298 \\
4 & 0.543 & 0.195 & 0.225 & 0.302 & 0.184 & 0.444 & 0.234 \\
5 & 0.756 & 0.144 & 0.191 & 0.317 & 0.183 & 0.481 & 0.236 \\
6 & 0.790 & 0.213 & 0.295 & 0.463 & 0.341 & \textbf{0.691} & 0.284 \\
\midrule
\multicolumn{8}{c}{HubLink (F)} \\
\midrule
1 & \textbf{0.770} & \textbf{0.524} & \textbf{0.591} & \textbf{0.758} & \textbf{0.499} & 0.538 & \textbf{0.533} \\
2 & 0.674 & 0.150 & 0.228 & 0.438 & 0.202 & 0.272 & 0.158 \\
3 & 0.611 & 0.262 & 0.313 & 0.347 & 0.162 & 0.385 & 0.240 \\
4 & 0.524 & 0.268 & 0.330 & 0.390 & 0.254 & 0.503 & 0.291 \\
5 & 0.675 & 0.234 & 0.308 & 0.370 & 0.247 & \textbf{0.597} & 0.253 \\
6 & 0.676 & 0.271 & 0.336 & 0.489 & 0.293 & 0.534 & 0.306 \\
\midrule
\multicolumn{8}{c}{HubLink (O)} \\
\midrule
1 & 0.689 & \textbf{0.370} & \textbf{0.436} & 0.667 & \textbf{0.581} & \textbf{0.719} & \textbf{0.395} \\
2 & \textbf{0.776} & 0.195 & 0.284 & \textbf{0.690} & 0.297 & 0.380 & 0.250 \\
3 & 0.531 & 0.152 & 0.162 & 0.374 & 0.275 & 0.512 & 0.254 \\
4 & 0.357 & 0.047 & 0.068 & 0.220 & 0.156 & 0.473 & 0.148 \\
5 & 0.463 & 0.044 & 0.077 & 0.251 & 0.150 & 0.484 & 0.135 \\
6 & 0.609 & 0.112 & 0.176 & 0.360 & 0.243 & 0.620 & 0.195 \\
\bottomrule
\end{tabular}%
% }
\caption[HubLink Performance by Use Case]{Assessment of different scholarly use cases on the retrieval performance of HubLink. The results are based on graph variant \hyperref[enum:gv1]{\textbf{GV1}} and all metrics have been macro-averaged. The use cases are introduced in Section~\ref{sec:qa_use_cases}. The results for the baseline approaches are provided in Appendix~\ref{sec:appendix:additional_evaluation_results_use_cases}.}
\label{tab:q13:use_cases}
\end{table}

\autoref{tab:q13:use_cases} presents the evaluation results for the HubLink approach across six predefined scholarly literature search use cases. These use cases, as detailed in Section~\ref{sec:qa_use_cases}, are distinguished by their input condition type (metadata, content, or both) and expected answer type (metadata or content). The subsequent discussion analyzes the performance across the four different HubLink variants.

Concerning Recall performance, use cases involving metadata conditions in the query demonstrate superior results. Specifically, Use Case 1 (metadata input, metadata output) and Use Case 2 (metadata input, content output) generally have the highest Recall scores across all HubLink variants. Use Case 5 (mixed input, content output) and Use Case 6 (mixed input, metadata output), which also include metadata conditions, tend to follow closely in Recall performance. This pattern suggests that the approach encounters greater challenges with query conditions based solely on \emph{content}, as Use Case 3 (content input, metadata output) and Use Case 4 (content input, content output) consistently show lower Recall values.

We further observe that the second, fourth, and fifth use cases, which require content-type answers, tend to exhibit lower Precision scores across all HubLink variants. This suggests that the approach is less effective in accurately identifying relevant information when the expected answer type is content. Regarding the ranking metrics (MAP@10, MRR@10, EM@10), Use Case 1 consistently demonstrates the strongest performance across all HubLink variants for most of these metrics. However, for the other five use cases, no consistently discernible pattern emerges that would suggest a general superiority or inferiority of any specific use case across all variants or ranking metrics.

Overall, the results suggest that optimal performance is most frequently achieved in use cases that involve metadata-based query conditions or require metadata as output. This observation aligns with current scholarly research practice, where researchers typically search by titles or keywords. Nevertheless, the data indicate that the Recall performance of HubLink for queries requiring content-specific information is also relatively high. This suggests that HubLink has the potential to transform current research workflows from metadata-based to content-based searches. However, our findings also reveal limitations in effective ranking and filtering, particularly when dealing with content-based query conditions or answers, as well as mixed condition scenarios. This indicates that the integration of subsequent filtering and reranking mechanisms could be beneficial to mitigate these limitations.

% The retriever is generally better at handling metadata conditions in the query
% The Recall performance is generally robust across most use cases
% The Precision and ranking effectiveness tends to decline in scenarios involving purely content-based query conditions or when content-type answers are required


% Q1.3 How does retrieval performance vary across distinct scholarly literature-search use cases?
\begin{enumerate}[label={}]
    \item \textbf{Answer to \hyperref[sec:evaluation_gqm_plan]{Q3}:} \textit{The Recall performance of the proposed HubLink approach is generally robust across most use cases. However, the overall performance for Precision and ranking effectiveness tends to decline in scenarios involving content-based question conditions or when content-type answers are required. The approach often includes a higher proportion of irrelevant information and faces challenges in ranking the most relevant triples at the top positions for content-focused use cases.}
\end{enumerate}


\subsection{Impact of Type Information in the Question}


\begin{table}[t]
\centering
% \resizebox{\textwidth}{!}{%
\begin{tabular}{@{}llllllll@{}}
\toprule
Semi-Typed & Recall & Precision & F1 & Hits@10 & MAP@10 & MRR@10 & EM@10 \\ 
\midrule
\multicolumn{8}{c}{HubLink (T)} \\
\midrule
True & \textbf{0.763} & \textbf{0.258} & \textbf{0.352} & \textbf{0.567} & \textbf{0.328} & \textbf{0.512} & \textbf{0.323} \\ 
False & 0.747 & 0.233 & 0.302 & 0.456 & 0.268 & 0.488 & 0.273 \\ 
\midrule
\multicolumn{8}{c}{HubLink (D)} \\
\midrule
True & \textbf{0.730} & \textbf{0.229} & \textbf{0.302} & \textbf{0.461} & \textbf{0.272} & \textbf{0.493} & \textbf{0.284} \\ 
False & 0.691 & 0.212 & 0.251 & 0.411 & 0.247 & 0.485 & 0.263 \\ 
\midrule
\multicolumn{8}{c}{HubLink (F)} \\
\midrule
True & \textbf{0.680} & \textbf{0.321} & \textbf{0.394} & \textbf{0.498} & \textbf{0.318} & \textbf{0.535} & \textbf{0.334} \\ 
False & 0.622 & 0.237 & 0.296 & 0.407 & 0.218 & 0.415 & 0.248 \\ 
\midrule
\multicolumn{8}{c}{HubLink (O)} \\
\midrule
True & 0.377 & 0.258 & 0.495 & 0.110 & 0.537 & 0.156 & 0.196 \\ 
False & \textbf{0.441} & \textbf{0.287} & \textbf{0.565} & \textbf{0.179} & \textbf{0.583} & \textbf{0.222} & \textbf{0.251} \\ 
\bottomrule
\end{tabular}%
% }
\caption[Results of Semi-Typed Questions on Retrieval Performance]{The impact of questions that add information about the condition types compared to those that do not. The results are based on graph variant \hyperref[enum:gv1]{\textbf{GV1}} and all metrics have been macro-averaged. The results for the baseline approaches are provided in Appendix~\ref{sec:appendix:additional_evaluation_results_type_information}.}
\label{tab:q14:semi_typed}
\end{table}

In \autoref{tab:q14:semi_typed}, the performance of four different HubLink variants is presented, comparing questions that include semi-typed information with those that do not. Such type annotations could assist the retriever in disambiguating the roles of entities within the graph and narrowing the candidate search space by filtering semantically irrelevant triples.

The results indicate that semi-typed questions lead to improved performance across all reported metrics for HubLink variants (T), (D), and (F). The magnitude of these improvements is generally modest for variants (T) and (D), while variant (F) exhibits more noticeable gains, particularly in Precision, F1, MAP@10, MRR@10, and EM@10. Conversely, for HubLink variant (O), which uses an open-source \gls{llm} and embedding model, the inclusion of type information results in a performance decline across all metrics compared to when type information is absent.

Although a consistent positive trend is observed for variants (T), (D), and (F) and a negative trend is observed for variant (O), the differences in performance, especially for (T) and (D), are relatively small. Overall, while the presence of type information appears to exhibit a minor to moderate positive effect for the non-open-source variants, its impact is not consistently substantial across all these variants in the current experimental setup. Interestingly, for the open-source variant, the inclusion of type information seems to have a negative effect on retrieval performance.

% \textbf{Q1.4} What impact does the presence or absence of explicit type information in questions have on the retrieval performance?
\begin{enumerate}[label={}]
    \item \textbf{Answer to \hyperref[sec:evaluation_gqm_plan]{Q4}:} \textit{A minor positive impact on retrieval performance is observed for non-open-source \glspl{llm} when explicit type information is included in questions. However, the magnitude of this positive effect is quite small and is not significant. In contrast, the open-source variant shows a negative impact when type information is included.}
\end{enumerate}

\subsection{Robustness to Structural and Lexical Variability in Graph Schema}
\label{sec:evaluation_robustness_to_structural}

In the following section, we analyze the robustness of the HubLink retrieval approach in terms of performance consistency across different graph schemas introduced in Section~\ref{sec:contribution_templates}. Furthermore, we analyze the impact of the number of hops required to reach the relevant triples. In this context, robustness refers to the ability of the retrieval system to maintain consistent performance in terms of accuracy and relevance.

In the subsequent evaluations, we focus on the performance of HubLink (T), as the execution of all HubLink variants on all graph variants would have been too costly. We chose HubLink variant (T) as it consistently shows the highest retrieval performance in our previous evaluations. 

\subsubsection{Analyzing Different Graph Variants}
\begin{table}[t]
\centering
% \resizebox{\textwidth}{!}{%
\begin{tabular}{@{}llllllll@{}}
\toprule
Graph & Recall & Precision & F1 & Hits@10 & MAP@10 & MRR@10 & EM@10 \\
\midrule
\multicolumn{8}{c}{HubLink (T)} \\
\midrule
GV1 & 0.755 & 0.246 & 0.327 & 0.513 & 0.298 & 0.500 & 0.299 \\
GV2 & 0.759 & 0.276 & 0.352 & 0.518 & 0.333 & 0.522 & 0.324 \\
GV3 & \textbf{0.812} & 0.350 & 0.423 & 0.596 & 0.408 & 0.650 & 0.406 \\
GV4 & 0.804 & \textbf{0.393} & \textbf{0.452} & \textbf{0.597} & \textbf{0.425} & \textbf{0.661} & \textbf{0.444} \\
\midrule
\multicolumn{8}{c}{DiFaR} \\
\midrule
GV1 & 0.352 & 0.011 & 0.022 & 0.207 & 0.150 & 0.295 & 0.104 \\
GV2 & 0.314 & 0.009 & 0.019 & 0.199 & 0.142 & 0.268 & 0.096 \\
GV3 & 0.523 & \textbf{0.017} & \textbf{0.035} & 0.302 & \textbf{0.230} & 0.442 & 0.154 \\
GV4 & \textbf{0.528} & \textbf{0.017} & \textbf{0.035} & \textbf{0.304} & 0.228 & \textbf{0.449} & \textbf{0.158} \\
\midrule
\multicolumn{8}{c}{Mindmap} \\
\midrule
GV1 & 0.119 & 0.030 & 0.045 & 0.015 & 0.002 & 0.013 & 0.007 \\
GV2 & 0.093 & 0.025 & 0.037 & 0.008 & 0.001 & 0.006 & 0.005 \\
GV3 & 0.133 & 0.043 & 0.061 & \textbf{0.030} & 0.007 & 0.023 & 0.015 \\
GV4 & \textbf{0.127} & \textbf{0.044} & \textbf{0.062} & \textbf{0.030} & \textbf{0.010} & \textbf{0.036} & \textbf{0.018} \\
\midrule
\multicolumn{8}{c}{FiDeLiS} \\
\midrule
GV1 & 0.092 & 0.052 & 0.063 & 0.092 & 0.062 & 0.103 & 0.053 \\
GV2 & 0.099 & 0.055 & 0.064 & 0.099 & 0.065 & 0.110 & 0.054 \\
GV3 & 0.259 & 0.114 & 0.139 & 0.259 & 0.150 & 0.240 & 0.112 \\
GV4 & \textbf{0.276} & \textbf{0.121} & \textbf{0.142} & \textbf{0.276} & \textbf{0.156} & \textbf{0.248} & \textbf{0.117} \\
\bottomrule 
\end{tabular}%
% }
\caption[Results on Retrieval Performance on Different Graph Variants]{Test results for the evaluation of four different graph variants introduced in Section~\ref{sec:contribution_templates}. All metrics have been macro-averaged.}
\label{tab:q5:different_graph_variants}
\end{table}

A key characteristic of the proposed HubLink approach and the applied baseline approaches is their schema-agnostic design, which allows the application to various graph schemas without having to change the configuration or implementation of the approach. To explore the practical implications of this flexibility, we evaluated the performance of HubLink and the baseline approaches on four different graph variants (\hyperref[tab:q5:different_graph_variants]{\textbf{GV1-GV4}}) introduced in Section~\ref{sec:contribution_templates}. The results are presented in \autoref{tab:q5:different_graph_variants} and discussed in the following.

The retrieval performance of HubLink (T) against the baseline approaches for the first graph variant (\hyperref[enum:gv1]{\textbf{GV1}}) has already been extensively discussed in Section~\ref{sec:results_retrieval_relevance_and_accuracy}. In summary, the results for \hyperref[enum:gv1]{\textbf{GV1}} indicate that the HubLink approach is capable of effectively retrieving relevant information from the \gls{orkg} and achieving high Recall and ranking scores. The baseline approaches, on the other hand, struggled to achieve comparable performance.

For graph variant \hyperref[enum:gv2]{\textbf{GV2}}, which also features long paths similar to \hyperref[enum:gv1]{\textbf{GV1}} but incorporates semantic grouping in which all information is stored in a single \gls{orkg} contribution, we observe minor improvements for HubLink (T). Although the Recall remains similar to \hyperref[enum:gv1]{\textbf{GV1}}, the Precision increased by approximately 12\%, leading to a higher F1 score. In addition, minor improvements in ranking performance can also be observed, particularly with MAP@10 increasing by approximately 12\% and EM@10 by 8\%. This suggests that the semantic grouping of information in \hyperref[enum:gv2]{\textbf{GV2}} may have a positive impact on the relevance assessment. This positive trend is further supported by FiDeLiS, which also shows improved performance in \hyperref[enum:gv2]{\textbf{GV2}} compared to \hyperref[enum:gv1]{\textbf{GV1}}. However, for Mindmap and DiFaR, the results indicate a reduction across all metrics, contradicting this assumption.

With \hyperref[enum:gv3]{\textbf{GV3}}, characterized by shorter paths and distributing information across multiple \gls{orkg} contributions, significant improvements were observed across all models. For HubLink (T), most metrics increase by approximately 15 to 25\% when compared to \hyperref[enum:gv2]{\textbf{GV2}}. However, for the Recall metric, the increase is only minor (approximately 7\%), suggesting that the Recall score stays relatively stable. For the baseline approaches, on the other hand, the shorter paths led to substantial improvements in overall retrieval performance. The scores of DiFaR increased by approximately 52 to 89\% across all metrics when compared to \hyperref[enum:gv2]{\textbf{GV2}}. Similarly, FiDeLiS more than doubled its scores across all metrics, with the Recall and Hits@10 scores increasing by over 160\%. The same trend can be observed for the Mindmap retriever, though the absolute scores still remain low. These results suggest that shorter paths are beneficial for the retrieval performance of all approaches, in particular for the baseline approaches.

Finally, the last graph variant \hyperref[enum:gv4]{\textbf{GV4}} combines shorter paths with the semantic grouping of information in a single \gls{orkg} contribution. Compared to \hyperref[enum:gv3]{\textbf{GV3}}, a minor performance increase can be observed in particular for the Precision (by approximately 12\%) and EM@10 (by approximately 9\%) scores. Similar improvements can be seen for the baseline methods, but the increase is only minor. These results suggest that the performance impact of grouping information in a single \gls{orkg} contribution compared to distributing it across multiple contributions is not significant.

In summary, the evaluations reveal consistent patterns across all \gls{kgqa} approaches examined: shorter path lengths significantly enhance performance. Moreover, the results indicate that the baseline approaches struggle substantially with longer paths, highlighting their limitations in multi-hop reasoning. In contrast, the performance of HubLink (T) remained substantially superior across all variants, demonstrating high adaptability and robust performance regardless of the underlying graph structure variations.


% The HubLink Recall stays relatively stable across all variants with only a minor increase of approximately 7\% when paths are short
% The HubLink Precision and ranking is significanlty improved when paths are shorter
% No noticable difference is observed for distributing information across multiple contributions or grouping all information in a single contribution
% Baselines perform significantly better when paths are shorter

\subsubsection{Analyzing Number of Hops}
\begin{table}[t]
\centering
% \resizebox{\textwidth}{!}{%
\begin{tabular}{@{}llllllll@{}}
\toprule
Hop Count & Recall & Precision & F1 & Hits@10 & MAP@10 & MRR@10 & EM@10 \\
\midrule
\multicolumn{8}{c}{HubLink (T)} \\
\midrule
1 & \textbf{1.000} & \textbf{1.000} & \textbf{1.000} & \textbf{1.000} & \textbf{1.000} & \textbf{1.000} & \textbf{1.000} \\
2 & 0.937 & 0.792 & 0.818 & 0.938 & 0.755 & 0.750 & 0.792 \\
3 & 0.732 & 0.358 & 0.446 & 0.699 & 0.427 & 0.509 & 0.404 \\
4 & 0.934 & 0.300 & 0.446 & 0.933 & 0.408 & 0.450 & 0.300 \\
5 & 0.813 & 0.212 & 0.313 & 0.524 & 0.268 & 0.459 & 0.246 \\
6 & 0.720 & 0.188 & 0.262 & 0.414 & 0.238 & 0.494 & 0.255 \\
\midrule
\multicolumn{8}{c}{DiFaR} \\
\midrule
1 & \textbf{1.000} & 0.010 & 0.010 & \textbf{1.000} & 0.167 & 0.167 & 0.100 \\
2 & 0.667 & 0.007 & 0.013 & 0.333 & 0.153 & 0.250 & 0.050 \\
3 & 0.380 & 0.007 & 0.018 & 0.311 & \textbf{0.307} & \textbf{0.469} & \textbf{0.113} \\
4 & 0.300 & 0.004 & 0.004 & 0.000 & 0.000 & 0.000 & 0.000 \\
5 & 0.465 & 0.014 & \textbf{0.028} & 0.210 & 0.149 & 0.326 & 0.109 \\
6 & 0.285 & \textbf{0.011} & 0.022 & 0.176 & 0.120 & 0.262 & 0.108 \\
\midrule
\multicolumn{8}{c}{Mindmap} \\
\midrule
1 & \textbf{1.000} & \textbf{0.070} & \textbf{0.130} & 0.000 & 0.000 & 0.000 & 0.000 \\
2 & 0.000 & 0.000 & 0.000 & 0.000 & 0.000 & 0.000 & 0.000 \\
3 & 0.136 & 0.031 & 0.048 & 0.005 & 0.001 & 0.006 & 0.004 \\
4 & 0.000 & 0.000 & 0.000 & 0.000 & 0.000 & 0.000 & 0.000 \\
5 & 0.132 & 0.029 & 0.043 & 0.013 & 0.002 & 0.011 & \textbf{0.009} \\
6 & 0.115 & 0.033 & 0.048 & \textbf{0.019} & \textbf{0.003} & \textbf{0.017} & 0.008 \\
\midrule
\multicolumn{8}{c}{FiDeLiS} \\
\midrule
1 & \textbf{1.000} & \textbf{0.500} & \textbf{0.670} & \textbf{1.000} & \textbf{1.000} & \textbf{1.000} & \textbf{0.857} \\
2 & 0.667 & 0.358 & 0.440 & 0.667 & 0.639 & 0.667 & 0.433 \\
3 & 0.060 & 0.037 & 0.045 & 0.060 & 0.052 & 0.062 & 0.037 \\
4 & 0.000 & 0.000 & 0.000 & 0.000 & 0.000 & 0.000 & 0.000 \\
5 & 0.126 & 0.068 & 0.082 & 0.126 & 0.072 & 0.118 & 0.067 \\
6 & 0.050 & 0.031 & 0.035 & 0.050 & 0.021 & 0.070 & 0.024 \\
\bottomrule
\end{tabular}%
% }
\caption[Results on Retrieval Performance for Different Hops]{Test results for the evaluation of different hops. The results are based on graph variant \hyperref[enum:gv1]{\textbf{GV1}} and all metrics have been macro-averaged.}
\label{tab:q5:hops_results}
\end{table}

% TODO: Verlinkung zur Mindmap diskussion

\begin{table}[t]
\centering
\begin{tabular}{cc}
\toprule
\textbf{Hops} & \textbf{Value} \\
\midrule
1 & 1 \\
2 & 6 \\
3 & 24 \\
4 & 5 \\
5 & 33 \\
6 & 101 \\
\bottomrule
\end{tabular}
\caption[Distribution of Hops on GV1]{The distribution of hops for graph variant \hyperref[enum:gv1]{\textbf{GV1}}.}
\label{tab:distribution_of_hops_gv1}
\end{table}

The number of hops signifies the maximum count of triples situated between the topic entity and the expected triples. It indicates how deeply the required information is embedded within the graph. In \autoref{tab:q5:hops_results}, the outcomes for HubLink and the baseline methods are presented, categorized by the necessary number of hops required to reach the desired triples. These evaluations were conducted using the graph variant \hyperref[enum:gv1]{\textbf{GV1}}. The distribution of questions based on the number of hops is detailed in \autoref{tab:distribution_of_hops_gv1}.

For questions requiring only one or two hops representing information located in the immediate vicinity of the topic entity, distinct performance patterns emerge. In the 1-hop scenario, all methods achieve perfect Recall. However, a significant difference is observed in Precision and rank-based metrics, while the baseline methods exhibit substantially lower performance, particularly \gls{difar} and Mindmap. For those questions that necessitate two hops, the performance advantage of HubLink becomes more evident. The Recall remains very high at 0.937, indicating retrieval of almost all relevant triples. In contrast, FiDeLiS and DiFaR achieve a Recall of 0.667, while Mindmap fails entirely for this hop count. It is evident from these results that the Mindmap approach has considerable issues with our evaluation task, which we discuss in Section~\ref{sec:discussion_on_evaluation_results}. Concerning Precision at two hops, the results for HubLink (0.792) again significantly surpass those of the baseline methods.

A notable divergence in performance occurs for questions that require three or more hops, corresponding to information deeper within the graph. Baseline methods generally exhibit a substantial degradation in effectiveness as the hop count increases. The performance of FiDeLiS drops sharply beyond two hops, with the Recall score only reaching 0.060 at three hops and becomes negligible or zero at four and six hops, but interestingly reaches a score 0.126 at five hops. Mindmap consistently yields very low Recall values (between 0.115 and 0.136) and near-zero values for all other metrics. \gls{difar} demonstrates greater resilience to high hop counts compared to FiDeLiS and Mindmap, maintaining Recall values between 0.285 and 0.465. However, even for \gls{difar}, a decline relative to the 1- and 2-hop scenarios is observed across all metrics. This collective performance decline across the baseline metrics suggests that the methods face significant challenges in effectively identifying relevant triples deep within the graph. In particular, the performance at 6 hops is based on a high number of questions, making the observed trend especially relevant.

In contrast, the results for HubLink demonstrate considerably greater robustness to increasing hop counts. The Recall score remains relatively high across all evaluated hop distances, consistently exceeding 0.717 and peaking at 1.00 (1-hop) and 0.934 (4-hops). Furthermore, the Recall value of HubLink consistently exceeds the best performance baseline (\gls{difar}) by a substantial margin, often nearly doubling the Recall value for hop counts of three and more. Similarly, the Precision and rank-based metrics for HubLink remain significantly higher than those of the baselines.

However, the results indicate that the effectiveness of HubLink is influenced by the hop count. We observe a large drop in Precision as the number of hops increases. This decline likely reflects the inherent challenge of maintaining high Precision as the search space expands with each additional hop. Exploring deeper into the graph retrieves a larger set of candidate triples, possibly making the accurate assessment of relevance for each triple more difficult. Rank-based metrics also show a generally decreasing trend for HubLink as hops increase, although the decline is less steep than for Precision. The Recall metric exhibits the least sensitivity to hop count for HubLink, indicating robustness in finding relevant information even when it is stored deep within the graph.

In summary, the analysis based on hop count reveals that HubLink provides a substantial improvement in retrieval performance compared to the evaluated baseline methods. This advantage is particularly pronounced for questions where the target information is located deeper within the \gls{kg}, requiring higher hop counts. While baseline methods struggle significantly as the number of hops increases, HubLink maintains comparatively high Recall and superior Precision and ranking performance, demonstrating a greater capability to handle complex queries requiring multi-hop reasoning.

% The baseline methods exhibit a significant degradation in performance as the hop count increases
% HubLink shows considerably greater robustness to increasing hop counts, though especially the Precision and rank-based metrics show decreasing trends

\subsubsection{Discussion on Robustness}

Our analysis reveals that HubLink exhibits substantially greater robustness to different graph variants than the baseline approaches. Our findings suggest that a primary factor that influences performance across all approaches is the path length within the graph. We found that shorter paths generally yield better retrieval results, particularly for Precision and rank-based metrics. Although all approaches benefit from shorter paths, the baseline methods demonstrate a pronounced degradation with longer paths, highlighting their limitations. In contrast, HubLink maintains considerably more stable and superior performance.

Furthermore, the types of entities in the four different applied graph variants differ. Because it was not necessary to change the configurations or implementations of the tested \gls{kgqa} approaches, this demonstrates their schema-agnostic design.

Moreover, the examination of performance across different hop counts underscores the enhanced multi-hop reasoning capabilities of HubLink. As the required number of hops increases, baseline methods show a significant decline in effectiveness. HubLink, however, sustains high Recall and superior Precision and ranking scores, even when navigating multiple hops. While we observe some decline in the Precision for HubLink with increasing hops, its overall ability to retrieve deeply located information far surpasses that of the baselines. In essence, the results indicate that HubLink adapts more effectively to diverse graph structures and complexities, particularly excelling in scenarios demanding robust multi-hop inference.


% \item \textbf{Q5} How robust is the proposed approach to structural and lexical variability across alternative knowledge graph schema representations?
\begin{enumerate}[label={}]
    \item \textbf{Answer to \hyperref[sec:evaluation_gqm_plan]{Q5}:} \textit{Our proposed HubLink approach significantly improves robustness compared to baseline methods. Our evaluations show that, while shorter graph paths generally improve retrieval performance, particularly with regard to relevance-focused metrics, HubLink is more effective across varied structures and excels at multi-hop reasoning, an area in which baseline methods struggle.}
\end{enumerate}

\subsection{Analysis of Runtime and LLM Token Consumption}
\label{sec:evaluation_runtime_and_tokens}

\begin{table}[t]
\centering
% \resizebox{\textwidth}{!}{%
\begin{tabular}{@{}lllllll@{}}
\toprule
Approach & Recall & Runtime (s) & LLM Tokens \\
\midrule 
HubLink (T) & \textbf{0.754} & 177.470 & 82365 \\
HubLink (D) & 0.709 & 155.387 & 79467 \\
HubLink (F) & 0.649 & 80.740 & 28404 \\
HubLink (O) & 0.559 & 239.593 & 82839 \\
DiFaR & 0.352 & \textbf{3.442} & \textbf{25020} \\
FiDeLiS & 0.092 & 129.914 & 102178  \\
Mindmap & 0.119 & 90.186 & 13986 \\
\bottomrule 
\end{tabular}%
% }
\caption[Results on Runtime and LLM Token Consumption]{Results of runtime and \gls{llm} token consumption per question. The results are based on graph variant \hyperref[enum:gv1]{\textbf{GV1}} and all metrics have been macro-averaged.}
\label{tab:runtime_tokens}
\end{table}

The runtime and \gls{llm} token consumption for the different HubLink versions and baseline approaches are presented in \autoref{tab:runtime_tokens}. These metrics provide insights into the computational efficiency and operational costs associated with each approach.

The comparison between the different HubLink variants reveals distinct trade-offs between Recall, runtime, and token consumption. HubLink (T), which uses the graph traversal strategy, had an average runtime of approximately 177.47 seconds and consumed 82,365 \gls{llm} tokens per question. Switching to the direct retrieval strategy, HubLink (D) shows improved efficiency, as the runtime decreases to approximately 155.39 seconds, a reduction of about 12.4\% compared to HubLink (T). However, token consumption stays almost the same with only a reduction of approximately 3.5\%. Although the runtime is slightly lower, the Recall decreases as well (approximately 5.8\%).

HubLink (F) is specifically optimized for speed by using the direct retrieval strategy and limiting the number of hubs to 10. It demonstrates the highest efficiency among the HubLink variants, as it required an average runtime of approximately 80.74 seconds per question. This makes HubLink (F) more than twice as fast as HubLink (T). Furthermore, this speed improvement is coupled with a significant decrease in resource usage since only 28,404 \gls{llm} tokens are needed. This token count is approximately 65.5\% fewer than HubLink (T). However, these efficiency gains result in a lower Recall of 0.649, a relative decrease of approximately 13.8\% compared to HubLink (T) and about 8.5\% compared to HubLink (D).

Conversely, the open-source version of HubLink (O) exhibits the longest average runtime at approximately 239.59 seconds. Its token consumption (82,839) is nearly identical to that of HubLink (T), an expected outcome given their shared configuration parameters despite differing underlying models. This high resource usage for HubLink (O) is further associated with the lowest Recall (0.559) among the HubLink variants.

Compared with baseline approaches, \gls{difar} stands out with an exceptional runtime efficiency, processing each question in approximately 3.44 seconds on average. However, the \gls{llm} token requirement of 25,020 is only marginally lower than that of HubLink (F) (28,404), indicating comparable operational costs between these two methods despite the vast difference in execution speed. Among the other baselines, Mindmap requires approximately 90.19 seconds per question, slightly slower than HubLink (F), while FiDeLiS takes approximately 129.91 seconds. In terms of token consumption, Mindmap is the most cost-effective approach, using only about 13,986 tokens per question. FiDeLiS, in contrast, incurs the highest cost, requiring over 102,178 tokens, substantially more than any other evaluated method.

In summary, the results demonstrate that all HubLink variants achieve considerably higher Recall than the baseline methods. While \gls{difar} offers the lowest runtime by a significant margin, its Recall is substantially lower. HubLink (F) emerges as a balanced configuration, providing Recall performance nearly double that of \gls{difar} (0.649 vs. 0.352) with a runtime that, although higher than \gls{difar}, remains considerably faster than HubLink (T) or (O). Furthermore, the token consumption, and therefore the estimated operational cost, of HubLink (F) is comparable to \gls{difar}. Although Mindmap offers the lowest token cost and FiDeLiS incurs the highest, both approaches exhibit significantly lower Recall compared to HubLink. Therefore, HubLink (F) presents a convincing trade-off, delivering strong Recall performance with manageable runtime and reasonable computational expense when compared to the baselines.


% \item \textbf{Q6} How efficient is the proposed approach considering runtime and language model tokens required when compared to baseline KGQA methods?
\begin{enumerate}[label={}]
    \item \textbf{Answer to \hyperref[sec:evaluation_gqm_plan]{Q6}:} \textit{The proposed HubLink approach achieves significantly higher Recall compared to baseline KGQA methods, with HubLink (F) balancing Recall, runtime efficiency, and operational token costs. While baseline methods like \gls{difar} offer superior runtime, they provide considerably lower Recall. HubLink (F) thus presents a favorable trade-off, offering strong retrieval performance alongside manageable computational costs.}
\end{enumerate}

\subsection{Environmental Sustainability Impact}
\label{sec:evaluating_sustainability}

\begin{table}[t]
\centering
% \resizebox{\textwidth}{!}{%
\begin{tabular}{@{}lllllll@{}}
\toprule
Approach & Recall & CE ($CO_2$) & CE$_{rel}$ & $\Delta_{CE}$ & n(CE) & n(CE$_{rel}$) \\
\midrule 
HubLink (T) & \textbf{0.753905} & 0.004187 & 0.005554 & 0.482545 & 0.550789 & 0.837496 \\
HubLink (D) & 0.709053 & 0.003664 & 0.005168 & 0.514019 & 0.608081 & 0.849320 \\
HubLink (F) & 0.649467 & 0.001907 & 0.002936 & 0.892193 & 0.800543 & 0.917609 \\
HubLink (O) & 0.559290 & 0.009216 & 0.016478 & 0.154695 & min & 0.503199 \\
DiFaR & 0.352012 & \textbf{0.000086} & \textbf{0.000244} & \textbf{9.233593} & max & max \\
FiDeLiS & 0.092840 & 0.003056 & 0.032921 & base & 0.674635 & min \\
Mindmap & 0.119467 & 0.002130 & 0.017828 & 0.038211 & 0.776120 & 0.461889 \\
\bottomrule 
\end{tabular}%
% }
\caption[Results on Environmental Sustainability]{Results of environmental sustainability analysis. The results are based on graph variant \hyperref[enum:gv1]{\textbf{GV1}} and all metrics have been macro-averaged.}
\label{tab:ablation_sustainability}
\end{table}
% \emph{HubLink (T)} uses the configuration from our parameter selection process. \emph{HubLink (F)} uses a faster configuration with only 10 hubs considered per inference and using the direct retrieval strategy. \emph{HubLink (O)} is the same configuration as (T) but uses mxbai-embed-large and Qwen2.5-14B.

Traditional performance metrics like Recall do not capture the environmental footprint associated with computational methods. Responding to appeals for the incorporation of such factors \cite{kaplan_responsible_2025}, we evaluate the environmental sustainability of the approaches by analyzing their \acrfull{ce} relative to their Recall performance. The results are based on the graph variant \hyperref[enum:gv1]{\textbf{GV1}} and are presented in \autoref{tab:ablation_sustainability}.

Similar to the previous runtime and cost analysis, we can see distinct sustainability profiles of the HubLink variants relative to their Recall capabilities. Although HubLink (T) achieves the highest Recall (0.753905), it incurs the highest absolute carbon emissions ($CE = 0.004187$ $CO_2$) relative to HubLink variants (D) and (F). Only HubLink (O) has higher absolute emissions ($CE = 0.009216$ $CO_2$) but, as indicated below, this is due to measurement limitations. Compared to the baseline approaches, the absolute carbon emissions of DiFaR are significantly lower ($CE = 0.000086$ $CO_2$), making it the most environmentally efficient approach across all our tested approaches. The consumption of FiDeLiS is similar to that of HubLink (D), and the consumption of Mindmap is similar to that of HubLink (F).

Looking at the relative carbon emissions ($CE_{rel}$), which is the ratio of carbon emissions to Recall, we observe that DiFaR is the most efficient approach with a relative carbon emission of 0.000244. This is followed by HubLink (F), with a relative carbon emission of 0.002936, which is approximately 12 times higher than DiFaR. The relative carbon emissions of HubLink (D) and (T) are 0.005168 and 0.005554, respectively, indicating that the direct retrieval strategy in HubLink (D) is more efficient than the graph traversal strategy in HubLink (T). The open-source variant, HubLink (O), has the highest relative carbon emissions at 0.016478, making it the least efficient option. While the relative carbon emissions of Mindmap (0.017828) are similar to HubLink (O), its Recall is significantly lower, indicating a less favorable sustainability profile. FiDeLiS has the highest relative carbon emissions (0.032921) among all approaches, making it the least efficient in terms of environmental impact and Recall performance. The same trends are observed for the delta values ($\Delta_{CE}$).

In conclusion, the evaluation shows a trade-off between Recall performance and environmental sustainability. \gls{difar} provides the most sustainable option, characterized by very low absolute and relative environmental impacts, but achieves considerably lower Recall than the HubLink approaches. In contrast, the fast HubLink variant (F) almost doubles the Recall compared to \gls{difar}, but incurs a disproportionately large increase in environmental cost per unit of Recall. Therefore, selecting an appropriate method requires careful consideration of the priority assigned to maximizing Recall versus minimizing environmental footprint. While HubLink (F) represents an optimized balance, it remains substantially less environmentally efficient per unit of Recall than \gls{difar}.

\textit{Note that this observation applies only to the inference phase. The substantial environmental costs associated with \gls{llm} training are not included as they fall outside the scope of this work. Furthermore, only HubLink (O) was executed on infrastructure, allowing for the direct measurement of energy consumption and carbon emissions, whereas all other approaches used the OpenAI \gls{api}. Therefore, direct sustainability measurements for operations on external \gls{api} servers are not feasible. Consequently, this difference limits the direct comparability of the sustainability figures between the locally run open-source model and the \gls{api}-based models.}

% Q7: How does the proposed approach compare with regard to the environmental impact when compared to baseline KGQA methods?
\begin{enumerate}[label={}]
    \item \textbf{Answer to \hyperref[sec:evaluation_gqm_plan]{Q7}:} \textit{Our proposed HubLink approach generally exhibits a higher Recall performance at the cost of increased environmental impact compared to the most sustainable baseline approach, \gls{difar}. While HubLink (F) achieves a Recall nearly double that of \gls{difar}, it incurs a disproportionately larger environmental cost per unit of Recall. This highlights the need for careful consideration of the trade-off between maximizing Recall and minimizing environmental footprint.}
\end{enumerate}

\section{Evaluating Answer Alignment}
\label{sec:evaluating_answer_alignment}

In this section, we present the evaluation results that address the generation targets. We begin by analyzing the semantic and factual consistency of the generated answers, which relate to the generation target \hyperref[sec:evaluation_goals_and_metrics]{\textbf{GeT1}}. Then we evaluate the relevance of the generated answers to the question and their alignment with the instructions provided in the question, which relates to \hyperref[sec:evaluation_goals_and_metrics]{\textbf{GeT2}}. Finally, we assess the consistency of the generated answers with the retrieved context, which is relevant to \hyperref[sec:evaluation_goals_and_metrics]{\textbf{GeT3}}. 

\textit{Note that unlike in the retrieval target evaluation, here only the HubLink variant (T) is evaluated. This is because the evaluation of the generated answers uses \gls{llm}-as-a-judge metrics, which incur additional costs for their computation. We have chosen to evaluate the HubLink variant that performed best in the retrieval target evaluation. Moreover, all the following evaluations are based on the graph variant \hyperref[enum:gv1]{\textbf{GV1}}, also due to cost considerations. Finally, as detailed in Section~\ref{sec:prompting_an_llm_for_final_answer_generation}, HubLink includes source citations and a corresponding reference list in the answer. Since these elements are absent from the reference answers, their presence likely penalizes similarity and precision metrics. Therefore, we do not include the reference list in the evaluation of the generated answers.}



\subsection{Semantical and Factual Consistency of Generated Answers}
\label{sec:semantical_and_factual_consistency}

\begin{table}[t]
\centering
% \resizebox{\textwidth}{!}{%
\begin{tabular}{@{}lllllll}
\toprule
Approach & FC-Reca. & FC-Prec. & FC-F1 & Bert-Reca. & Bert-Prec. & Bert-F1 \\ 
\midrule
HubLink (T) & \textbf{0.543} & \textbf{0.301} & \textbf{0.361} & 0.678 & 0.515 & 0.580 \\ 
DiFaR & 0.387 & 0.290 & 0.321 & \textbf{0.702} & 0.588 & \textbf{0.635} \\ 
Mindmap & 0.203 & 0.212 & 0.184 & 0.652 & 0.625 & 0.633 \\ 
FiDeLiS & 0.131 & 0.201 & 0.154 & 0.516 & \textbf{0.629} & 0.562 \\ 
\toprule
Approach & ROG-Reca. & ROG-Prec. & ROG-F1 & Str. Sim. & Sem. Sim. & BLEU \\ 
\midrule
HubLink (T) & \textbf{0.757} & 0.298 & 0.373 & 0.261 & 0.761 & 0.105 \\ 
DiFaR & 0.674 & 0.374 & \textbf{0.448} & \textbf{0.338} & \textbf{0.772} & \textbf{0.160} \\ 
Mindmap & 0.487 & 0.432 & 0.397 & 0.296 & 0.682 & 0.105 \\ 
FiDeLiS & 0.195 & \textbf{0.503} & 0.251 & 0.202 & 0.499 & 0.046 \\ 
\bottomrule
\end{tabular}%
% }
\caption[Results for Semantical and Factual Answer Consistency]{Evaluation results for assessing the semantic and factual consistency of generated answers. The table includes various abbreviations: FC-Reca. (Factual Correctness Recall); FC-Prec. (Factual Correctness Precision); FC-F1 (Factual Correctness F1); Bert-Prec. (BERTScore Precision); Bert-Reca. (BERTScore Recall); Bert-F1 (BERTScore F1); ROG-Reca. (ROG-1 Recall); ROG-Prec. (ROG-1 Precision); ROG-F1 (ROG-1 F1); Str. Sim. (String Similarity); Sem. Sim. (Semantic Similarity). All metrics have been macro-averaged.}
\label{tab:evaluation_correctness_of_answer}
\end{table}

Building upon the observation from \autoref{tab:q11:relevance_and_accuracy} that HubLink demonstrates superior performance in retrieving relevant triples compared to baseline \gls{kgqa} approaches, this section evaluates the semantic and factual consistency of the answers generated based on these retrieved triples. In the following, we assess the evaluation results presented in \autoref{tab:evaluation_correctness_of_answer}. We begin by analyzing the Recall and similarity metrics, followed by an examination of the precision metrics. 

\subsubsection{Assessment of Recall and Similarity}

The results in \autoref{tab:evaluation_correctness_of_answer} illustrate a notable divergence from the retrieval performance assessment. Regarding Recall for factual correctness, HubLink achieves the highest value (0.543), although the advantage over competitors is less pronounced than observed in retrieval performance. Furthermore, compared to the retrieval Recall (0.754), the factual correctness Recall of HubLink is approximately 39\% lower. This suggests limitations in preserving all retrieved facts during the answer generation process. However, this observation is contradicted by the particularly high ROUGE-1 Recall of 0.757, indicating that a substantial majority of the lexical items present in the reference answers are captured within the generated responses. Nevertheless, this does not necessarily imply that the generated answers include all the relevant facts from the retrieved triples. For example, if the generated answer includes many of the words also present in the reference answer, the ROGUE-1 Recall is high, even if the provided facts are wrong. Consequently, we conclude that HubLink does not fully retain all relevant information during answer generation.

In contrast, the baseline methods all achieved similar or higher factual correctness Recall values than their retrieval Recall. For instance, \gls{difar} achieved a factual correctness Recall value of 0.387, which closely aligns with its retrieval Recall (0.352), suggesting effective information transfer to the generation stage. Notably, Mindmap exhibits a factual correctness Recall of 0.203, largely exceeding its retrieval Recall (0.119). The same can be said for ROUGE-1 Recall, where all baseline methods achieved higher values than their retrieval Recall. 

Regarding the BERTScore Recall, the results present a different pattern. \gls{difar} leads (0.702) with a slight edge over HubLink (0.678), followed by Mindmap (0.652) and FiDeLiS (0.516). A similar trend can be observed with BLEU scores, as well as the \emph{String Similarity} and \emph{Semantic Similarity} metrics, albeit with lower absolute values for BLEU. From the results, we can observe that \gls{difar} provides answers that are most similar to the expectation. Although HubLink does provide answers that are semantically related, the lower values in string similarity suggest that the generated answers tend to diverge, possibly because of the more comprehensive answers provided by the method.

\textit{Note that the lower absolute values for BLEU likely arise because the golden answers in the \gls{kgqa} dataset were designed to be concise, only stating the facts asked for. Because BLEU measures exact n-gram overlap \cite{papineni_bleu_2001}, this heavily penalizes any deviation in phrasing, structure, or additional information present. Since the scores are low, the \glspl{llm} seem to create more verbose answers.}

\subsubsection{Assessment of Precision}

Analyzing precision for factual correctness, the score for HubLink (0.301) increased slightly over the retrieval precision (0.246). In stark contrast, baseline methods exhibit significantly higher factual correctness precision compared to their respective retrieval precision values. \gls{difar} achieves the highest factual correctness precision (0.290), surpassing HubLink despite having a very low retrieval precision (0.011). Mindmap (0.212 vs. 0.030) and FiDeLiS (0.201 vs. 0.052) show similar substantial increases from generation to retrieval precision.

However, for ROUGE-1 and BERTScore precision, HubLink records the lowest values among the evaluated methods (0.166 for ROG-Prec and 0.405 for Bert-Prec, respectively). This further highlights the structural and lexical differences between its generated answers and the reference targets.

\subsubsection{Discussion on Results:} 

Several key observations arise from the evaluation of \autoref{tab:evaluation_correctness_of_answer}. First, the finding that factual correctness sometimes exceeds observations from the retrieval metrics warrants explanation, as generated answers should ideally be constrained by retrieved information. We attribute this to the implementation of the factual correctness metric within the RAGAS evaluation framework, which appears to assign partial credit based on the granularity of the answer. An answer deemed factually incomplete might still receive a positive evaluation if some parts are correct. For example, if the answer acknowledges the existence of a publication without providing specific requested details, this contributes positively to the metric. This characteristic could contribute to higher scores of factual correctness for baseline methods relative to their retrieval performance.

Despite these limitations, valuable insights emerge. The collective results suggest that HubLink tends to generate more comprehensive, potentially overly elaborate answers compared to the reference targets. This may stem from its synthesis process, where the \gls{llm} integrates information from multiple retrieved sources, possibly leading to less concise outputs. Furthermore, the Recall scores are mediocre, suggesting that not all facts are transferred from the retrieved triples to the answer. Moreover, low precision and similarity scores indicate deviations in structure and the potential inclusion of extraneous details. Therefore, refining the integration of facts, the conciseness, and the focus of the generated answers of HubLink presents a direction for future improvement.

% Q6 How semantically and factually consistent are the generated answers of the proposed approach when compared to answers generated by baseline KGQA approaches?
\begin{enumerate}[label={}]
 \item \textbf{Answer to \hyperref[sec:evaluation_gqm_plan]{Q8}:} \textit{Compared to the baseline \gls{kgqa} approaches, the answers generated by HubLink demonstrate limitations. The inclusion of facts from the retrieved triples into the generated answer is mediocre. Furthermore, lower scores in precision and similarity suggest that answers generated by HubLink may include additional, potentially unrequested, information and differ structurally from reference answers. This points to current limitations in the inclusion of facts, semantic consistency, and conciseness of answer generation.}
\end{enumerate}


\subsection{Generation of Relevant Answers}

\begin{table}[t]
\centering
% \resizebox{\textwidth}{!}{%
\begin{tabular}{@{}lcc}
\toprule
Approach & Answer Relevancy & Instruction Following  \\ 
\midrule
HubLink (T) & \textbf{0.570} & \textbf{0.653} \\
DiFaR & 0.203 & 0.312 \\
Mindmap & 0.545 & 0.388 \\
FiDeLiS & 0.432 & 0.388 \\
\bottomrule
\end{tabular}%
% }
\caption[Results of Alignment with Intent and Content of the Question]{Evaluation results assessing the alignment of generated answers with the intent and content of the question. All metrics have been macro-averaged.}
\label{tab:q21:intent_and_content_alignment}
\end{table}

A crucial aspect of evaluating answer generation quality is determining whether the response is relevant to the posed question. \autoref{tab:q21:intent_and_content_alignment} presents the results of the \emph{answer relevancy} metric relevant to this aspect.

HubLink achieves the highest score of 0.570 in answer relevancy among the evaluated approaches. Mindmap follows closely with a score of 0.545, suggesting comparable effectiveness between these two methods in aligning generated responses with the intent of the question. FiDeLiS demonstrates moderate performance (0.432), whereas DiFaR shows considerably lower relevancy (0.203).

These findings indicate that although HubLink leads in answer relevancy relative to the baselines, its absolute score suggests that a notable portion (43\%) of its generated answers may not be optimally aligned with the question, highlighting the scope for improvement. However, it is critical to underscore that the Answer Relevancy metric assesses the perceived alignment between the question and the topic or intent of the answer, independent of the factual accuracy. Consequently, a response could be deemed relevant yet contain factual inaccuracies or hallucinations. The factual correctness aspect is specifically addressed in \autoref{tab:evaluation_correctness_of_answer}.

% Q9 To what extent do the answers generated by HubLink reflect the semantic intent of scholarly questions when compared to baseline KGQA approaches?
\begin{enumerate}[label={}]
 \item \textbf{Answer to \hyperref[sec:evaluation_gqm_plan]{Q9}:} \textit{HubLink demonstrates the strongest performance among the evaluated methods in generating answers that align with the semantic intent of scholarly questions, as measured by answer relevancy. However, its absolute performance indicates limitations, suggesting that further refinement is necessary to consistently ensure optimal semantic alignment between questions and generated answers.}
\end{enumerate}

\subsection{Following the Instructions provided in the Question}

Beyond requesting specific information, questions may include explicit instructions regarding the desired answer format or structure. The \gls{kgqa} dataset that has been used incorporates such questions derived from complex retrieval operations. This requires the retriever, for instance, to present results in a specific order or perform aggregations. The ability of each approach to comply with these requirements is evaluated using the \emph{Instruction Following} metric, with results presented in \autoref{tab:q21:intent_and_content_alignment}.

The results in \autoref{tab:q21:intent_and_content_alignment} indicate that HubLink substantially outperforms the baseline \gls{kgqa} approaches in adhering to question instructions by achieving a score of 0.653, which is approximately 68\% higher than the scores of the next-best-performing methods, Mindmap and FiDeLiS (both 0.388). DiFaR demonstrated lower performance on this metric (0.312).

Despite its relative advantage, the absolute performance of HubLink reveals limitations, as the score of 0.653 implies that the system did not fully adhere to instructions in approximately one third (34\%) of the cases. This indicates that while HubLink demonstrates a significantly stronger capability for instruction following compared to the baselines, further refinement of its generation process is warranted to improve reliability in this aspect.

% Q10 To what extent do the generated answers follow the instructional expectations of scholarly questions when compared to baseline KGQA approaches?
\begin{enumerate}[label={}] 
 \item \textbf{Answer to \hyperref[sec:evaluation_gqm_plan]{Q10}:} \textit{HubLink exhibits a significantly superior ability to follow specific instructions embedded within scholarly questions compared to the baseline \gls{kgqa} approaches evaluated. Nonetheless, its absolute performance indicates that adherence to instructions is not fully consistent, highlighting the need for further enhancements in the answer generation mechanism to ensure instructions are followed more reliably.} 
\end{enumerate}

\subsection{Consistency of the Generated Answers to the Retrieved Context}

\begin{table}[t]
\centering
% \resizebox{\textwidth}{!}{%
\begin{tabular}{@{}lc}
\toprule
Approach & Faithfulness  \\ 
\midrule
HubLink (T) & 0.445 \\
DiFaR & \textbf{0.645} \\
Mindmap & 0.396 \\
FiDeLiS & 0.112 \\

\bottomrule
\end{tabular}%
% }
\caption[Results on Answer to Context Consistency]{Evaluation results assessing how consistent the generated answer is with the retrieved contexts.}
\label{tab:evaluation_of_faithfulness}
\end{table}

A critical requirement for trustworthy answer generation, particularly when using \glspl{llm}, is to ensure that the output is strictly grounded in the retrieved context. The generated answer must refrain from introducing extraneous information, and all presented assertions should be directly verifiable against the source data. \autoref{tab:evaluation_of_faithfulness} presents the evaluation results using the \emph{Faithfulness} metric, designed to measure conformity with the retrieved context.

The data reveals that \gls{difar} achieves the highest faithfulness score (0.645), indicating strong adherence to its retrieved context. The score of HubLink (0.445) is notably lower, comparable to the performance of Mindmap (0.396), while FiDeLiS exhibits substantially lower faithfulness (0.112).

These results suggest that HubLink exhibits notable limitations in constraining its answers solely to the provided context. The lower faithfulness score of HubLink compared to \gls{difar} indicates that the latter is more effective in ensuring that generated answers are strictly grounded in the retrieved context. This confirms findings from previous generation metrics, as it underscores that the current answer generation strategy in HubLink constitutes a limitation and necessitates refinement to enhance strict factual grounding alongside overall answer quality.

% Q9: To what extent are generated answers of HubLink faithful to the retrieved context and free from unsupported claims when compared to baseline KGQA approaches?
\begin{enumerate}[label={}] 
 \item \textbf{Answer to \hyperref[sec:evaluation_gqm_plan]{Q11}:} \textit{The generated answers of HubLink demonstrate weaker grounding in the retrieved context compared to the baseline with the highest performance, \gls{difar}. Therefore, improving the faithfulness of responses to the retrieved context and minimizing potentially unsupported claims is an area that requires improvement.} 
\end{enumerate}

% As indicated by our evaluation, HubLink offers significant advantages in comprehensive information retrieval and handling complex questions within scholarly \gls{kg} retrieval for literature search. Its main challenges lie in precision, ranking refinement, and generating concise, tightly focused answers. This section discusses the key findings from the evaluation of our proposed HubLink approach (\hyperref[enum:c1]{\textbf{C1}}) focusing on the implications of the performance in retrieving information from the \gls{kg} and generating answers for the scholarly literature search task.


\section{Discussion on Evaluation Results}
\label{sec:discussion_on_evaluation_results}


In the following, we discuss the interpretation of the evaluation results concerning our proposed HubLink approach. This discussion focuses on the two primary aspects evaluated: data retrieval from the \gls{kg} and the subsequent answer generation. Furthermore, as the baseline approaches yielded notably low scores, we will discuss their results separately.

\subsection{Analysis of Retrieval Performance}

Our evaluation indicates that HubLink presents a considerable advancement in retrieving scholarly information from \glspl{rkg} compared to existing baseline methods, positioning it as a promising approach for scholarly literature searches. The consistently high recall observed across our tests underscores that the core strength of HubLink lies in its capacity to identify and retrieve a comprehensive set of relevant triples from the \gls{kg}. This capacity remains robust when HubLink is applied to diverse graph structures, showcasing the schema-agnostic characteristic of the approach. The results further demonstrate the capability of HubLink to retrieve relevant information even when questions necessitate traversing multiple hops or when information is spread across broader paths. This multi-hop retrieval capability is particularly important for scholarly literature search tasks, where questions necessitate multi-hop reasoning to connect diverse pieces of information.

The evaluation also demonstrates that HubLink can effectively handle a range of retrieval operations, including basic factual lookups and moderately complex operations such as aggregation and ranking. In addition, the results show that HubLink can operate economically during the retrieval process. Specifically, a faster configuration of the approach achieves performance comparable to that of a more complex configuration while offering the benefits of reduced runtime, lower \gls{llm} token costs, and diminished environmental impact. This efficiency is particularly relevant for real-world applications, where efficiency and cost-effectiveness are crucial considerations.

However, the evaluation also highlights areas where HubLink faces challenges. Although HubLink achieves high recall, the moderate precision and ranking scores indicate difficulty in distinguishing relevant triples from less pertinent ones within the retrieved set. This suggests that HubLink retrieves a broader set of information than required, which includes noise alongside the relevant context. This challenge becomes even more pronounced for questions that require complex logical operations, such as negation or superlatives, and for use cases involving less structured content data. These limitations suggest that, while HubLink excels at retrieving information for answering scholarly questions, downstream filtering or reranking mechanisms may be necessary to refine the results for optimal precision and ranking quality.

\subsection{Analysis of Answer Generation Performance}
% While HubLink excels in the retrieval of relevant facts, the translation into consistently high-quality answers presents challenges.

With regard to the answer generation performance, HubLink demonstrates improved capacity over baseline approaches to generate relevant answers and adhere to specific instructions embedded within the query. This suggests that HubLink captures user intent more effectively, even for complex tasks. 

However, our analysis shows that the translation of retrieved facts into consistently high-quality answers presents limitations. Our evaluations indicate that HubLink only partially incorporates the retrieved factual knowledge into its generated answers. Furthermore, the absolute recall scores for answer generation are lower than those observed during retrieval, suggesting that not all relevant information from the retrieved context is integrated into the final answers. Moreover, lower scores in precision and similarity comparisons suggest that the answers generated by HubLink tend to be less concise and may deviate structurally from the expected reference answers. A reduced faithfulness score also indicates this trend. 

Overall, the generation component exhibits a tendency towards producing comprehensive yet potentially overly verbose answers rather than consistently concise presentations of information. Therefore, an important direction for future improvement is the refinement of the generation strategy to produce more focused and concise answers while still maintaining accuracy and relevance to the context.

\subsection{Analysis of Baseline Performances}

Our experimental results reveal that the evaluated baseline methods generally exhibit substantially lower retrieval performance compared to our proposed HubLink approach. In particular, FiDeLiS and Mindmap demonstrated low performance in the retrieval experiments, while StructGPT and ToG struggled to answer any question during the parameter selection process (see \autoref{ch:parameter_selection_process}). Consequently, we did not use StructGPT and ToG in further testing. In the following subsections, we examine the specific characteristics and limitations of each baseline method to understand these performance differences.

\subsubsection{DiFaR}

Among the baseline approaches evaluated, DiFaR \cite{baek_direct_2023} achieved the best overall retrieval performance across most metrics. This approach, similar to HubLink, is based on leveraging embeddings for retrieval tasks. The relative success of DiFaR compared to the non-embedding-based baselines suggests potential advantages of embedding strategies for the types of questions and graph structures used in our experiments.

However, HubLink consistently outperforms DiFaR. This comparison highlights that, while a general embedding-based strategy is beneficial, the specific retrieval mechanisms and greater complexity introduced by HubLink yield a justifiable improvement in performance over the DiFaR approach. 

\subsubsection{Mindmap}

The Mindmap approach \cite{wen_mindmap_2024} operates by constructing evidence subgraphs. However, it encountered difficulties in our experiments, demonstrating a limited ability to answer the questions correctly. During the analysis of the algorithm, we observed a substantial limitation that renders the Mindmap approach unsuitable for scholarly literature search in our experimental setup. The following example illustrates the issue with a representative question from our dataset. The fact that Mindmap is unable to answer this question highlights its fundamental limitation, which hinders its performance on many other questions in the dataset.

Given is the following question: \enquote{Who are the authors of the paper 'A Taxonomy of Blockchain-Based Systems for Architecture Design'?}. When processing this question, Mindmap first extracts entities from the query. A plausible extraction would yield the terms \emph{Authors}, \emph{Paper}, and the title \emph{A Taxonomy of Blockchain-Based Systems for Architecture Design}. After this term recognition process, the approach queries the graph to collect entities that match these terms with the highest similarity scores. These entities are then used to build evidence subgraphs.

Mindmap constructs two types of evidence subgraphs. The first is the path-based subgraph, which finds the shortest paths between the identified entity nodes and stores them. The second is the neighbor-based subgraph, which collects all one-hop neighbors for each identified entity. However, in our example, only one term in the question corresponds directly to an entity in the graph with a meaningful match: the title of the publication. The other two extracted terms (Authors, Paper) represent semantic types or concepts rather than specific entity nodes within the graph structure, so no corresponding entities that are meaningful are found. Consequently, the only way for Mindmap to answer the question correctly would be if the authors were stored as immediate neighbors of the entity containing the title of the publication. However, in the \gls{orkg} this is not the case, as the triples of the authors are stored deeper in the graph and are therefore not collected by the retrieval.

Consequently, Mindmap performs poorly in our experiments because many questions in the \gls{kgqa} dataset provide only a single known entity and ask for another unknown entity. To be able to correctly answer these questions, it is required to find the path from the known entity to the unknown entity, which the Mindmap approach is unable to do since it builds paths only between entities provided in the question. This highlights a fundamental limitation of the Mindmap retriever and explains why many questions in the \gls{kgqa} dataset have not been answered correctly. 


\subsubsection{Beam Search Retrievers}
The StructGPT \cite{jiang_structgpt_2023}, ToG \cite{sun_think--graph_2024}, and FiDeLiS \cite{sui_fidelis_2024} approaches all rely on the beam search algorithm for graph exploration. Our experiments suggest that these approaches perform poorly on the \gls{kgqa} dataset. During the analysis of these algorithms, we discovered two major issues that explain the low performance, which are discussed below.

\paragraph{Local Information Deficit:} The beam search algorithm explores the knowledge graph by iteratively expanding a limited set of the most promising nodes or paths up to a predefined depth. A key challenge arises from its decision-making process: the choice of which paths to retain in the beam at each step is based on local information associated with the current nodes or their immediate neighbors. If the relevance of a path towards the final answer is not apparent from this local context, the path may be pruned, even if further exploration along that path would eventually lead to the correct answer.

To illustrate, consider the question \enquote{Which papers have used an interview as a method for their evaluation?}. A potentially relevant path might be structured as:
\[
\textit{Paper Title Node} \xrightarrow{\text{hasContribution}} \textit{Contribution Node} \xrightarrow{\text{usesMethod}} \textit{'Interview' Node}
\]

When the beam search exploration reaches a specific \textit{Paper Title Node}, it must decide whether to keep exploring paths originating from it based primarily on information available at that node. The title itself, or even the immediate 1-hop neighbors, may provide insufficient evidence that this specific path will lead to the target \textit{'Interview' node} when traversing further. As suggested by our analysis of the number of hops in Section~\ref{sec:evaluation_robustness_to_structural}, this issue is even more pronounced for questions that require more than three hops to arrive at the answer. Because most of the questions in our experiment require a larger number of hops, this exaggerates the issue, leading to the premature pruning of relevant paths and degrading the overall performance of these retrievers.

\paragraph{Stochastic Selection:} The second challenge comes from the sheer breadth of the graph. Because only a limited number of entities can be expanded at each iteration, the selection of the most relevant candidates is critical. StructGPT and ToG rely on an \gls{llm} to classify the relevance of predicates and then consider entities connected via the selected predicates for the next beam exploration. If this set of candidate entities exceeds the predefined beam width threshold, the approaches randomly prune the entities to reduce the size of the candidate set. This introduces non-determinism and the risk of discarding correct entities or paths purely by chance.

FiDeLiS addresses this challenge by using embedding similarity to score and select entities instead of random sampling. This results in more deterministic and often more relevant entity selection, contributing to its generally better performance compared to StructGPT and ToG in our results. However, FiDeLiS remains constrained by the first limitation, which hinders its effectiveness on questions that require deeper graph traversal where relevance is not immediately apparent.



% Old Discussion on retrieval, more of a summary:

% The evaluation on the retrieval performance consistently demonstrate that HubLink significantly improves the relevance and accuracy of retrieved knowledge from the \gls{kg} compared to baseline \gls{kgqa} approaches. A key strength of HubLink is observed in its recall capability. The results show, that HubLink consistently outperforms all baseline methods in retrieving a larger proportion of the ground truth triples (see  \autoref{tab:q11:relevance_and_accuracy}). This superior recall suggest, that HubLink exhibits a substantially broader coverage of potentially relevant information for answering scholarly questions.

% Furthermore, analysis across different retrieval operations (see \autoref{tab:q12:retrieval_operation}) confirms that HubLink can handle the various operations outlined in the \gls{kgqa} retrieval taxonomy. It performs exceptionally well on basic factual lookup and demonstrates competence in moderately complex operations such as aggregation, counting, and ranking. However, the results also highlight limitations, particularly with operations that require more complex logical reasoning, such as negation, relationship identification, and superlative evaluation. Here, the performance degrades noticeably, especially in precision.

% Regarding the applicability to scholarly literature search tasks, the evaluation across six different use cases (see \autoref{tab:q13:use_cases}) indicates that HubLink effectively retrieves desired information in various scenarios, encompassing metadata-based, content-based and mixed questions. These results underscore the potential utility of the approach in supporting researchers with different search needs. Retrieval performance remains robust in terms of recall in these use cases, although precision and ranking effectiveness vary, with metadata-centric tasks yielding the best results.

% The investigation of the impact of explicit type information in the questions (see \autoref{tab:q14:semi_typed}) reveals only a marginal positive effect on performance. This suggests that HubLink is insensitive to lexical variations of terms and explicit type cues.

% Concerning the robustness to graph structure variability (see \autoref{tab:q5:different_graph_variants}), the evaluations demonstrate that while HubLink operates in a schema-agnostic manner, its performance is influenced by the underlying graph topology. The data suggests that shorter path lengths inside of the hub structures correlate with performance gains, particularly for precision and ranking metrics. Further analysis of the hop count (see \autoref{tab:q5:hops_results}) further reinforces the robustness of HubLink compared to the baselines, especially for information located deeper within the graph. The results show that baseline methods struggle significantly as the hop count increases, but HubLink maintains a comparatively high recall. This showcases the capability of HubLink in multi-hop reasoning tasks.

% Despite the clear advantages in the recall and robustness of HubLink, the evaluation also points to limitations in the assessment of context relevance. The precision values, although superior to baselines, remain relatively low across most experiments. Similarly, ranking metrics indicate that HubLink does not always prioritize the most relevant triples at the top ranks, especially for complex questions or use cases that involve content retrieval. This evaluation suggests that while HubLink successfully retrieves a comprehensive set of candidates, refining the filtering and ranking of these candidates remains an area for improvement.


\section{Threats to Validity}
\label{sec:general_threats_to_validity}

In this section, we discuss the threats to validity for our experiment results. For this discussion, we are using the descriptions and checklists provided by \textcite[131-140]{wohlin_experimentation_2024} who propose to discuss the concepts \emph{Conclusion Validity}, \emph{Internal Validity}, \emph{Construct Validity}, and \emph{External Validity}. However, their checklist applies to experiments with human subjects, which is not the case for our experiments. Consequently, we only include the points that are relevant for our experimental setup. In addition, we also include the concepts \emph{Credibility}, \emph{Dependability}, and \emph{Confirmability} proposed by \textcite{feldt_validity_2010}.

\subsection{Conclusion Validity}

The threats to conclusion validity are concerned with issues that affect the ability to draw the correct conclusion about the relations between the treatment and the outcome of an experiment.

\emph{Reliability of Measures} describes that the outcome of a measurement should be the same for equal inputs. In our experiments, we use both traditional metrics and \gls{llm}-as-a-Judge metrics. For the \gls{llm}-based metrics, there is no guarantee that they will always produce the same evaluation for identical inputs. To mitigate this issue, we employ RAGAS \cite{es_ragas_2023}, a specialized evaluation framework for \gls{llm}-as-a-Judge metrics, because one of the main objectives of the framework is to enhance the reliability and reproducibility of these metrics.

\emph{Reliability of Treatment Implementation} considers whether the treatments are applied correctly. We developed the \gls{sqa} framework, which allows us to maintain consistent configurations while selectively varying the treatments for the experiments. Consequently, we do not see any issues with the implementation of the treatments.

\emph{Random Irrelevancies in the Experimental Setting} are concerned with random elements outside the experimental setting that disturb the results. For our experiments, we use a server provided by the institute that is shared among several users. This shared usage may introduce random disturbances, such as variations in execution time when others place a high load on the server during our experiments. To mitigate this, we verify that the server is not under load before starting the experiments.


\subsection{Internal Validity}

Threats to internal validity are those influences that can affect dependent variables with respect to causality without the knowledge of the researcher. As such, they threaten the conclusion about a possible causal relationship between the treatment and the outcome. From the checklist provided by \textcite[133-134]{wohlin_experimentation_2024}, we only see instrumentation as relevant for our experiments.

\emph{Instrumentation} is about considering the quality of the artifacts used for the execution of the experiment that may negatively affect the results. To realize the execution of the experiments, we have implemented the \gls{sqa} framework and adapted baseline retrievers based on their descriptions provided by the authors and the available code to work with our framework. As such, there is a risk that if the implementations are poorly designed and executed, the results of the experiments are negatively affected. To mitigate this risk, the \gls{sqa} framework has undergone an architectural review and two rounds of code reviews with domain experts. Furthermore, the implementations of the baseline retrievers have been done with minimal changes to the original code (see Section~\ref{sec:implementation_baselines}).

\subsection{Construct Validity}
Construct validity ensures that the metrics and methods that we have used accurately capture the intended evaluation constructs that we outline in Section~\ref{sec:exp_prelim_evaluation_framework}. The following points from the checklist provided by \textcite[146-137]{wohlin_experimentation_2024} are relevant to our experiments.

\emph{Inadequate Pre-operational Explication of Constructs} relates to the issue that the constructs are not sufficiently described before they are translated into measures or treatments. We do not see a risk of inadequate pre-operational explication of the constructs because the constructs that we are evaluating are based on the evaluation framework \gls{rgar} \cite{yu_evaluation_2024} and multiple surveys about \gls{rag} evaluation (see Section~\ref{sec:fundamentals_evaluation_rag}).

\emph{Mono-Operation Bias} is concerned with the underrepresentation of constructs due to a singular independent variable, case, subject, or treatment. Although this singularity is not the case for our experiments, we still see a considerable threat of the underrepresentation of constructs. This is because we have only included a subset of the possible configurations for each retriever. However, this was necessary to keep the experiments within a reasonable scope. 

\emph{Mono-Method Bias} is concerned with the risks of using only one type of measure or observation, which can become an issue if measurement bias occurs. However, as we are using established metrics (e.g., recall, precision) in the field and have tested them prior to the experimentation, we do not see the need to conduct multiple measurements for the same constructs. There is a small risk that the measurement of the faithfulness and relevance of the generated answers is underrepresented. However, we believe that the metrics chosen from the RAGAS framework are representative of the constructs. Furthermore, regarding singular observations, we are only performing each experiment once. Hence we only have a singular observation for each treatment. However, because we expect the results to be mostly consistent across multiple runs, we do not see this as a risk to our experiments.

\emph{Interaction of Different Treatments} describes the risk of having one subject participate in more than one study, which could lead to a treatment interaction. In our experiments, we are not using human subjects, hence this is not a risk for our experiments. However, there is a risk that the results of the experiments are affected by the interaction of different treatments. This is because we are using the \gls{ofat} method to evaluate the effect of each factor on the outcome. However, this was necessary to keep the experiments within a reasonable scope. To reduce this risk, we have carefully considered the parameters of the retrievers that interact with each other.

\emph{Restricted Generalizability Across Constructs} is about treatments that positively affect one construct but unintentionally negatively affect another construct. We do not see this as a risk for our experiments because the constructs are all evaluated at the same time, which makes it possible to see the trade-offs of treatments on each of the constructs. 

\subsection{External Validity}

The external validity ensures that the results of the experiment can be generalized beyond the experimental setting.

\emph{Interaction of Selection and Treatment} is concerned with the effect that the subject population that is used does not represent the population of interest. In our experiments, there is a risk that the questions in our \gls{kgqa} dataset do not represent actual questions of interest that a researcher would ask. To mitigate this risk, we have generated the questions based on a question taxonomy of desired question types and six use cases for the literature research task.

\emph{Interaction of Setting and Treatment} describes the issue of not using an experimental setting or tools that are representative of the real world. We mitigated this risk by developing the \gls{sqa} framework according to the state-of-the-art approach applied in\gls{qa} systems, which is the \gls{rag} approach. Furthermore, we researched common evaluation metrics for \gls{rag} systems and applied them to our experiments using the formulas and implementations provided.

\subsection{Credibility}

The credibility describes whether there is a risk that the results of the experiments are not true. We do not see a risk of credibility in our experiments. The experiments are carried out in a unified framework, the \gls{sqa} framework, which ensures that all treatments are applied under consistent conditions. This is further achieved by maintaining identical experimental settings and also by using the same hardware and software environment for all experiments. This makes it highly likely that any variance in outcomes is attributable to treatments rather than uncontrolled external factors.

\subsection{Dependability}

Dependability concerns the risk that the results of the experiments are not repeatable. There is a risk that the results of the experiments cannot be reproduced exactly, as we are working with \glspl{llm}, which are inherently non-deterministic. However, we expect the performances of the \gls{kgqa} approaches to be similar across multiple runs and the overall trends to be consistent. To allow for high reproducibility, the \gls{sqa} framework is able to exactly reproduce the same experimental setting, reducing the risk to only the randomness of the \glspl{llm}.

\subsection{Confirmability}

The confirmability is concerned with the risk that the results of the experiments are not based on the data but on the bias of the researcher. Based on the evaluation of the experimental results, we acknowledge the risk of bias in the interpretation of the results. This is because the interpretation of the results has been made by the understanding of the author of each of the constructs and their metrics. However, we mitigate this risk by clearly presenting the results in multiple diagrams, tables, and in a replication package. In addition, we thoroughly discuss the results. This allows the reader to draw their own conclusions. 

Furthermore, the baselines have been chosen by the author of this paper and there is a risk that the selected retrievers are not representative of the state of the art in \gls{qa} systems for the literature research task. To mitigate this risk, we have carefully reviewed the most recent surveys \cite{yani_challenges_2021,procko_graph_2024,agrawal_can_2024,peng_graph_2024,li_survey_2024,pan_unifying_2024} on \gls{kgqa} and selected retrievers that were applicable for our task. More details on the selection of the retrievers can be found in Section~\ref{sec:implementation_baselines}.




% "Experiment is a controlled type of study where the objective commonly is to compare two or more alternatives. A hypothesis is formulated and the researcher would like to be able to show a cause and effect relationship based on the treatments provided to the participants." \cite{wohlin_experimentation_2024}

% "An experiment is a formal, rigorous, and controlled investigation. In an experiment the key factors are identified and manipulated, while other factors in the context are kept unchanged" \cite{wohlin_experimentation_2024}

% "Experiments are foremost quantitative since they have a focus on measuring different variables, changing them, and measuring them again. During these investigations quantitative data is collected and then statistical methods are applied." \cite{wohlin_experimentation_2024}

% "Definition 6.1 An experiment (or controlled experiment) in software engineering is an empirical enquiry that manipulates one factor or variable of the studied setting. Based on randomization, different treatments are applied to or by different subjects, while keeping other variables constant, and measuring the effects on the outcome variables. In human-oriented experiments, humans apply different treatments to objects, while in technology-oriented experiments, different technical treatments are applied to objects." \cite{wohlin_experimentation_2024}

% "The objective is to manipulate one or more variables and control all other variables at fixed levels. The effect of the manipulation is measured, and based on this a statistical analysis can be performed." \cite{wohlin_experimentation_2024}

% "Those variables that we want to study to see the effect of the changes in the independent variables are called dependent variables (or response variables). Often there is only one dependent variable in an experiment. All variables in a process that are manipulated and controlled are called independent variables." \cite{wohlin_experimentation_2024}

% KB Variante:
% KB v1: 
% KB v2: 
% KB v3:
% KB v4:  

% AKTUELLE PLANNUNG 
% ---------------
% 1. Experiment

% Tuning-Metrik: Recall bzw. F2 (unterschiedliche Gewichtung Recall und Precision)

% - Knowledge Base: Hierfür das Annotated Dataset im ORKG verwenden [ICSA2022] -> TODO: v3 (?)
% -- Link zur Knowledge Base: https://gitlab.com/software-engineering-meta-research/ak-theses/mastertheses/ma-marco-schneider/implementation/-/blob/experiments/sqa-system/data/external/merged_ecsa_icsa.json?ref_type=heads
% TODO Graph

% - Erstellen des KGQA Dataset auf diesem Graphen anhand der Matrix 
% -- Fragen: Matrix im Onenote -> https://1drv.ms/o/c/64993a62c00d7835/Eqir56XA5pNClx4UMo2GURoB5tZoYijw9ms4DsUJOu6vXA?e=477yKA
% -- Generation Notebook -> https://gitlab.com/software-engineering-meta-research/ak-theses/mastertheses/ma-marco-schneider/implementation/-/blob/experiments/sqa-system/experiments/qa_dataset_generation/annotations_graph/qa_generation.ipynb?ref_type=heads
% -- Final QA Dataset.csv -> https://gitlab.com/software-engineering-meta-research/ak-theses/mastertheses/ma-marco-schneider/implementation/-/blob/experiments/sqa-system/experiments/qa_dataset_generation/annotations_graph/final_qa_dataset.csv?ref_type=heads

% - Festlegen der Base-Konfigurationen und die Tuning-Spaces fest
% -- Configurationen im Wiki -> https://gitlab.com/software-engineering-meta-research/ak-theses/mastertheses/ma-marco-schneider/implementation/-/wikis/pages/experiments/experiment_1_configs (X)

% - Dann Ausführung des Experiments: 
% -- Der ordner wo das erste Experiment stattfindet: https://gitlab.com/software-engineering-meta-research/ak-theses/mastertheses/ma-marco-schneider/implementation/-/tree/experiments/sqa-system/experiments/1_experiment?ref_type=heads
% -- (7 Fragen aus Use Case 1?) Zwischenergebnis: https://gitlab.com/software-engineering-meta-research/ak-theses/mastertheses/ma-marco-schneider/implementation/-/tree/experiments/sqa-system/experiments/1_experiment/runs/llm_tests?ref_type=heads
%Testfragen: https://gitlab.com/software-engineering-meta-research/ak-theses/mastertheses/ma-marco-schneider/implementation/-/blob/experiments/sqa-system/experiments/1_experiment/runs/llm_tests/test_qa_dataset.csv?ref_type=heads
%% TODO: Warum Testfragen?
% Ergebnis: https://gitlab.com/software-engineering-meta-research/ak-theses/mastertheses/ma-marco-schneider/implementation/-/blob/experiments/sqa-system/experiments/1_experiment/runs/llm_tests/visualization/average_metrics_per_config/average_Retrieval_metrics_per_config_part_2.pdf?ref_type=heads
%% 
% - Dann alle Retriever (Hublink und 5 Baselines) ausführen und die "Beste Konfiguration" bestimmen anhand von F2
%% TODO: Systematisches Vorgehen nach [Quelle]
% - Jetzt die besten Konfigurationen zusammenbauen und dann erneut das Experiment ausführen. Die Ergebnisse dann analysieren
% - Hypothese: The proposed retrieval approach (HubLink) improves the accuracy of answering SWA literature research-related questions in a QA system compared to baseline retrieval methods when operating on a sparse RKG.
% - H1: The HubLink retriever improves the accuracy and relevance of the retrieved Knowledge Graph (KG) facts compared to baseline retrievers
% - H2: The HubLink retriever improves the faithfulness, correctness, and relevance of the generated answers compared to baseline retrievers.
% - H3: The HubLink retriever can answer a wider range of desired question types.
%% TODO: xx
% - H4 The runtime of the HubLink retriever is comparable to baseline retrievers.
% - H5 The environmental and monetary costs of the HubLink retriever are comparableto baseline retrievers.
%% TODO: COmbined Metrics


% ---------------
% 2. Experiment
% - Jetzt variieren wir die Struktur der Contributions um herauszufinden, was es für eine Auswirkung hat, dass die Informationen tiefer im Graph liegen oder wenn mehr Pfade traversiert werden müssen um zu den Informationen zu gelangen.

% - Validity Threat: Wir tunen nur auf Variante 3 und nicht auf alle Varianten, um die Auführungszeit und die Kosten im Rahmen zu halten. Denoch, durch das Fix halten der Parameter der Retriever und das Variieren des Graphen lässt sich prüfen, welche auswirkungen

% - Varianten die wir probieren:
% -- 1. Graph: Tiefe Hierarchy, alle Annotated Daten in einer Contribution (Classifcations 2)
% -- 2. Graph: Flache Hierarchy, alle Annotated Daten in einer Contribution (Classification 1)
% -- 3. Graph: Tiefe Hierarchy, Annotated Daten semantisch aufgeteilt auf verschiedene Contributions (Die Contributions mit markierung 2)
% -- 4. Graph: Flache Hierarchy, Annotated Daten semantisch aufgeteilt auf verschiedene Contributions (Die Contributions mit markierung 1)
% - Comparison:
% -- Hublink vs. Baselines für jeden Graphen
% - Hypotheses: Varying the structural template of the ORKG graph (deep vs. shallow hierarchy and aggregated vs. semantically separated contributions) does not have a significant impact on the quality of answers produced by HubLink.
% - Potentielles Problem: Durch die Veränderung des Graphen wird die HOP Anzahl die benötigt wird für eine Frage im QA Dataset verändert. Außerdem wird es passieren das "Goldene Triple" im QA Dataset nicht mehr richtig sind. 
% - Lösung: Ich muss jede Frage manuell überprüfen und anpassen.
% ---------------

% 3. Experiment
% - Statt auf den Annotationsdaten zu arbeiten, testen wir jetzt Sentence-Based Daten.

% - 1. Extrahieren von Sentence-Based Daten aus den Volltexten der Publikationen
% - 2. Hochladen in den ORKG mit einer neuen Contribution
% - 3. Erstellen eines neuen QA Datensets
% - 4. Ausführen der besten Konfigurationen (aus Experiment 1) auf den Volltext Daten
    
% - Hypothese: When fulltext data is added to the ORKG graph (resulting in a denser RKG), the proposed retrieval approach (HubLink) outperforms baseline retrieval methods in accurately answering literature research–related questions.



% --------------------------
% EXPERIMENTE

% 1. Experiment
% - Recall als Tuning Metric (?)
% - Annotated und Metadaten 
% - KGQA Dataset auf diesem Graphen (ca. 30-50 Fragen)
% - Baseline Konfiguration festlegen für HubLink und Baseline Retriever
% - Base Configs auf allen vier Graphen ausführen
% - Beste Graph-Config für jede Baseline festlegen
% - Baselines und HubLink tunen für die jeweilig beste Graph-Config
% - Nochmal auf allen Graph-Configs mit dem besten Tuning alle Baselines und HubLink ausführen
% - Jetzt kann man für jede Graph-Config die Performance von HubLink mit den Baselines vergleichen
% - Hypotheses:
% -- Performance and Structure
% --- The HubLink retriever improves the accuracy and relevance of the retrieved Knowledge Graph (KG) facts compared to baseline retrievers.
% --- The HubLink retriever improves the faithfulness, correctness, and relevance of the generated answers compared to baseline retrievers.
% -- Relevance
% --- The HubLink retriever can answer a wider range of desired question types.
% -- Efficiency
% --- The runtime of the HubLink retriever is comparable to baseline retrievers.
% --- The environmental and monetary costs of the HubLink retriever are comparable to baseline retrievers.

% ---------------

% 2. Experiment
% - Jetzt lade ich zusätzlich die Fulltext Daten in den ORKG Graphen
% - Dann erstelle ich ein neues KGQA Dataset auf diesem Graphen (ca. 100 Fragen)
% - Um die Experimente im Rahmen zu halten, verwende ich die "besten Konfigurationen" aus Experiment 1
% - Comparison:
% -- Die Hublink Configurationen untereinander
% -- Die Baseline Configurationen untereinander
% -- Die beste Hublink Configuration vs. die beste Baseline Configurationen
% - Hypothese: When fulltext data is added to the ORKG graph (resulting in a denser RKG), the proposed retrieval approach (HubLink) outperforms baseline retrieval methods in accurately answering literature research–related questions.

% ---------------

% 3. Experiment
% We refer to this experiment as "Document-based", because our intention is to evaluate "How well the HubLink retriever is able to retrieve relevant document passages from a document for a given question", as opposed to the "Knowledge Graph based" experiments, where we evaluate "How well the HubLink retriever is able to retrieve relevant statements from a knowledge graph for a given question".
% - Jetzt testen wir, wie gut HubLink + KARAGEN in der Lage sind relevante Textabschnitte in Publikationen im Vergleich zu State-of-the-Art Document Retrieval Approaches zu finden.
% - Dafür nutzen wir das gleiche QA Dataset wie in Experiment 3
% - Die HubLink Ausführung aus Experiment 3 kann wiederverwendet werden
% - Ausführen von den Document-based Baselines
% - Comparison:
% -- Die Baseline Configurationen untereinander
% -- Hublink vs. Baselines
% - Hypothese: The combination of HubLink and KARAGEN retrieves relevant text passages in publications more effectively than state-of-the-art document retrieval approaches.


% ALT
% #################################################################



% ---------------
% 1. Experiment
% -> Die Realistische Contribution nehmen -> Distributed Deep Variante -> Weil besser Erweiterbar und durch Tiefe mehr Semantischer Zusammenhang entsteht
% - Beim ersten Experiment lade ich nur die Annotated und Metadaten in den ORKG Graphen.
% - Dann erstelle ich das KGQA Dataset auf diesem Graphen (ca. 30-50 Fragen)
% - Dann HubLink ausführen und "die beste Konfiguration" finden
% - Dann die Baselines ausführen und auch hier die "beste Konfiguration" finden
% - Comparison:
% -- Die Hublink Configurationen untereinander
% -- Die Baseline Configurationen untereinander
% -- Die beste Hublink Configuration vs. die beste Baseline Configurationen
% - Hypothese: The proposed retrieval approach (HubLink) improves the accuracy of answering SWA literature research-related questions in a QA system compared to baseline retrieval methods when operating on a sparse RKG.
% ---------------
% 2. Experiment
% - Jetzt variiere ich das Template des ORKG Graphen und teste mit den Konfigurationen von Experiment 1 die Retriever auf dem Graphen
% -- 1. Graph: Tiefe Hierarchy, alle Annotated Daten in einer Contribution
% -- 2. Graph: Flache Hierarchy, alle Annotated Daten in einer Contribution
% -- 3. Graph: Tiefe Hierarchy, Annotated Daten semantisch aufgeteilt auf verschiedene Contributions
% -- 4. Graph: Flache Hierarchy, Annotated Daten semantisch aufgeteilt auf verschiedene Contributions
% - Comparison:
% -- Hublink vs. Baselines für jeden Graphen
% - Hypotheses: Varying the structural template of the ORKG graph (deep vs. shallow hierarchy and aggregated vs. semantically separated contributions) does not have a significant impact on the quality of answers produced by HubLink.
% - Potentielles Problem: Durch die Veränderung des Graphen wird die HOP Anzahl die benötigt wird für eine Frage im QA Dataset verändert. Außerdem wird es passieren das "Goldene Triple" im QA Dataset nicht mehr richtig sind. 
% - Lösung: Ich muss jede Frage manuell überprüfen und anpassen.
% ---------------
% 3. Experiment
% - Jetzt lade ich zusätzlich die Fulltext Daten in den ORKG Graphen
% - Dann erstelle ich ein neues KGQA Dataset auf diesem Graphen (ca. 100 Fragen)
% - Um die Experimente im Rahmen zu halten, verwende ich die "besten Konfigurationen" aus Experiment 1
% - Comparison:
% -- Die Hublink Configurationen untereinander
% -- Die Baseline Configurationen untereinander
% -- Die beste Hublink Configuration vs. die beste Baseline Configurationen
% - Hypothese: When fulltext data is added to the ORKG graph (resulting in a denser RKG), the proposed retrieval approach (HubLink) outperforms baseline retrieval methods in accurately answering literature research–related questions.
% ---------------
% 4. Experiment
% We refer to this experiment as "Document-based", because our intention is to evaluate "How well the HubLink retriever is able to retrieve relevant document passages from a document for a given question", as opposed to the "Knowledge Graph based" experiments, where we evaluate "How well the HubLink retriever is able to retrieve relevant statements from a knowledge graph for a given question".
% - Jetzt testen wir, wie gut HubLink + KARAGEN in der Lage sind relevante Textabschnitte in Publikationen im Vergleich zu State-of-the-Art Document Retrieval Approaches zu finden.
% - Dafür nutzen wir das gleiche QA Dataset wie in Experiment 3
% - Die HubLink Ausführung aus Experiment 3 kann wiederverwendet werden
% - Ausführen von den Document-based Baselines
% - Comparison:
% -- Die Baseline Configurationen untereinander
% -- Hublink vs. Baselines
% - Hypothese: The combination of HubLink and KARAGEN retrieves relevant text passages in publications more effectively than state-of-the-art document retrieval approaches.

