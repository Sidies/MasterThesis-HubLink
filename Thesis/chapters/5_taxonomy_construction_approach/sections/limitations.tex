
\section{Limitations}
\label{sec:tax_proce_limitations}

In this section, we acknowledge several limitations of our proposed taxonomy construction approach. 

First, several phases involve manual effort and qualitative judgments, which can introduce subjectivity and potentially affect reproducibility. For example, the extraction of classes, the deduplication and categorization steps within the \textsc{Clustering} phase, and the \textsc{Relevance Assessment} depend on human interpretation and the definition of the guiding criteria. The iterative nature of refinement also relies on judgments regarding when a satisfactory state is achieved, which can influence the overall duration and outcome of the process.

Second, the effectiveness of the \textsc{Application} phase as a final validation step is dependent on the selection and diversity of the use cases. If the chosen scenarios do not adequately represent the intended scope of the taxonomy, some of its deficiencies might not be identified. As a consequence, the generalizability of the validation results from this phase is tied to the representativeness of these application instances.

Finally, the initialization of the taxonomy is based on a literature survey. This foundation means that the process is dependent on the availability, quality, and scope of the existing literature. The comprehensiveness and representativeness of the resulting taxonomy are therefore contingent upon the body of literature identified and processed. For example, if the literature in a specific domain is sparse, exhibits bias, or is of inconsistent quality, this will inevitably impact the foundation upon which the taxonomy is built. Furthermore, the selection of seed publications and the execution of searches using Google Scholar may introduce selection bias if not carefully considered and justified.