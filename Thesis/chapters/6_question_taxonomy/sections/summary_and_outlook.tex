
\section{Summary and Outlook}

Our final taxonomy consists of eight categories, each highlighting a relevant aspect of retrieval in a \gls{kgqa} system designed for scholarly literature search. These categories serve as guidelines for the formulation of questions and help to assess the capabilities of \gls{kgqa} systems. In summary, the taxonomy provides a granular framework to classify questions aimed at a \gls{kgqa} system in the literature search task. The category \emph{Retrieval Operation} focuses on the specific operations a retriever must perform on the data to generate answers. This category is particularly informative, as the difficulty levels of the included classes vary significantly. It allows for precise assessments, such as whether a retriever can identify relationships between information in the graph or sort quantitative and qualitative data. The categories \emph{Answer Type} and \emph{Condition Type}, evaluate the proficiency of the retriever in handling various data types, such as numeric constraints or textual descriptions. Furthermore, the category \emph{Graph Representation} assesses the complexity of graph patterns the retriever can process, distinguishing between the retrieval of single or multiple triples. Additional categories such as \emph{Answer Format}, \emph{Intention Count}, \emph{Question Goal}, and \emph{Answer Credibility} further support a comprehensive evaluation of retriever capabilities. They address aspects like the format of the answers, the handling of multipart questions, the objective of the literature search, and the transparent treatment of varying levels of truth.

We use the taxonomy in \autoref{ch:experimentation_preliminaries} to prepare a \gls{kgqa} dataset that we subsequently apply in \autoref{ch:experimentation} to evaluate our novel \gls{kgqa} retrieval approach.