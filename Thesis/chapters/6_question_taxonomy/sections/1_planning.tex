
\section{Planning}
\label{sec:taxonomy_planning}

According to the construction approach, the planning phase should include five different steps that guide the objective and intended use of the proposed taxonomy (see Section~\ref{sec:tax_dev_planning}). In the following, we will describe each of the individual steps.

\hyperref[enum:step1]{\textbf{S1:}} The main objective of the taxonomy is to \emph{classify the characteristics of questions posed to \gls{kgqa} systems for scholarly literature search tasks to facilitate the creation of diverse sets of questions that test a broad range of retrieval capabilities.}

\hyperref[enum:step2]{\textbf{S2:}} The taxonomy is designed to be utilized specifically within the domain of scholarly data that is stored within an \emph{\gls{rkg}}.

\hyperref[enum:step3]{\textbf{S3:}} The classification structure type selected for the taxonomy is a \emph{facet-based} structure. It defines multiple distinct categories, where each has its own set of classes. The advantage of such a structure is that it can easily be adapted to changes in the classification requirements.

\hyperref[enum:step4]{\textbf{S4:}} The classification procedure type is \emph{qualitative} since it is based on nominal scales.

\hyperref[enum:step5]{\textbf{S5:}} The taxonomy consists of question classification types drawn from the literature. For the literature survey, we have defined the following categories as relevant: The first category is \emph{Research Questions}. In this category, the literature provides information on questions relevant to the research process of a practitioner. The second category is \emph{Question Classifier} since it includes literature that provides a model or approach to classify questions according to a taxonomy. As these publications are likely to provide information about the types that the classifier is classifying, this category is expected to provide valuable insights. Finally, the \emph{Knowledge Graph Question Answering Dataset} category includes literature proposing a \gls{qa} dataset specifically for \glspl{kg}. This is particularly useful for our use case as it has the potential to provide \gls{kg}-specific information.
