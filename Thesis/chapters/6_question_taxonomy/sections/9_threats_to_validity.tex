
\section{Threats to Validity}
\label{sec:taxonomy_threats_to_validity}

In the following section, we discuss the threats to validity of the second contribution \hyperref[enum:c2]{\textbf{C2}} (the \gls{kgqa} retrieval taxonomy). This extends the limitations of the taxonomy construction methodology, which we discuss in Section~\ref{sec:tax_proce_limitations}. The structure of the subsequent threat evaluation follows the classification framework suggested by \textcite{runeson_guidelines_2009}, which encompasses the concepts of construct validity, internal validity, external validity, and reliability.

\paragraph{Construct Validity}
Construct validity addresses whether the measurements applied truly represent the evaluation objectives. Fundamentally, during the validation of the taxonomy, we employed a \gls{gqm} plan to minimize the risk that the created increments of the taxonomy fail to align with the defined goals (\hyperref[tab:gqm_taxonomy_validation]{\textbf{G1 and G2}}). To facilitate this, we used the evaluation method from \textcite{kaplan_introducing_2022}, which provides a comprehensive framework for the validation of taxonomies. Additionally, we interpreted the results of the evaluation goals not just based on individual metrics but also critically questioned the results to consider the impact on the overall objective. This critical evaluation aimed to reduce the risk of drawing conclusions inconsistent with the intended goals based solely on metric values.

\paragraph{Internal Validity}
Internal validity examines whether the results depend on other factors or are solely attributable to the observed factors. The primary threat to validity here is the selection of literature used to create the taxonomy. There is a risk that insufficient literature was considered and that adding more literature could have introduced additional classes for the taxonomy creation. This threat was mitigated through several measures: 1) supplementing the initial seed-based search with broader queries on academic search engines like Google Scholar, 2) categorizing identified papers by domain and topic to ensure a diverse input, and 3) thoroughly documenting the entire search process and outcomes in structured artifacts (e.g., JSON files), enhancing transparency and facilitating future extensions.

Another threat is the inclusion of literature beyond the primary domain of \gls{kgqa}. We argue that this broad inclusion was necessary because we did not find sufficient resources within the \gls{kgqa} field for a broad classification. This introduces the risk of incorporating concepts or structures irrelevant to the specific application context of \gls{kgqa} retrieval. This risk was primarily addressed during the \textsc{Relevance Assessment} phase, where the extracted and clustered concepts were explicitly evaluated against the defined objectives and scope of the \gls{kgqa} retrieval taxonomy. Consequently, concepts deemed insufficiently relevant to this specific context were excluded.


\paragraph{External Validity}
External validity concerns the generalizability of the findings, specifically the extent to which the developed taxonomy can be applied beyond the immediate context of its creation. The taxonomy was specifically designed to classify aspects related to literature search tasks within the \gls{kgqa} domain. However, the process required incorporating insights from the literature covering various application areas due to the limited availability of dedicated \gls{kgqa} classification studies. Consequently, certain structural elements or categories within the taxonomy are likely to capture fundamental aspects that are not exclusive to the intended use case. Therefore, parts of the taxonomy might have relevance for other research domains or information retrieval contexts. However, determining the precise extent of this generalizability would require further empirical validation in different settings.

\paragraph{Reliability}
Reliability addresses the consistency and repeatability of the study, specifically whether the outcomes depend on the specific researchers involved. The primary threat to reliability in this work stems from the fact that the taxonomy development process, particularly its subjective phases, was conducted by a single researcher. Key steps involving subjective judgment include: 1) filtering literature based on inclusion/exclusion criteria, 2) assessing semantic similarity for class deduplication and clustering, 3) evaluating the relevance of categories and classes, and 4) making refinement decisions based on quantitative and qualitative evaluations. There is a possibility that if different researchers were to replicate the process, the results might vary. This threat was mitigated by: 1) documenting the entire procedure in detail, including the reasons for key decisions, both in this thesis and the associated replication package artifacts to improve transparency and traceability. 2) Operationalizing the process and providing structured artifacts in JSON and BibTeX format facilitates replication and allows other researchers to analyze, adapt, or extend the taxonomy based on alternative judgments or additional evidence. Although subjectivity cannot be entirely eliminated, these measures aim to maximize transparency and the potential for independent verification or refinement of the results.