
\chapter{Parameter Selection Process}
\label{ch:parameter_selection_process}

The HubLink and the baseline approaches each expose a variety of parameters that, in an ideal scenario, would be optimized via a full hyperparameter-tuning procedure using separate training and test datasets. However, we deviate from this optimal approach due to practical constraints, as we found that experiments with \glspl{llm} are both costly and time-consuming. Since the durations of multi-week experiments fall outside the scope of this master thesis, we adopt an alternative workflow we refer to as the \emph{Parameter Selection Process}. The goal of this process is to determine the parameter configurations that optimize the retrieval performance of the \gls{kgqa} retrievers in our experimental setting while respecting the time and resource constraints of this project. Consequently, the results of this process are configurations for each retriever, which we then used in our experiments.

The following sections are organized as follows. First, in Section~\ref{sec:selection_planning}, we explain the planning process that was done prior to conducting the selection process. Then, Sections \ref{sec:param_selection_hublink}, \ref{sec:param_selection_difar}, \ref{sec:param_selection_fidelis}, and \ref{sec:param_selection_mindmap}, describe the selection process for HubLink and the baseline retrievers: DiFaR, FiDeLiS, and Mindmap. Finally, in Section~\ref{sec:param_selection_threats_to_validity}, we discuss threats to the validity of the parameter selection process.




\section{Planning}
\label{sec:selection_planning}

This section documents the planning of the parameter selection process. We first describe the methodology that was applied to realize the process. Next, we explain why we chose the Recall and Hits@10 metrics for the selection. Following this, we describe the dataset that was applied and the \glspl{llm}. Then, we describe the preliminary pipeline steps that are implemented before we describe why we decided to omit the StructGPT and ToG \gls{kgqa} approaches.

\subsection{Methodology}
\label{sec:selection_planning_methodology}

To carry out this process, we apply the \gls{ofat} method to identify the best parameters according to the Recall and Hits@10 metrics. In this method, a base configuration is selected and then each factor is successively varied over its range while all other factors are kept constant \cite{montgomery_design_2017}. Consequently, we defined a base configuration and a range of parameter values for each \gls{kgqa} approach. Then, following the \gls{ofat} method, multiple run configurations have been created based on the base configuration and ranges to subsequently test them using a reduced version of the \gls{kgqa} dataset.


\subsection{Metrics for Selection}
\label{sec:selecting_tuning_metric}

To conduct the selection, metrics need to be chosen that allow us to determine whether a change in the value of a parameter is justified. For the selection process, we focused on retrieval performance rather than generation. This is because the generation of the answers depends on the retrieved contexts, which means that if the retriever cannot find the contexts that are relevant, it will not be able to generate an answer based on it. With regard to the choice of retrieval metrics, our primary metric is Recall, although we also use Hits@10 where needed. There are several reasons for these metrics, which we explain in the following:

First, when constructing the \gls{kgqa} datasets (see Section~\ref{sec:implementation_qa_dataset_generation}), we ensured that only those triples were designated as the ground truth for which the content required to answer the question is actually present. We deliberately omitted any intermediate triples that must be traversed to reach the target since they are not strictly necessary to answer the actual question. To illustrate, consider a question that is asking for the authors of a paper. The authors are stored in separate nodes that all connect to a root node named \emph{Authors List}. That root node serves only as a means to reach the author nodes and does not provide the information necessary to answer the question. In our preliminary tests, we observed that retrievers tend to include such intermediate nodes in their retrieved context. Furthermore, because answer generation occurs after context retrieval, these nodes could be helpful during answer generation by providing additional context to the \gls{llm}. Consequently, we chose not to penalize this behavior, which is consistent with using the Recall metric.

With regard to the Hits@10 metric, this metric allows us to understand whether the retriever ranks those contexts that are more relevant higher than those that are less relevant. For example, if the retriever includes intermediate triples in their output, the Hits@10 metric is still maximal if those triples that are actually relevant are higher on the list of outputs.

\subsection{Choosing a Graph Variant}
\label{sec:selection_planning_graph_variant}

In Section~\ref{sec:contribution_templates} we introduced four different graph variants for our experiments to test the robustness of \gls{kgqa} approaches. However, due to cost reasons, we were unable to run the parameter selection process (and experiments) on all four graph variants. We therefore decided on \hyperref[enum:gv1]{\textbf{GV1}} as we believe that it is the graph variant with the most realistic modeling for real-world scenarios. The reason for this is that long paths allow the relationships between information to be captured, which preserves crucial context. Furthermore, the content is distributed by concern, which allows for extensibility in the future.

\subsection{Using a Reduced KGQA Dataset}
\label{sec:selection_planning_reduced_qa}

In addition to only running the selection process on one graph variant as mentioned above, we used a reduced version of the label-based \gls{kgqa} dataset (for \hyperref[enum:gv1]{\textbf{GV1}}) during the selection process. As described in Section~\ref{sec:implementation_qa_dataset_generation}, the \gls{kgqa} datasets were created with respect to use cases and retrieval operations. Each question is also classified as either semi-typed or untyped. For almost any pairing of a use case with a retrieval operation, there are four corresponding pairs of questions and answers. When constructing the reduced dataset, we therefore selected one semi-typed question per combination to ensure that each question is representative of the larger dataset. We chose semi-typed over untyped questions, as we expected them to perform better, which is important when selecting parameters. Consequently, the reduced \gls{kgqa} dataset for graph variant \hyperref[enum:gv1]{\textbf{GV1}} contains a total of 44 questions.



% We created \textbf{GV1} in such a way that it provides continued value for future use. This means that it should be easy to read and extendable for future contexts. 

% We gathered the majority of our results using graph variant \hyperref[enum:gv1]{\textbf{GV1}} as we expect \hyperref[enum:gv1]{\textbf{GV1}} to be the most relevant model for real-world scenarios due to its inherent extendability and its capability to capture semantic relationships. We explicitly state in each following section which graph variants we employed for the presented results.

\subsection{Large Language and Embedding Models}
\label{sec:selection_planning_llms}

As all retrievers are based on \glspl{llm}, the model selection is crucial for the performance of the retriever. Since HubLink, DiFaR, Mindmap, and FiDeLiS also work with embeddings, the selection of the embedding model is equally important.

For our experiments and the selection process, we implemented the following endpoints: \emph{OpenAI} as a proprietary provider as well as \emph{Ollama} and \emph{Hugging Face} for open-source models, both of which are run locally on the server. Furthermore, when choosing which models to use, we considered the following points:

\begin{enumerate}
    \item The OpenAI endpoint is proprietary and can introduce high costs if not managed carefully. As such, we considered the associated costs of the models and how many models from OpenAI we are using.
    \item Through testing, we found the Hugging Face models to be less optimized than the Ollama ones. This means that the amount of hardware memory resources required to run models on the Hugging Face endpoint is higher than on the Ollama endpoint, which may lead to \emph{out-of-memory} errors.
    \item We are restricted to the hardware resources available on our server. We have $32 GB$ of GPU memory available, which is enough to fit optimized \glspl{llm} of the size of $32B$ parameters on the GPU. However, running embedding models in parallel is then not feasible. Moreover, even if a large model fits on the GPU, its response time is likely too slow to be used in our experiments. Consequently, we chose to use smaller models. 
\end{enumerate}

To help in the selection process, we reviewed popular leaderboards to assess the performance of the models available. We examined two leaderboards, both reflecting the status as of February 16, 2025. For \glspl{llm}, we examined the \emph{Chatbot Arena Leaderboard}\footnote{\url{https://huggingface.co/spaces/lmarena-ai/chatbot-arena-leaderboard} [last accessed on 16.02.2025]}, proposed by \textcite{chiang_chatbot_2024}. For embedding models, we observed the \emph{Massive Multilingual Text Embedding Benchmark (MMTEB)}\footnote{\url{https://huggingface.co/spaces/mteb/leaderboard} [last accessed on 16.02.2025]}, introduced by \textcite{enevoldsen_mmteb_2025}. A snapshot of both leaderboards at the time of review is available in our replication package \cite{schneider_replication_2025}.

\subsubsection{Selection of LLMs}

We selected the following \glspl{llm} for our experiments: \emph{GPT-4o}, because the model is ranked at the highest position in the Chatbot Arena leaderboard via the OpenAI endpoint. \emph{GPT-4o-mini}, ranked 24th yet delivering a strong performance at a fraction of the cost and also \emph{O3-mini}, a newly released model that inherently implements chain-of-thought reasoning \cite{wei_chain--thought_2023}. To include open-source options, we chose \emph{Qwen2.5}, which is the Ollama endpoint model that performs the best on the leaderboard. However, due to our hardware constraints, we had to reduce the model to its $14B$ parameter variant. Furthermore, we selected \emph{Llama3.1}, which represents the second-best Ollama model in the leaderboard. However, we had to scale it down to the $8B$ parameter model because of hardware constraints. We also evaluated \emph{DeepSeek-R1} \cite{deepseek-ai_deepseek-r1_2025}, a new open-source reasoning model with promising benchmarks, but its performance-to-runtime ratio was substantially worse than that of our selected models, so we excluded it.

\subsubsection{Selection of Embedding Models} 

For embedding models, we included \emph{text-embedding-3-large}, the largest embedding model available via the OpenAI API. With regard to open-source models, we chose the \emph{Mxbai-Embed-Large} model, which is a fast and popular open-source model ranked 41st on the MMTEB leaderboard. Because it has a quick response time with good performance, it is a good choice for the base configurations in our selection process. We also evaluated \emph{Granite-Embedding}, a new Ollama endpoint model that is not yet on the leaderboard. Still, it is a promising model that is fast and looks to have a good performance. Finally, we tested \emph{gte-Qwen2-7B-instruct}, the top-ranked MMTEB model, but it exhibited slow inference and unexpectedly poor performance. We are not entirely sure why the performance of the model was poor, but we suspect that it may be due to the fact that it was used over the Hugging Face endpoint, which uses unoptimized models. Ollama, on the other hand, provides expert optimization for their models, which makes them faster and could make them perform better. This is the reason we opted to use models from Ollama over those provided on Hugging Face.

\subsection{Pre-, Post-Retrieval and Generation}
\label{sec:selection_planning_steps}

Our \gls{rag} pipeline involves four steps: 1) Pre-Retrieval, 2) Retrieval, 3) Post-Retrieval, and 4) Generation. In the following, we are going to introduce each step and its relevance to the parameter selection process.

\paragraph{Pre-Retrieval:} The pre-retrieval step is responsible for the preprocessing of the input question. We implemented a question augmentation technique that prompts an \gls{llm} to improve the given question by clarifying ambiguities, incorporating related keywords or phrases that will help the retrieval system retrieve more accurate and comprehensive information, and adding nouns or noun phrases to terms to clearly indicate their types or roles. Regarding the parameter selection, we tested each retriever with and without augmentation.

\paragraph{Retrieval:} The retrieval step is where both HubLink and the \gls{kgqa} baseline retrievers are applied. Each \gls{kgqa} approach has its own set of parameters relevant to the parameter selection process. For each parameter, we chose a range of values that were tested. The ranges are documented for each approach in the following sections.

\paragraph{Post-Retrieval:} In the post-retrieval step, the retrieved context from the previous step is processed. We implemented a function that prompts an \gls{llm} to rerank the retrieved contexts based on the provided question. During the parameter selection process, we then tested each \gls{kgqa} approach with and without reranking.

\paragraph{Generation:} The generation step is responsible for generating the final answer based on the question and the contexts that have been retrieved. The generation is done by prompting an \gls{llm} with the question and the contexts and asking it to generate an answer. However, because almost all \gls{kgqa} approaches provide an answer as part of their procedure, the generation step is skipped to retain the original answer of the approach. The only exception is \gls{difar}, for which generation prompting is used.

\textit{We provide the prompts that have been used for the question augmentation, reranking, and generation procedures in Appendix \ref{sec:appendix:prompts}.}

\subsection{Omitting StructGPT and ToG}
\label{sec:selection_planning_omitted_retrievers}

The use of the StructGPT \cite{jiang_structgpt_2023} and ToG \cite{sun_think--graph_2024} \gls{kgqa} approaches proved to be unsuitable in our experimental setting. Both approaches were unable to retrieve any relevant information from the graph, which is why we omitted them from the selection process and the experiments. A more detailed analysis of why these approaches are unable to answer the questions in our experiment can be found in Section~\ref{sec:discussion_on_evaluation_results}. 

\section{Parameter Selection for HubLink}
\label{sec:param_selection_hublink}

In the sections that follow, we initially outline the base configuration and the range of parameters explored during the parameter selection phase for the HubLink retriever. Subsequently, we analyze the results of the test runs and detail the parameter values chosen for the final configuration utilized in our subsequent evaluations.

\subsection{Base Configuration and Parameter Ranges}

\begin{table}[t]
    \centering
    \begin{tabularx}{\textwidth}{l X}
        \toprule
        \textbf{Parameter} & \textbf{Parameter Space} \\
        \midrule
        \texttt{Do Traversal Strategy} & \underline{False}, True \\
        \texttt{Extract Question Components} & False, \underline{True} \\
        \texttt{Top Paths to Keep} & \underline{10}, 20, 30 \\
        \texttt{Number of Hubs} & \underline{10}, 20, 30 \\
        \texttt{Filter Output Context} & False, \underline{True} \\
        \texttt{Diversity Ranking Penalty} & 0, 0.01, \underline{0.05}, 0.1 \\
        \texttt{Path Weight Alpha} & 0, 3, \underline{5}, 9 \\
        \texttt{Do Question Augmentation} & \underline{False}, True \\
        \texttt{Do Reranking} & \underline{False}, True \\
        \texttt{\gls{llm}} & \underline{gpt-4o-mini}, gpt-4o, o3-mini, \underline{Qwen2.5-14B}, Llama3.1-8B \\
        \texttt{Embedding Model} & \underline{mxbai-embed-large}, text-embedding-3-large, granite-embedding \\
        \bottomrule
    \end{tabularx}
    \caption[Base Configuration and Parameter Space for HubLink]{The base configuration (\underline{underlined}) and the parameter space for HubLink.}
    \label{tab:hublink_tuning_configs}
\end{table}


In \autoref{tab:hublink_tuning_configs}, the parameter values for the base configuration and the ranges of parameters tested for HubLink are presented, which we will discuss in the following:

\paragraph{Do Traversal Strategy:} The parameter determines whether the approach can use a provided topic entity that accompanies the question. When set to \texttt{True}, the \emph{graph traversal} strategy is employed, using the topic entity as the starting point in the graph. However, for the base configuration of the selection process, we used the \emph{direct retrieval} strategy because this strategy is faster, allowing us to test more configurations during the selection process.

\paragraph{Extract Question Components:} This parameter specifies whether the component extraction process is used. If used, the question components are extracted using an \gls{llm} and utilized in addition to the question during the search for relevant hubs. This technique is integral to the HubLink retriever and has the purpose of enhancing the performance of queries that have multiple constraints. To evaluate our design decision to add this extraction, we tested the difference in performance during the selection process when the extraction is enabled or disabled. For the base configuration, it remains enabled.

\paragraph{Filter Output Context:} When generating partial answers, the hub paths are used. Each path can include multiple triples that form the path. To determine which of the triples are actually needed for the answer, we added a filtering based on an \gls{llm}. During the selection process, we tested our design decision of adding this filtering step by disabling it in one execution. However, for the base configuration the filtering remains enabled, as it is an integral part of the HubLink approach.

\paragraph{Top Paths to Keep:} This parameter defines the number of paths retained per hub. Increasing this value allows more context to be used during partial answer generation, which can improve performance, but may also introduce noise. We set this value at 10, as this is the maximum number of triples requested in the applied \gls{kgqa} dataset. We also increased the value to 20 and 30 to test the effect of allowing more context.

\paragraph{Number of Hubs:} The parameter specifies how many hubs are used to generate partial answers. A higher value increases the chances of finding relevant context, but also increases runtime and cost. We set 10 as the starting point and varied the values in increments of 10 up to 30 to assess the impact of additional context on retrieval performance. As mentioned previously, the starting value is based on the maximum number of triples requested from the \gls{kgqa} dataset.

\paragraph{Diversity Ranking Penalty:} This parameter influences the overall score of each hub and determines the emphasis on diversity among the hub paths during reranking. This technique is a core part of HubLink and was added during the design phase to decrease the likelihood of hubs being pruned because they have a high diversity of scores. The higher the value, the more diversity is tolerated. During development, we found that the value of 0.05 provides satisfactory results, which is why we set it as the default. We further tested the values of 0, 0.01 and 0.1 to see how they change the results.

\paragraph{Path Weight Alpha:} During the design of HubLink, we added weighting when calculating the final scores of the hubs before pruning them, which in theory should reduce the likelihood of pruning hubs that have unevenly distributed hub path scores. During development, we found the value of 5 to provide satisfactory results, which is why we set it as the value in the base configuration. We also tested the values 0, 3, and 9.

\paragraph{Do Question Augmentation:} This parameter determines whether the question is augmented before it is provided to the \gls{kgqa} approach. It is not part of the HubLink retriever itself but rather a pre-retrieval step that is applied before sending the question to the retriever. We disabled it by default and assessed its impact on the results during the parameter selection process.

\paragraph{Do Reranking:} The reranking process is also not part of the approach but rather a post-retrieval step that is applied on the list of contexts returned by the approach. In theory, it should improve the results of rank-based scores, as those contexts that are relevant are put at the top of this list. However, if the approach is already good at ranking by relevance, adding this step should not have much of an effect. 

\paragraph{\gls{llm}:} The \gls{llm} is used to generate partial answers, filter the context, convert paths to texts, and generate the final answer. The models selected for the parameter selection process were already introduced in Section~\ref{sec:selection_planning_llms}. For our base configuration, we used two models, \emph{gpt-4o-mini} and \emph{Qwen2.5-14B}, to reduce the runtime of the parameter selection process by running multiple configurations in parallel. Therefore, the embedding models and the other \glspl{llm} have been tested with the \emph{Qwen2.5-14B} model as the base, and the remaining parameters have been run with \emph{gpt-4o-mini} as the base. 

\paragraph{Embedding Model:} The embedding model is used to transform the question and the hubs into vectors. The models that we used during the selection process are introduced in Section~\ref{sec:selection_planning_llms}. Regarding the base configuration, we used the \emph{mxbai-embed-large} model because it is cost-efficient and fast.



% \subsection{Static Parameters}

% In addition to the parameters that were varied during the parameter selection process, other parameters are not relevant for selection and were static. We explain them in the following.

% \paragraph{Max Workers:} This parameter determines the number of threats to use during the indexing and partial answer generation. It depends on the environment in which the retriever is used and must be set as such. For our experiments we set this number to 8.

% \paragraph{Compare Hubs with Same Hop Amount:} This parameter is only relevant for the traversal strategy as it determines during the retrieval process, whether the hubs that are compared with each other need to have the same amount of hops from the topic entity. Because the \gls{orkg} has a predefined static structure with no unexpected intermediate nodes, the value is set to false.

% \paragraph{Indexing Root Entity Types:} This parameter determines the types of root entities that are used as entry points in the graph to start the indexing process. We are not setting this parameter as we are providing the IDs of the entities directly with the next parameter.

% \paragraph{Indexing Root Entity IDs:} This parameter determines the root entities that are used as entry points in the graph to start the indexing process. For our use case, we set the node of the research field of \emph{Software Architecture and Design} as the indexing entry point, since all our experimentation data are stored within this research field.

% \paragraph{Force Index Update:} With this parameter, it is possible to force the indexing to take place during the initialization of the retriever. This is helpful to capture changes in the graph but it is not interesting parameter for testing.

% \paragraph{Check Updates During Retrieval:} Allows to update those Hubs that are touched during the retrieval process and update their index without having to reindex the whole graph. This is also a helpful parameter to capture changes in the graph, but not an interesting parameter to test.

% \paragraph{Max Hub Path Length:} This parameter controls the maximum length of the hub paths. In the \gls{orkg}, we want each publication node to be a hub. Because we know the structure of each publication and their paths prior, we know that there are no cases where the length exceeds a point where it would be relevant for restriction. As such, we do not restrict the length of the paths.

% \paragraph{Hub Types:} This parameter specifies the types of nodes to be considered as roots for a hub. For our application, which involves retrieving and comparing paper information, we assign this parameter to the type of paper as defined in the \gls{orkg}.

% \paragraph{Hub Edges:} This is another parameter to define which node in the graph should act as a root for a hub. In this case, this parameter defines a threshold that determines a node as a hub if their outbound edge count exceeds the value. In our experiment, we are already setting the roots of the hubs by type, which is why we do not need to set this parameter.

% \paragraph{Max Level:} In the traversal strategy, answer generation is performed in levels. First, hubs of the current level are searched and then partial answers are generated for each hub. If no answers are generated, the next level is searched. This parameter determines the maximum number of levels in the search. Because of the subgraph structure of the ORKG that we are using for our experiments, we know that there are no Hubs that are reached on the second level. As such, we set this parameter to 1.

% \paragraph{Use Source Documents:} This parameter determines whether the HubLink retriever linking process is used. Because we used the label-based dataset in this experiment, we did not use the source linking process. This is because the data that are asked for in the questions are on such an abstraction level that the answer cannot be found in the full texts of the publication.

% \paragraph{Distance Metric:} The distance metric determines how the similarity score between the embeddings in the vector store and the question is calculated. Three distance metrics are supported: cosine, inner-product and l2 distance. In our preliminary testing we tried each of those variants and found them to perform exactly the same. Therefore, we chose \emph{cosine} as the default distance metric due to its popularity.


\subsection{Parameter Selection}

\begin{table}[t]
    \centering
    \begin{tabular}{llll}
        \toprule
        \textbf{Parameter} & \textbf{Value} & \textbf{Recall} & \textbf{Hits@10} \\
        \midrule
        \multirow{5}{*}{Large Language Model} 
            & \underline{gpt-4o-mini} & 0.512 & 0.372 \\
            & gpt-o3-mini & 0.608 (+18.7\%) & \textbf{0.499 (+34.1\%)} \\
            & gpt-4o & \textbf{0.615 (+20.1\%)} & 0.481 (+29.3\%) \\
            & Qwen2.5-14B & 0.448 (-12.5\%) & 0.367 (-1.3\%) \\
            & Llama3.1-8B & 0.374 (-27.0\%) & 0.259 (-30.4\%) \\
        \midrule
        \multirow{3}{*}{Embedding Model}
            & \underline{mxbai-embed-large} & 0.448 & 0.367 \\
            & text-embedding-3-large & \textbf{0.555 (+23.9\%)} & \textbf{0.494 (+34.6\%)} \\
            & granite-embedding & 0.490 (+9.4\%) & 0.405 (+10.4\%) \\
        \midrule
        \multirow{4}{*}{Path Weight Alpha}
            & 0 & 0.391 (-23.6\%) & 0.248 (-33.3\%) \\
            & 3 & 0.491 (-4.1\%) & 0.340 (-8.6\%) \\
            & \underline{5} & \textbf{0.512} & \textbf{0.372} \\
            & 9 & \textbf{0.505} (-1.4\%) & 0.343 (-7.8\%) \\
        \midrule
        \multirow{4}{*}{Diversity Ranking Penalty}
            & 0.00 & 0.338 (-34.0\%) & 0.265 (-28.8\%) \\
            & 0.01 & 0.412 (-19.5\%) & 0.323 (-13.2\%) \\
            & \underline{0.05} & \textbf{0.512} & \textbf{0.372} \\
            & 0.10 & 0.474 (-7.4\%) & 0.328 (-11.8\%) \\
        \midrule
        \multirow{3}{*}{Top Paths to Keep}
            & \underline{10} & \textbf{0.512} & \textbf{0.372} \\
            & 20 & 0.438 (-14.5\%) & 0.301 (-19.1\%) \\
            & 30 & 0.247 (-51.8\%) & 0.155 (-58.3\%) \\
        \bottomrule
    \end{tabular}
    \caption[Results of the Parameter Selection Process for HubLink Part 1]{The results of the parameter selection process for HubLink. The base configuration parameter is \underline{underlined}, and the highest metric score per parameter is \textbf{bold}. This is the first out of two tables that display the results.}
    \label{tab:hublink_parameter_selection_part_1}
\end{table}

\begin{table}[t]
    \centering
    \begin{tabular}{l l l l}
        \toprule
        \textbf{Parameter} & \textbf{Value} & \textbf{Recall} & \textbf{Hits@10} \\
        \midrule
        \multirow{3}{*}{Number of Hubs} 
            & \underline{10} & 0.512 & 0.372 \\
            & 20 & 0.517 (+1.0\%) & 0.325 (-12.6\%) \\
            & 30 & \textbf{0.554 (+8.2\%)} & \textbf{0.356 (-4.3\%)} \\
        \midrule
        \multirow{2}{*}{Output Filtering} 
            & \underline{True}  & 0.512 & \textbf{0.372} \\
            & False & \textbf{0.631 (+23.2\%)} & 0.191 (-48.7\%) \\
        \midrule
        \multirow{2}{*}{Extract Question Components} 
            & \underline{True} & \textbf{0.512} & 0.372 \\
            & False & 0.440 (-14.1\%) & \textbf{0.375 (+0.8\%)} \\
        \midrule
        \multirow{2}{*}{Do Question Augmentation} 
            & True & 0.500 (-0.2\%) & 0.333 (-8.8\%) \\
            & \underline{False} & \textbf{0.512} & \textbf{0.372} \\
        \midrule
        \multirow{2}{*}{Do Reranking} 
            & True & 0.497 (-2.8\%) & \textbf{0.334 (-10.4\%)} \\
            & \underline{False} & \textbf{0.512} & 0.372 \\
        \midrule
        \multirow{2}{*}{Do Traversal Strategy} 
            & True & \textbf{0.559 (+9.2\%)} & \textbf{0.422 (+13.4\%)} \\
            & \underline{False} & 0.523 & 0.395 \\
        \bottomrule
    \end{tabular}
    \caption[Results of the Parameter Selection Process for HubLink Part 2]{The results of the parameter selection process for HubLink. The base-configuration parameter is \underline{underlined}, and the highest metric score per parameter is \textbf{bold}. This is the second out of two tables that display the results.}
    \label{tab:hublink_parameter_selection_part_2}
\end{table}

We ran a total of 22 configurations that we split into two base configurations to run them in parallel. The final configuration parameters are shown in \autoref{tab:hublink_final_config} and the results of the parameter selection process are presented in \autoref{tab:hublink_parameter_selection_part_1} and \autoref{tab:hublink_parameter_selection_part_2}, which we will discuss in the following:

\paragraph{Large Language Model:} 
Comparing the five \glspl{llm}, we observe a strong correlation between the capability of the model and the retrieval performance. The \emph{gpt-4o-mini} model achieved a Recall of 0.512 and a Hits@10 of 0.372. Larger and more advanced models like \emph{gpt-o3-mini} and \emph{gpt-4o} significantly outperformed the baseline, achieving Recall improvements of +18.7\% and +20.1\% respectively, and Hits@10 improvements of +34,1\% and +29.3\%. The results also indicate that the open-source models, \emph{Qwen2.5-14B} and particularly \emph{Llama3.1-8B}, have considerably lower performance. This suggests that the capabilities of the \gls{llm} have a high impact on the effectiveness of HubLink, and potentially even better results could be achieved with larger or more capable future models. Although \emph{gpt-4o} achieved the highest Recall (+20.1\%), \emph{o3-mini} reached the highest Hits@10 score (+34.1\%). Therefore, both are strong choices for the final configuration. We selected \textbf{\emph{o3-mini}} for our final configuration because it is the newer model according to its release date.

\paragraph{Embedding Model:} 
For the embedding model comparison, the baseline \emph{mxbai-embed-large} resulted in a Recall of 0.448 and Hits@10 of 0.367. Switching to \emph{text-embedding-3-large} produced substantial improvements across both metrics, boosting Recall by +23.9\% to 0.555 and Hits@10 by +34.6\% to 0.494. The \emph{granite-embedding} model also outperformed the baseline (+9.4\% Recall, +10.4\% Hits@10) but was clearly inferior to \emph{text-embedding-3-large}. This indicates that the choice of embedding model has a major impact on the performance of HubLink. Based on these results, we selected \textbf{\emph{text-embedding-3-large}} as the embedding model for the final configuration.

\paragraph{Path Weight Alpha:} 
Setting alpha to 0 resulted in a drastic performance drop (-23.6\% Recall, -33.3\% Hits@10), indicating that path weighting is essential. The baseline value of \textbf{5} achieved the highest scores for both Recall (0.512) and Hits@10 (0.372). Increasing alpha further to 9 led to a slight decrease in performance compared to the baseline. This suggests that while weighting paths is beneficial, excessive weighting might negatively impact the results. Consequently, we retained the baseline value of \textbf{5} for the alpha parameter.

\begin{sloppypar}
\paragraph{Diversity Ranking Penalty:} 
The results for the diversity penalty mirror those of the \texttt{PathWeightAlpha} parameter. Disabling the diversity penalty (0.00) resulted in a drastic performance drop (-34.0\% Recall, -28.8\% Hits@10). The baseline value of \textbf{0.05} yielded the best results, achieving the highest Recall (0.512) and Hits@10 (0.372) among the values tested. Increasing the penalty to 0.10 resulted in lower scores compared to 0.05. This suggests that while diversification is beneficial, an excessive penalty might negatively impact results. We therefore selected \textbf{0.05} for the diversity penalty.
\end{sloppypar}

\paragraph{Top Paths to Keep:} 
The baseline for the number of paths that are kept per hub was set to 10. Increasing the number of paths to 20 or 30 led to a significant drop in performance. Keeping 20 paths reduced Recall by 14.5\% and Hits@10 by 19.1\%, while keeping 30 paths caused a drastic drop (-51.8\% Recall, -58.3\% Hits@10). The baseline value of \textbf{10} yielded the best performance, which is unexpected because, in theory, adding more paths provides more context for each hub when generating partial answers, increasing the chance of retrieving relevant information from the graph. We hypothesize that this discrepancy is due to the use of the \emph{gpt-4o-mini} model during execution. It is likely that the model was overwhelmed by the large volume of context, causing relevant information to be lost because of additional noise in the data. As a result, the model may have struggled to extract key facts from the data. It remains to be tested whether the results persist with a different \gls{llm} that handles large contexts more effectively. However, for our final configuration, we rely on the current results rather than assumptions and set the number of paths to 10.

\begin{sloppypar}
\paragraph{Number of Hubs:} 
The effect of increasing the number of hubs contrasts with the \texttt{TopPathsToKeep} parameter. While keeping more paths per hub degraded performance, increasing the number of hubs from the baseline of 10 improved Recall (+1.0\% for 20 hubs, +8.2\% for 30 hubs). This supports the intuition that providing more context increases the likelihood of retrieving relevant information. We hypothesize that this works better than increasing paths per hub because the context is processed sequentially by the \gls{llm} rather than simultaneously. However, the Hits@10 metric showed a less positive trend, decreasing for 20 hubs (-12.6\%) and remaining slightly below the baseline even for 30 hubs (-4.3\%). This suggests a trade-off where more hubs lead to more relevant information overall, increasing the Recall, but this might make it slightly harder to rank the relevant contexts within the top 10. For our final configuration, we prioritized the notable Recall improvement and selected \textbf{30} hubs, accepting the minor Hits@10 decrease relative to the 10-hub baseline.
\end{sloppypar}

\paragraph{Output Filtering:} 
The output filtering step has the aim of reducing the HubLink output to the triples that are actually relevant to the question. Without filtering, HubLink returns all triples from the paths used in the final answer, including many that are not useful. The baseline configuration had filtering enabled, yielding Recall 0.512 and Hits@10 0.372. Disabling filtering resulted in a substantial Recall increase to 0.631 (+23.2\%) but caused a drastic drop in Hits@10 to 0.191 (-48.7\%). This clearly demonstrates the function of the filter as removing irrelevant triples boosts precision (increasing Hits@10) at the cost of potentially removing some relevant triples (reducing the Recall). Because we were using \emph{gpt-4o-mini} for the baseline, we hypothesize that its limitations might contribute to removing relevant triples alongside irrelevant ones. Given that our final configuration uses the more capable \emph{o3-mini} model, we anticipate its filtering performance will be improved. Therefore, despite the Recall drop observed with the baseline model, we \textbf{enabled} output filtering in the final configuration, prioritizing cleaner, more precise results.

\begin{table}[t]
    \centering
    \begin{tabular}{l l}   
        \toprule
        \textbf{Parameter} & \textbf{Value} \\
        \midrule
        \texttt{LLM} & gpt-o3-mini \\
        \texttt{Embedding Model} & text-embedding-3-large \\ 
        \texttt{Do Traversal Strategy} & True \\
        \texttt{Extract Question Components} & True \\
        \texttt{Top Paths to Keep} & 10 \\
        \texttt{Number of Hubs} & 30 \\
        \texttt{Filter Output Context} & True \\
        \texttt{Diversity Ranking Penalty} & 0.05 \\
        \texttt{Path Weight Alpha} & 5 \\ 
        \texttt{Do Question Augmentation} & False \\
        \texttt{Do Reranking} & False \\
        \bottomrule
    \end{tabular}
    \caption[Final Configuration for HubLink]{The final configuration used in subsequent experiments for our proposed HubLink approach.}
    \label{tab:hublink_final_config}
\end{table}

\paragraph{Extract Question Components:} 
The goal of extracting components from questions is to help the retriever better handle complex queries with multiple constraints. Enabling this component extraction resulted in a Recall of 0.512 and Hits@10 of 0.372. Disabling it caused a substantial drop in Recall to 0.440 (-14.1\%), although it yielded a marginally higher Hits@10 score of 0.375 (+0.8\%). However, the significant impact on Recall outweighs the negligible gain in Hits@10. Therefore, we conclude that component extraction is crucial for effectiveness and \textbf{enabled} the extraction for the final configuration.

\paragraph{Do Question Augmentation:} 
Disabling the prior augmentation of the question yielded a Recall 0.512 and Hits@10 0.372. Enabling the augmentation had a negligible impact, with Recall dropping to 0.500 (-0.2\%) but resulted in a noticeable decrease in the Hits@10 score to 0.333 (-8.8\%). Although question augmentation might be more beneficial for less structured or untyped queries, our experiments on the typed questions in the reduced \gls{qa} dataset did not show an advantage. Based on the observed decrease in ranking performance without any Recall benefit, we \textbf{disabled} the augmentation for the final configuration.

\paragraph{Do Reranking:} 
The inclusion of a ranking step, where the \gls{llm} reorders the initially retrieved paths based on relevance to the question, should improve the Hits@10 score. The results show that disabling the ranking resulted in a Recall of 0.512 and Hits@10 of 0.372. Enabling the reranker unexpectedly led to slightly worse performance on both metrics. Recall decreased slightly to 0.497 (-2.8\%), and Hits@10 decreased to 0.334 (-10.4\%). These results contradict the expectation that reranking should primarily boost Hits@10. The observed degradation suggests either that the specific reranking implementation or model used was ineffective for this task, or potentially that the initial ranking provided by HubLink was already close to optimal, and the reranking step introduced noise. Given that enabling reranking demonstrated worse performance, we \textbf{disabled} it for the final configuration.

\paragraph{Traversal Strategy:} 
This parameter switches HubLink from the \emph{direct retrieval strategy} to the \emph{graph traversal strategy}. The former achieved a Recall of 0.523 and Hits@10 of 0.395, while the latter achieved a Recall of 0.559 (+9.2\%) and Hits@10 of 0.422 (+13.4\%). Given these results, we observe a small increase in performance using the traversal strategy. This would suggest that it is helpful to provide a topic entity that guides the retrieval procedure. However, we argue that the \gls{kg} that we are using for the experiments is too small to gather significant results. It remains to be tested on a large graph whether the retrieval substantially improves with the traversal strategy. For our final configuration, we \textbf{enabled} the traversal strategy due to the improvement in performance on our data.


\section{Parameter Selection for DiFaR}
\label{sec:param_selection_difar}

In the sections that follow, we initially describe the base configuration and the range of parameters evaluated during the parameter selection process for the \gls{difar} approach. Subsequently, we analyze the outcomes of the test runs and outline the parameter values chosen for the final configuration employed in our later experiments.

\subsection{Base Configuration and Parameter Ranges}

\begin{table}[t]
    \centering
    \begin{tabularx}{\textwidth}{l X}
        \toprule
        \textbf{Parameter} & Parameter Space \\
        \midrule
        \texttt{Distance Metric} & \underline{cosine}, L2, IP \\
        \texttt{Number of Results} & \underline{30}, 60, 90, 120, 150 \\
        \texttt{Do Question Augmentation} & \underline{False}, True \\
        \texttt{Do Reranking} & \underline{False}, True \\
        \texttt{\gls{llm}} & \underline{gpt-4o-mini} \\
        \texttt{Embedding Model} & \underline{mxbai-embed-large}, text-embedding-3-large, \newline granite-embedding \\
        \bottomrule
    \end{tabularx}
    \caption[Base Configuration and Parameter Space for DiFaR]{The base configuration (\underline{underlined}) and parameter space for \gls{difar}.}
    \label{tab:difar_tuning_configs}
\end{table}


In \autoref{tab:difar_tuning_configs}, the parameter values for the base configuration and the ranges of tested parameters for \gls{difar} are presented. In the following, we briefly introduce each parameter.

\paragraph{Distance Metric:} The distance metric determines how the similarity score between the embeddings in the vector store and the question is calculated. Three distance metrics are supported: cosine, inner-product, and l2 distance. We tried each of the metrics in the selection process. The cosine distance was used in the base configuration because of its popularity.

\paragraph{Number of Results:} This parameter determines how many triples are obtained from the vector store for the generation of the answers. The more triples are fetched, the more context is available for the \gls{llm} to generate the answer. However, this also increases the cost of the retrieval process and can introduce more noise into the data. We tested the values 30, 60, 90, 120, and 150 to see how the number of triples affects the performance. For the base configuration we set the value to 30.

\paragraph{Do Question Augmentation:} This parameter determines whether the question is augmented before it is sent to \gls{difar}. As such, it is not part of the approach itself but rather a pre-retrieval step. We have disabled it in the base configuration and assessed its impact on the results when enabling it during testing.

\paragraph{Do Reranking:} The reranking process is also not part of the approach but a post-retrieval step that is applied to the list of contexts returned by the approach. We also disabled it in the base configuration but ran a configuration where it was enabled.

\paragraph{\gls{llm}:} The \gls{llm} is used to generate the final answer based on the retrieved contexts. Because it is not part of the retrieval process itself, it has no impact on the Recall and Hit@10 metrics, which is why we set the \gls{llm} to the \emph{gpt-4o-mini} model, which is cost effective and fast. 

\paragraph{Embedding Model:} The embedding model, on the other hand, influences the retrieval process as it is used to generate the embeddings of the question and the triples. The embedding models were already introduced in Section~\ref{sec:selection_planning_llms}. For the base configuration, we used the model \emph{mxbai‑embed‑large} because it is cost effective and fast.

\subsection{Parameter Selection}

\begin{table}[t]
    \centering
    \begin{tabular}{llll}
        \toprule
        \textbf{Parameter} & \textbf{Config} & \textbf{Recall} & \textbf{Hits@10} \\
        \midrule
        \multirow{5}{*}{Number of Results}
            & \underline{30} & 0.300 & 0.272 \\
            & 60  & 0.310 (+3.4\%)  & 0.272 (+0.0\%) \\
            & 90  & 0.312 (+4.2\%)  & 0.272 (+0.0\%) \\
            & 120 & 0.322 (+7.4\%) & 0.272 (+0.0\%) \\
            & 150 & \textbf{0.366 (+22.1\%)} & \textbf{0.288 (+5.6\%)} \\
        \midrule
        \multirow{3}{*}{Embedding Model}
            & \underline{mxbai-embed-large} & \textbf{0.300} & \textbf{0.272} \\
            & granite-embedding & 0.276 (-7.8\%) & 0.254 (-6.8\%) \\
            & text-embedding-3-large & 0.238 (-20.5\%) & 0.194 (-28.8\%) \\
        \midrule
        \multirow{3}{*}{Distance Metric}
            & \underline{cosine} & 0.300 & 0.272 \\
            & IP & 0.300 (+0.0\%) & 0.272 (+0.0\%) \\
            & L2 & 0.300 (+0.0\%) & 0.272 (+0.0\%) \\
        \midrule
        \multirow{2}{*}{Do Reranking}
            & \underline{False} & 0.300 & 0.272 \\
            & True & 0.300 (+0.0\%) & 0.278 (+2.2\%) \\
        \midrule
        \multirow{2}{*}{Do Question Augmentation}
            & \underline{False} & \textbf{0.300} & \textbf{0.272} \\
            & True & 0.248 (-17.4\%) & 0.196 (-28.1\%) \\
        \bottomrule
    \end{tabular}
    \caption[Results of the Parameter Selection Process for DiFaR]{The results for the parameter selection process for the \gls{difar} approach. Here, the base configuration parameter is \underline{underlined}, and the highest metric score per parameter is \textbf{bold}.}
    \label{tab:difar_parameter_selection}
\end{table}

In \autoref{tab:difar_parameter_selection} the results of 11 different configurations of the \gls{difar} approach are shown. In the following, we discuss the results and determine which parameters to use for the final configuration. The final configuration for the \gls{difar} approach is shown in \autoref{tab:difar_final_config}.

\paragraph{Number of Results:} 
As expected, increasing the number of retrieved contexts generally improves the probability of capturing the information sought after, leading to a higher Recall. Our results confirm this trend, with Recall steadily increasing from 0.300 at 30 results to 0.366 (+22.1\%) at 150 results. Interestingly, the Hits@10 metric remained constant at 0.272 for 30, 60, 90, and 120 results and even saw a minor improvement to 0.288 (+5.6\%) when retrieving 150 results. We set the value to 150 for the final configuration.

\paragraph{Large Language Model:}
As mentioned above, the \gls{llm} has no impact on retrieval performance, as it is only used to generate the answer after the contexts have been retrieved. We have set the number of retrieved triples to 150, which benefits Recall and eventually Hits@10, but also increases the amount of context fed into the answer generation component. This needs a strong model to successfully extract the necessary information. We therefore selected the model\emph{gpt-o3-mini} for the final configuration, which we expect to be large enough to handle the context window and to provide good performance in extracting contexts due to its reasoning capabilities.

\paragraph{Embedding Model:} 
The choice of the embedding model is critical for \gls{difar}, as the approach relies on a \gls{ann} search. The baseline model \emph{text-embedding-3-large} achieved the best performance with Recall 0.300 and Hits@10 0.272. Interestingly, \emph{text-embedding-3-large}, which performed strongly in other parts of our study, scored significantly lower here. The Recall is 0.238 (-20.5\%) and Hits@10 is 0.194 (-28.8\%). The \emph{granite-embedding} model also underperformed compared to the baseline with with Recall 0.276 (-7.8\%) and Hits@10 0.254 (-6.8\%). Based on these results, we selected the model \textbf{\emph{mxbai-embed-large}} for the final \gls{difar} configuration.

\paragraph{Distance Metric:} 
We tested three common distance metrics for vector similarity: \emph{Cosine}, \emph{Inner Product (IP)}, and \emph{Euclidean (L2)}. The results do not indicate any difference in performance. All three metrics yielded identical Recall (0.300) and Hits@10 (0.272) scores. Given that no difference was observed, we retained the default distance metric \textbf{\emph{cosine}} for the final configuration.

\paragraph{Do Reranking:} 
Reranking the contexts after retrieval should not affect the Recall metric, which is also reflected in the results (0.300). However, in theory, it should improve the Hits@10 score if the approach returns the context without ranking by relevance. \gls{difar} inherently ranks results based on vector similarity to the query embedding. Theoretically, an additional reranking step could refine this order to further improve Hits@10. Our results show that enabling reranking led to a negligible increase in Hits@10 to 0.278 (+2.2\%) compared to the baseline score of 0.272. Although this represents a small improvement, considering the added computational cost and complexity of a reranking step, we deem this gain insufficient to warrant its inclusion. Therefore, we decided to \textbf{disable} reranking in the final configuration.

\paragraph{Do Question Augmentation:} 
Similar to our findings with HubLink, enabling question augmentation before querying the \gls{difar} approach significantly worsened performance compared to the baseline without augmentation. Both Recall (0.248, -17.4\%) and Hits@10 (0.196, -28.1\%) dropped significantly. This suggests that modifications introduced by the augmentation process might add noise or irrelevant terms that mislead the nearest-neighbor search in the embedding space. Consequently, we \textbf{disabled} the augmentation for the final \gls{difar} configuration.

\begin{table}[t]
    \centering
    \begin{tabular}{l l}   
        \toprule
        \textbf{Parameter} & \textbf{Value} \\
        \midrule
        \texttt{Number of Results} & 150 \\
        \texttt{Distance Metric} & Cosine \\
        \texttt{LLM} & gpt-o3-mini \\
        \texttt{Embedding Model} & mxbai-embed-large \\ 
        \texttt{Do Question Augmentation} & False \\
        \texttt{Do Reranking} & False \\
        \bottomrule
    \end{tabular}
    \caption[Final Configuration for DiFaR]{The final configuration used for subsequent experiments for the \gls{difar} \gls{kgqa} baseline approach.}
    \label{tab:difar_final_config}
\end{table}



\section{Parameter Selection for FiDeLiS}
\label{sec:param_selection_fidelis}

In the following sections, we first introduce the base configuration and the parameter ranges that were tested in the parameter selection process for the FiDeLiS approach. Then, we discuss the results of the test runs and explain which parameter values were selected for the final configuration used in our subsequent experiments.

\subsection{Base Configuration and Parameter Ranges}
\begin{table}[t]
    \centering
    \begin{tabularx}{\textwidth}{l X}
        \toprule
        \textbf{Parameter} & \textbf{Parameter Space} \\
        \midrule
        \texttt{Top k} & \underline{10}, 20, 30 \\
        \texttt{Top n} & \underline{10}, 20, 30 \\
        \texttt{Alpha} & 0.1, \underline{0.3}, 0.6 \\
        \texttt{Do Question Augmentation} & \underline{False}, True \\
        \texttt{Do Reranking} & \underline{False}, True \\
        \texttt{\gls{llm}} & \underline{gpt-4o-mini}, gpt-4o, o3-mini, \underline{Qwen2.5-14B}, Llama3.1-8B \\
        \texttt{Embedding Model} & \underline{mxbai-embed-large}, text-embedding-3-large, \newline granite-embedding \\
        \bottomrule
    \end{tabularx}
    \caption[Base Configuration and Parameter Space for FiDeLiS]{The base configuration (\underline{underlined}) and the parameter space for FiDeLiS.}
    \label{tab:fidelis_tuning_configs}
\end{table}


In \autoref{tab:fidelis_tuning_configs}, the parameter values for the base configuration and the ranges of tested parameters for FiDeLiS are presented. In the following, we briefly introduce each parameter.

\paragraph{Top $k$:} This parameter determines the width of the beam search at each given step. The paths that are found are pruned by relevance with an \gls{llm} to $k$ paths. As such, when the parameter value is increased, more contexts are gathered, which should increase the likelihood of finding relevant information. We started with a value of 10, as this is the maximum number of triples requested in the \gls{kgqa} dataset and increased the value from there to 20 and 30 to see how the number of paths affects performance.

\paragraph{Top $n$:} This parameter determines the number of neighbor paths that are kept when expanding each current path, which determines the number of paths that are sent to the \gls{llm} for subsequent pruning. The more paths are kept, the higher the likelihood of finding relevant information. Similarly to the $k$ parameter, we started with a value of 10 and increased the value from there to see how it affects the performance.

\paragraph{Alpha:} This parameter determines the weight that the path score gets in comparison to the relation and neighbor scores. The higher the number, the more weight the path score gets. We started from the proposed default value from the authors, which is 0.3, and varied the score to 0.1 and 0.6.

\paragraph{Do Question Augmentation:} Same as with the other approaches, we disabled the question augmentation in the base configuration and assessed its impact on the results during testing.

\paragraph{Do Reranking:} Same as with the other approaches, we disabled the reranking by default and assessed the impact of enabling the reranking in one of the test runs.

\paragraph{\gls{llm}:} We expect the \gls{llm} to have a major impact on the performance of the approach as it guides the process of finding the relevant paths. We already introduced the models that were used in the parameter selection in Section~\ref{sec:selection_planning_llms}. For our base configuration we used the model \emph{gpt-4o-mini} as it is cost effective and fast.

\paragraph{Embedding Model:} The embedding model is used to generate the embeddings of the keywords from the question, as well as the predicates and entities from the graph. We also expect this to have a major impact on the performance as this is the primary way of pruning the paths in each depth of the beam search. We used \emph{text-embedding-3-large} as the embedding model for our base configuration, as it allows to run multiple \gls{rag} pipelines in parallel because the model is accessed over an API and not run locally, circumventing hardware constraints. The other models that were tried are introduced in Section~\ref{sec:selection_planning_llms}.

\subsection{Parameter Selection}

\begin{table}[t]
    \centering
    \begin{tabular}{llll}
        \toprule
        \textbf{Parameter } & \textbf{Config} & \textbf{Recall} & \textbf{Hits@10} \\
        \midrule
        \multirow{5}{*}{Large Language Model} 
            & \underline{gpt-4o-mini} & 0.129 & 0.129 \\
            & gpt-4o & 0.076 (-41.1\%) & 0.076 (-41.1\%) \\
            & gpt-O3-mini & \textbf{0.136 (+5.4\%)} & \textbf{0.136 (+5.4\%)} \\
            & qwen2.5:14b & 0.004 (-96.9\%) & 0.004 (-96.9\%) \\
            & llama3.1 & 0.011 (-91.5\%) & 0.000 (-100.0\%) \\
        \midrule
        \multirow{3}{*}{Embedding Model} 
            & \underline{mxbai-embed-large} & \textbf{0.129} & \textbf{0.129} \\
            & text-embedding-3-large & 0.137 (+6.2\%) & 0.136 (+5.4\%) \\
            & granite-embedding & 0.083 (-35.7\%) & 0.083 (-35.7\%) \\
        \midrule
        \multirow{4}{*}{Top k}
            & \underline{10} & \textbf{0.129} & \textbf{0.129} \\
            & 20 & 0.083 (-35.7\%) & 0.072 (-44.2\%) \\
            & 30 & 0.068 (-47.3\%) & 0.053 (-58.9\%) \\
        \midrule
        \multirow{4}{*}{Top n}
            & \underline{10} & \textbf{0.129} & \textbf{0.129} \\
            & 20 & 0.118 (-8.5\%) & 0.117 (-9.3\%) \\
            & 30 & 0.110 (-14.7\%) & 0.110 (-14.7\%) \\
        \midrule
        \multirow{4}{*}{Alpha}
            & 0.1 & 0.095 (-26.4\%) & 0.095 (-26.4\%) \\
            & \underline{0.3} & \textbf{0.129} & \textbf{0.129} \\
            & 0.6 & 0.110 (-14.7\%) & 0.110 (-14.7\%) \\
        \midrule
        \multirow{2}{*}{Do Reranking} 
            & \underline{False} & \textbf{0.129} & \textbf{0.129} \\
            & True & 0.118 (-8.5\%) & 0.117 (-9.3\%) \\      
        \midrule
        \multirow{2}{*}{Do Question Augmentation} 
            & \underline{False} & 0.129 & 0.129 \\
            & True & 0.072 (-44.2\%) & 0.072 (-44.2\%) \\
        \bottomrule
    \end{tabular}
    \caption[Results of the Parameter Selection Process for FiDeLiS]{The results for the parameter selection process of FiDeLiS. The base configuration parameter is \underline{underlined}, and the highest metric score per parameter is \textbf{bold}.}
    \label{tab:fidelis_parameter_selection}
\end{table}

The results for 13 different configurations of the FiDeLiS approach are presented in \autoref{tab:fidelis_parameter_selection}. It is immediately apparent that the overall performance scores are considerably lower for FiDeLiS compared to the other approaches tested, indicating challenges with this specific question-answering task, which we discuss in Section~\ref{sec:discussion_on_evaluation_results}. Because the scores are rather low, it is hard to attribute the observed results to the change of the parameter or the randomness involved in the retrieval process. While we proceed with parameter selection based on the available data, the choices have been made with low confidence due to these factors. The final configuration for the FiDeLiS approach is shown in \autoref{tab:fidelis_final_config}.

\paragraph{Large Language Model:} 
Looking at the results, the choice of \gls{llm} has significantly impacted the performance of FiDeLiS. The baseline model, \emph{gpt-4o-mini}, achieved Recall and Hits@10 scores of 0.129. The \emph{gpt-o3-mini} model performed slightly better, yielding scores of 0.136 (+5.4\%). The \emph{gpt-4o} model performed considerably worse with a Recall of 0.076 (-41.1\%) and Hits@10 at 0.076 (-41.1\%). This is surprising because similar results are not reflected in the experiments of other approaches. In addition, the open-source models tested failed almost completely on the task as their scores are near zero. We assume that this can be attributed to the requirements for structured \gls{llm} outputs that the FiDeLiS approach has, which these models struggled to produce consistently. Based solely on the results of this experiment, \textbf{\emph{gpt-O3-mini}} is the best choice for the final FiDeLiS configuration.

\paragraph{Embedding Model:} 
The choice of the embedding model also influenced the results of FiDeLiS. The baseline \emph{mxbai-embed-large} model achieved the score 0.129 for both Recall and Hits@10. The \emph{text-embedding-3-large} model yielded a slightly better but almost negligible performance, with a Recall of 0.137 (+6.2\%) and Hits@10 of 0.136 (+5.4\%). In contrast, \emph{granite-embedding} performed significantly worse with a score of 0.083 (-35.7\%) on both metrics. Based on achieving the highest scores in the comparison, we selected \textbf{\emph{text-embedding-3-large}} for the final configuration.

\paragraph{Top k:} 
The baseline value of 10 resulted in a Recall and Hits@10 of 0.129. Increasing the parameter to 20 decreased Recall by 35.7\% and Hits@10 by 44.2\%, while setting the parameter to 30 resulted in an even larger drop of 47.3\% for Recall and 58.9\% for Hits@10. This indicates that increasing the width worsened the results. Therefore, we selected the value of \textbf{10} for the final configuration.

\paragraph{Top n:} 
For the top-n parameter, the baseline value of 10 achieved a Recall and Hits@10 of 0.129. Increasing the value to 20 resulted in a Recall of 0.118 (-8.5\%) and Hits@10 of 0.117 (-9.3\%), lowering the performance. Increasing the top-n parameter to 30 decreased performance again by -14.7\%. Although the absolute differences are small, given the low overall scores, the trend indicates that the baseline value performed the best. Therefore, we set the parameter value to \textbf{10} in the final configuration.

\paragraph{Alpha:}
The authors of the paper suggest using an alpha score of 0.3. It achieved a Recall and Hits@10 of 0.129. Lowering the value to 0.1 decreased the scores to 0.095 (-26.4\%) and increasing the value to 0.6 reduced the scores to 0.110 (-14.7\%). As a result of this data, we set the value to \textbf{0.3} in the final configuration.

\paragraph{Do Reranking:} 
Not using the reranking step after the retrieval yielded Recall and Hits@10 scores of 0.129. Enabling the reranking resulted in lower performance for both Recall (-8.5\%) and Hits@10 (-9.3\%). As the reranking process negatively affected the results, we \textbf{disabled} it for the final configuration of FiDeLiS.

\paragraph{Do Question Augmentation:}
Similar to the reranking step, enabling question augmentation proved detrimental to the performance. Compared to the baseline without augmentation, which has both scores at 0.129, enabling it caused a substantial drop in both Recall and Hits@10 to 0.072 (-44.2\%). Given this significant negative impact, we \textbf{disabled} the augmentation for the final configuration.

\begin{table}[t]
    \centering
    \begin{tabular}{l l}   
        \toprule
        \textbf{Parameter} & \textbf{Value} \\
        \midrule
        \texttt{Top k} & 10 \\
        \texttt{Top n} & 10 \\
        \texttt{Alpha} & 0.3 \\
        \texttt{LLM} & gpt-o3-mini \\
        \texttt{Embedding Model} & text-embedding-3-large \\ 
        \texttt{Do Question Augmentation} & False \\
        \texttt{Do Reranking} & False \\
        \bottomrule
    \end{tabular}
    \caption[Final Configuration of FiDeLis]{The final configuration used for subsequent experiments for the FiDeLiS \gls{kgqa} baseline approach.}
    \label{tab:fidelis_final_config}
\end{table}


\section{Parameter Selection for Mindmap}
\label{sec:param_selection_mindmap}

In the sections that follow, we initially outline the base configuration and the range of parameters explored during the parameter selection phase for the Mindmap \gls{kgqa} baseline approach. Subsequently, we analyze the results of the test runs and detail the parameter values chosen for the final setup utilized in our subsequent experiments.

\subsection{Base Configuration and Parameter Ranges}

\begin{table}[t]
    \centering
    \begin{tabularx}{\textwidth}{l X}
        \toprule
        \textbf{Parameter} & \textbf{Parameter Space} \\
        \midrule
        \texttt{Final Paths To Keep} & \underline{10}, 20, 30 \\
        \texttt{Shortest Paths To Keep} & \underline{10}, 20, 30 \\
        \texttt{Neighbors to Keep} & \underline{10}, 20, 30 \\
        \texttt{Do Question Augmentation} & \underline{False}, True \\
        \texttt{Do Reranking} & \underline{False}, True \\
        \texttt{\gls{llm}} & \underline{gpt-4o-mini}, gpt-4o, o3-mini, \underline{Qwen2.5-14B}, Llama3.1-8B \\
        \texttt{Embedding Model} & \underline{mxbai-embed-large}, text-embedding-3-large, \newline granite-embedding \\
        \bottomrule
    \end{tabularx}
    \caption[Base Configuration and Parameter Space for Mindmap]{The base configuration (\underline{underlined}) and parameter space for the Mindmap \gls{kgqa} approach.}
    \label{tab:mindmap_tuning_configs}
\end{table}

The parameter values for the base configuration, together with the ranges of parameters tested for Mindmap, are shown in \autoref{tab:mindmap_tuning_configs}. In what follows, we provide a concise overview of each parameter individually.

\paragraph{Final Paths To Keep:} This parameter determines how many of the computed evidence paths are retained for the final answer. Increasing this value allows for the consideration of more context but also increases cost and may introduce noise. We initially set this value to 10 because this is the maximum number of golden triples requested by the applied \gls{kgqa} dataset. We then tested the values 20 and 30.

\paragraph{Shortest Paths to Keep:} This parameter determines how many candidate shortest paths are retained during the search between entities. A higher value allows to consider more context but also increases cost and complexity. Same as with the previous parameter, the base configuration value has been set to 10 and we tested the values 20 and 30.

\paragraph{Neighbors to Keep:} This parameter determines how many one-hop neighbor relationships are included when building the prompt for the \gls{llm}. As with the previous parameters, a higher value allows for more context but increases cost and complexity. We set it to 10 by default and increased the values to 20 and 30.

\paragraph{Do Question Augmentation:} Same as with the other \gls{kgqa} approaches, we disabled the question augmentation in the base configuration and assessed its impact on the results during
testing.

\paragraph{Do Reranking:} Same as with the other \gls{kgqa} approaches, we disabled the reranking by default and assessed the impact of enabling the reranking in one of the test runs.

\paragraph{LLM:} The \gls{llm} has several tasks in the Mindmap approach. First, it is responsible for the extraction of entities from the question. Second, it transforms the path information into a natural language description. Lastly, it generates the final answer. The models that were tried out are introduced in Section~\ref{sec:selection_planning_llms}. In our initial setup, we employed two models, namely \emph{gpt-4o-mini} and \emph{Qwen2.5-14B}, to decrease the runtime of the parameter selection process by executing several configurations simultaneously. Consequently, embedding models and other \glspl{llm} were evaluated using the \emph{Qwen2.5-14B} model as the baseline, while all other parameters were tested with \emph{gpt-4o-mini} as the foundation.

\paragraph{Embedding Model:} The embedding model determines the model that is used for embedding the entities of the knowledge graph and the question. The models that we used are introduced in Section~\ref{sec:selection_planning_llms}. We use \emph{mxbai-embed-large} in the base configuration because of its runtime and cost efficiency.


\subsection{Parameter Selection}

\begin{table}[t]
    \centering
    \begin{tabular}{llll}
        \toprule
        \textbf{Parameter } & \textbf{Config} & \textbf{Recall} & \textbf{Hits@10} \\
        \midrule
        \multirow{5}{*}{Large Language Model} 
            & \underline{gpt-4o-mini} & 0.113 & \textbf{0.029} \\
            & gpt-4o & 0.107 (-5.3\%) & 0.021 (-27.6\%) \\
            & gpt-O3-mini & \textbf{0.113 (+0.0\%)} & 0.013 (-55.2\%) \\
            & qwen2.5:14b & 0.085 (-24.8\%) & 0.008 (-72.4\%) \\
            & llama3.1 & 0.000 (-100.0\%) & 0.000 (-100.0\%) \\
        \midrule
        \multirow{3}{*}{Embedding Model} 
            & \underline{mxbai-embed-large} & 0.085 & 0.008 \\
            & text-embedding-3-large & \textbf{0.118 (+38.8\%)} & 0.006 (-25.0\%) \\
            & granite-embedding & 0.083 (-2.4\%) & \textbf{0.019 (+137.5\%)} \\
        \midrule
        \multirow{4}{*}{Neighbors to Keep}
            & \underline{10} & \textbf{0.113} & \textbf{0.029} \\
            & 20 & 0.113 (+0.0\%) & 0.029 (+0.0\%) \\
            & 30 & 0.113 (+0.0\%) & 0.029 (+0.0\%) \\
        \midrule
        \multirow{4}{*}{Final Paths to Keep}
            & \underline{10} & \textbf{0.113} & \textbf{0.029} \\
            & 20 & 0.113 (+0.0\%) & 0.029 (+0.0\%) \\
            & 30 & 0.113 (+0.0\%) & 0.029 (+0.0\%) \\
        \midrule
        \multirow{4}{*}{Shortest Paths to Keep}
            & \underline{10} & \textbf{0.113} & \textbf{0.029} \\
            & 20 & 0.113 (+0.0\%) & 0.029 (+0.0\%) \\
            & 30 & 0.113 (+0.0\%) & 0.029 (+0.0\%) \\
        \midrule
        \multirow{2}{*}{Do Reranking} 
            & \underline{False} & \textbf{0.129} & \textbf{0.129} \\
            & True & 0.118 (-6.2\%) & 0.121 (-6.2\%) \\      
        \midrule
        \multirow{2}{*}{Do Question Augmentation} 
            & \underline{False} & 0.129 & 0.129 \\
            & True & 0.129 (\(\pm\)0.0\%) & 0.129 (\(\pm\)0.0\%) \\
        \bottomrule
    \end{tabular}
    \caption[Results of the Parameter Selection Process for Mindmap]{Results of the parameter selection process for Mindmap. The base configuration parameter is \underline{underlined}, and the highest metric score per parameter is \textbf{bold}.}
    \label{tab:mindmap_parameter_selection}
\end{table}

The parameter selection results for the Mindmap approach with 16 configurations are detailed in \autoref{tab:mindmap_parameter_selection}, with the final configuration shown in \autoref{tab:mindmap_final_config}. Similarly to the FiDeLiS approach, we observed low performance across all configurations, particularly for the Hits@10 metric, which often remains near zero. This indicates that Mindmap significantly struggles with this question-answering task, which we further discuss in Section~\ref{sec:discussion_on_evaluation_results}. Given these data, we only reviewed the Recall value as the Hits@10 scores were too low to observe significant differences. Moreover, given the generally low scores, it is not possible to attribute the results to parameter changes or random variations in the retrieval process. Consequently, our decisions have been made with limited confidence because of these issues.

\paragraph{Large Language Model:}
The baseline model that was used is \emph{gpt-4o-mini} which scored a Recall of 0.113. The model \emph{gpt-o3-mini} has the same Recall value of 0.113 while \emph{gpt-4o} lost 5.3\%. Similarly to FiDeLis, open source models performed poorly. The \emph{Llama3.1} model failed entirely and the \emph{qwen2.5:14b} model lost 24.8\% in Recall. Given that \emph{gpt-4o-mini} and \textbf{\emph{gpt-o3-mini}} achieved the same Recall, we decided on the latter for the final configuration because the general trend for the other approaches also tends towards this model.

\paragraph{Embedding Model:}
The baseline model \emph{mxbai-embed-large} achieved a Recall of 0.085, while the \emph{text-embedding-3-large} boosted the Recall value to 0.118 (+38.8\%). In contrast, \emph{granite-embedding} caused a minor dip in Recall (-2.4\%). Given these data, the final configuration used the \textbf{\emph{text-embedding-3-large}} model.

\paragraph{Path and Neighbor Parameters:}
The tests for the parameters \texttt{Final Paths to Keep}, \texttt{Shortest Paths to Keep}, and \texttt{Neighbor Paths to Keep} were run with a baseline value of 10. In all three cases, increasing the value to 20 or 30 had no measurable effect. Since increasing the limits did not provide any benefit, we retained the base configuration values for the final configuration.

\paragraph{Do Reranking:}
Disabling reranking resulted in a Recall and Hits@10 value of 0.129. Enabling reranking should theoretically improve the Hits@10 score if the approach does not already rank the values by their relevance. However, enabling the reranking resulted in lower scores for both metrics with a Recall of 0.118 (-6.2\%) and Hits@10 of 0.121 (-6.2\%). Consequently, since reranking did not improve the performance, we \textbf{disabled} it for the final configuration.

\paragraph{Do Question Augmentation:}
Disabling the augmentation yielded a Recall and Hits@10 value 0.129. Enabling augmentation did not change the results. Because the results of other approaches suggest that the augmentation can negatively affect performance, we decided to \textbf{disable} the augmentation.

\begin{table}[t]
    \centering
    \begin{tabular}{l l}   
        \toprule
        \textbf{Parameter} & \textbf{Value} \\
        \midrule
        \texttt{Final Paths to Keep} & 10 \\
        \texttt{Shortest Paths to Keep} & 10 \\
        \texttt{Neighbor Paths to Keep} & 10 \\
        \texttt{LLM} & gpt-o3-mini \\
        \texttt{Embedding Model} & text-embedding-3-large \\ 
        \texttt{Do Question Augmentation} & False \\
        \texttt{Do Reranking} & False \\
        \bottomrule
    \end{tabular}
    \caption[Final Configuration for Mindmap]{The final configuration used for subsequent experiments for the Mindmap \gls{kgqa} approach.}
    \label{tab:mindmap_final_config}
\end{table}

\section{Threats to Validity}
\label{sec:param_selection_threats_to_validity}


In this section, we discuss threats to validity specific to the parameter selection process. The goal of the parameter selection process was to find a near-optimal value for each parameter of the \gls{kgqa} approaches, capable of achieving high overall retrieval performance on our \gls{kgqa} dataset.

There is a risk that we did not achieve this goal for every parameter we tested. This is due to the limited number of questions used and the limited parameter ranges tested. We were required to impose these constraints to limit the overall execution time of the selection process to stay within the scope of this master thesis. Therefore, our requirement was not to achieve an optimum but to get as close to it as possible, given the constraints. We minimized this risk by carefully considering parameter ranges to ensure that each value tested was both useful and plausible.

Furthermore, we recognize a risk regarding the HubLink approach concerning model selection. This is because, during our development, the prompts were tested and optimized on the \emph{gpt-4o-mini} model. Consequently, there could be a bias towards OpenAI models. Moreover, we were unable to run larger open-source models as we were restricted by hardware requirements. 

Additionally, the results exhibit some level of stochasticity, which means that for the same question, the contexts retrieved by the retriever can vary. This can be attributed to two phenomena. First, since we are working with \glspl{llm}, they exhibit a certain degree of randomness in their outputs. However, to keep this randomness low, we set the temperature parameter to zero, which reduces the risk of varying outputs. Second, some retrievers work with embedding models, which also have a certain degree of randomness when transforming text into vectors. Finally, the vectors are stored in the Chroma\footnote{\url{https://www.trychroma.com/} [last accessed 28.03.2025]} database, which uses the HNSW \cite{malkov_efficient_2018} indexing algorithm that is inherently non-deterministic to a certain degree (see Section~\ref{sec:fundamentals_ann_search}). We further observed that each time the database starts, the index is reinitialized, which can cause minor changes to vectors that are very close to each other in the \gls{ann} search compared to previous runs. Despite this, our analysis remains valid, as the randomness induces negligible variations that we do not see as significant enough to impact the overall findings.



% Another risk to validity arises from the fact that we were not able to perform extensive hyper-parameter tuning. As a result, the parameters we selected are not necessarily the best possible ones. This means that the performance of the retrievers might actually be better than what is reflected in our results. Nevertheless, limiting the number of parameters was necessary in order to keep the execution time within a reasonable scope. When selecting the test parameters, we focused on their practical applicability and chose ranges that offer a wide test coverage.



% Chroma DB randomness: https://github.com/chroma-core/chroma/issues/2675