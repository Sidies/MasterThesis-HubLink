
\section{Thesis Outline}

After the introduction, the remainder of the master thesis is structured as follows: 

\textbf{\autoref{ch:fundamentals}} establishes the fundamental concepts upon which the research is built. This chapter introduces \glspl{kg}, with a specific focus on \glspl{rkg}. It further explains the \gls{orkg}, which is the graph used to conduct our experiments on. The chapter then continues with an explanation of \gls{kgqa}, the primary research field toward which this work is contributing. Furthermore, it includes an explanation of \glspl{llm}, covers \gls{grag}, \gls{ann} search, and a range of evaluation metrics relevant to retrieval and generation components. Finally, the chapter explains the concept of taxonomies, including their formal definition, development approaches, and evaluation methods.

\textbf{\autoref{ch:related_work}} provides a comprehensive review of existing literature relevant to this thesis. It first examines \gls{kgqa} approaches that utilize \glspl{llm} in the scholarly and open domains. The discussion then continues with taxonomy development and question classification, covering systematic approaches to taxonomy construction and evaluation, alongside question classification for scholarly \gls{qa} on \glspl{kg}. Finally, it examines the benchmarking of \gls{kgqa} systems, including datasets targeting open-domain and scholarly domain \glspl{kg}, and methodologies for \gls{qa} dataset construction.

\textbf{\autoref{ch:hublink}} presents our primary contribution of this thesis: \emph{HubLink}. It provides an overview of the HubLink approach, detailing the indexing, retrieval, and generation phases. It then proceeds to explain the formal definitions of key concepts of the approach, before explaining the approach in detail using pseudocode. In addition, the chapter outlines the challenges and design rationale for decisions made during the development of the approach. Finally, the chapter concludes with a discussion on the generalizability, scalability, index updating mechanisms, and limitations.

\textbf{\autoref{ch:taxonomy_construction_approach}} outlines the systematic process proposed in this thesis for the creation of taxonomies by synthesizing knowledge from the literature. The chapter begins with an evaluation plan that utilizes an \gls{gqm} model to facilitate the evaluation of a taxonomy. Subsequently, the iterative development process is outlined, which encompasses planning, literature survey, extraction of relevant concepts, clustering of these concepts, relevance assessment, the actual taxonomy construction and refinement stages, validation of the taxonomy, and finally, its application. The chapter also describes construction artifacts and discusses the limitations inherent in the proposed methodology.

\textbf{\autoref{ch:question_catalog}} details the application of the proposed systematic taxonomy construction approach, by creating a \gls{kgqa} retrieval taxonomy, aimed at the classification of questions for scholarly literature search. The chapter starts with the planning and a systematic literature survey, before describing the extraction of classes, including their distribution by category, domain, publication year, and citation metrics. Then, the clustering of extracted classes, including deduplication and categorization processes, is explained, followed by a relevance assessment of these clusters. Following this, the iterative construction and refinement of the taxonomy through three increments are presented. The chapter then systematically describes the categories and classes of the final taxonomy, before its application on \gls{swa} research questions is demonstrated. 

\textbf{\autoref{ch:implementation}} introduces the \gls{sqa} framework, central for the implementation of the \gls{kgqa} approaches, the creation of the \gls{kgqa} datasets, and the execution of the experiments. Furthermore, the chapter details the implementation of HubLink and the baselines \gls{kgqa} approaches, also describing the selection process by which the approaches have been chosen. The chapter further specifies the methods for accessing and populating the \gls{orkg}.

\textbf{\autoref{ch:experimentation_preliminaries}} sets the stage for the experimental evaluation of the HubLink approach against the baseline approaches. The chapter outlines the overall evaluation concept and describes the software and hardware environment used for the experiments. Furthermore, the creation of the \gls{kgqa} datasets, detailing the use cases for scholarly literature search, an overview of content granularity, and the dataset creation process is described. Following this, the evaluation framework and metrics are specified, including the evaluation targets and a detailed evaluation plan using the \gls{gqm} model. 

\textbf{\autoref{ch:parameter_selection_process}} details the methodology and results of the parameter selection process, which has been applied to select the configurations for the HubLink approach and the selected baseline methods. The chapter begins with the planning of the process, before presenting the results for the approaches HubLink, DiFaR, FiDeLiS, and Mindmap, each including their base configuration, parameter ranges explored, and the final selected parameters.

\textbf{\autoref{ch:experimentation}} presents and discusses the results of the comprehensive evaluation of the HubLink approach against the baseline methods. The evaluation is divided into two main parts: evaluating retrieval quality and evaluating answer alignment. The retrieval quality assessment examines the improvement of retrieval accuracy and relevance, the impact of operation complexity, applicability to different scholarly literature search use cases, the impact of type information in the question, robustness to structural and lexical variability in the graph schema, analysis of runtime and \gls{llm} token consumption, and the environmental sustainability impact. The answer alignment evaluation focuses on the semantical and factual consistency of generated answers, the generation of relevant answers, adherence to instructions provided in the question, and the consistency of generated answers with the retrieved context. The chapter concludes with a detailed discussion of the evaluation results.

\textbf{\autoref{ch:conclusion}} summarizes the key findings of the thesis and reflects on the research objectives and questions. It revisits the research questions posed at the beginning of the thesis, providing answers based on the research conducted and contributions made and concludes by discussing avenues for future work.