
\chapter{Conclusion}
\label{ch:conclusion}

This master thesis addressed a research gap in applying training-free and schema-agnostic \gls{kgqa} approaches to the task of scholarly literature search. Our research aimed to overcome the limitations associated with existing approaches, specifically their reliance on fixed schemas and training data (\hyperref[sec:problem_statements]{\textbf{P1}}), and the lack of a standardized taxonomy for assessing capabilities of \gls{kgqa} systems (\hyperref[sec:problem_statements]{\textbf{P2}}).

Consequently, the master thesis introduced HubLink (\hyperref[enum:c1]{\textbf{C1}}), a novel approach that represents a significant step toward schema-agnostic and training-free \gls{kgqa} retrieval. By conceptually decomposing the graph into structures termed \emph{hubs}, which aggregate the knowledge from scholarly publications, HubLink enables a modular and source-aware retrieval process. Specifically, the ability of the approach to transparently trace the origin of information during inference provides a key requirement for scholarly literature retrieval. Furthermore, the embedding-based methodology employed by HubLink leverages the semantic connections captured by \glspl{llm} without necessitating explicit training data or adhering to fixed graph schemas. This independence makes the approach inherently adaptable to dynamic and evolving graph structures. 

Our evaluation results demonstrate that HubLink substantially improves the relevance and accuracy of retrieved contexts compared to state-of-the-art baseline \gls{kgqa} approaches, many of which struggled in the scholarly domain application. The evaluation further highlights that HubLink can be successfully applied to six specific use cases for the scholarly literature task, showing the potential of the approach to improve the efficiency of scientific communication. Although the evaluation demonstrated the clear advantages of HubLink in retrieving relevant information, it also identified specific areas for future enhancement. In particular, the results suggest that there are opportunities to refine the precision of the retrieved context and their ranking by relevance. Furthermore, the process of generating coherent and precise responses based on the retrieved context warrants further investigation and development. Future work should therefore focus on improving these aspects to fully realize the potential of the approach.

Complementing the HubLink approach, this thesis proposed a new taxonomy for \gls{kgqa} specifically tailored to the literature search task (\hyperref[enum:c2]{\textbf{C2}}). Developed through a systematic and operationalized construction process (\hyperref[enum:c2]{\textbf{C2.2}}), this taxonomy provides a structured framework to classify questions based on characteristics relevant to retrieval performance and scholarly information needs. The validation process, which includes the application to research questions and comparison with existing dataset-specific classifications, confirms that the taxonomy addresses a notable gap by offering a more structured and fine-grained framework for understanding and assessing \gls{kgqa} system capabilities in the scholarly domain. The usefulness of this taxonomy is further demonstrated through its successful application in the construction of new \gls{kgqa} datasets (\hyperref[enum:c3]{\textbf{C3}}) for the \gls{orkg}. These datasets were used to conduct the comprehensive evaluation of HubLink and the baseline methods across diverse question types. 

To conclude, current scholarly practices embed scholarly findings within unstructured documents, rendering data extraction and targeted search cumbersome. In contrast, \glspl{rkg} store scholarly findings in a structured network. The application of HubLink to retrieve relevant data from \glspl{rkg} in a \gls{kgqa} setting enables direct and precise access to scholarly information, reducing manual effort. This paradigm significantly reduces the effort required by researchers to locate relevant scholarly findings. Consequently, this work makes a significant contribution to the advancement of research in scholarly \gls{kgqa}, offering a promising path towards more efficient and effective scientific discovery and communication in the future.


\section{Research Questions Revisited}
\label{sec:research_questions_revisited}

The primary research goal was to design a new schema-agnostic and training-free retrieval approach capable of effectively retrieving scholarly knowledge from \glspl{rkg}, along with developing a systematic taxonomy to assess the capabilities of such systems. In this section, we revisit the two primary research questions that we defined and successfully addressed based on our contributions.

\begin{enumerate}[label={}]
    \item \textit{\hyperref[enum:rq1]{\textbf{RQ1}} How can a schema-agnostic retrieval algorithm leveraging an \gls{rkg} and a pre-trained \gls{llm} be developed for the \gls{kgqa} setting to effectively integrate diverse scholarly sources, adapt to evolving schemas and account for the provenance of information during retrieval without relying on training data?} 
\end{enumerate}

We found that the most effective way is to implement an embedding-based approach, which we refer to as HubLink. This approach does not require any further training data as it relies on the inherent semantic connections between words and sentences. This makes it particularly suitable for handling evolving schemas and unseen entities, as no knowledge about the schema is required during retrieval. Moreover, by dividing the graph into hub structures, with each hub representing a publication, the source of the information is taken into account during retrieval. By doing so, an overreliance on a single source is mitigated and the \gls{llm} is encouraged to integrate diverse perspectives.

\begin{enumerate}[label={}]
    \item \textit{\hyperref[enum:rq2]{\textbf{RQ2}} How can existing general \gls{qa} and \gls{kgqa} taxonomies be synthesized and extended to form a comprehensive taxonomy tailored to define the characteristics of questions posed to \gls{kgqa} retrieval systems for the literature search task?}
\end{enumerate}
We found that the creation of such a taxonomy can be successfully achieved through the taxonomy construction methodology introduced in this thesis. This systematic approach involves extracting relevant concepts from the existing literature, followed by clustering and relevance assessment phases to initialize a synthesized taxonomy, which is then incrementally refined and validated. The utility of this methodology was confirmed through the successful development of a specific taxonomy designed to classify question characteristics within the domain of \gls{kgqa} systems applied to literature searches. The practical value of the resulting taxonomy was further underscored during the creation of \gls{kgqa} benchmarking datasets. Here, the taxonomy served as a framework to ensure that the datasets cover a diverse range of question types, enabling a comprehensive assessment of different system capabilities during evaluation.


\section{Future Work}
\label{sec:future_work}

This thesis presented HubLink, a novel schema-agnostic and training-free \gls{kgqa} retrieval approach, alongside a new taxonomy for classifying \gls{kgqa} questions within the scholarly domain. Building upon these contributions and the findings presented, this section outlines potential directions for future work. These directions are organized based on their relevance to the HubLink retrieval approach as well as the developed taxonomy and its construction process.

\subsection{Schema-Agnostic and Training-Free KGQA Retrieval}

Future work related to the HubLink retrieval approach could focus on the following areas:

\paragraph{Improving Answer Generation} Our evaluation results presented in Section~\ref{sec:evaluating_answer_alignment} indicate limitations in the alignment between retrieved contexts and the final generated answers for HubLink. Future work could investigate methods to enhance this alignment. Potential strategies include advanced prompt engineering techniques aimed specifically at improving the accuracy of the answer generation process relative to the provided context or exploring alternative synthesis methods beyond single-prompt generation. 

\paragraph{Further Evaluations} As described in Section~\ref{sec:content_granularity}, the \gls{kgqa} datasets that were used in this thesis reside at the label-based granularity level. Future work could apply HubLink to other datasets that involve the retrieval of knowledge from \glspl{rkg} at a different level of granularity to verify the results provided in this thesis. This includes the evaluation of the \emph{linking} feature provided by HubLink, which was not useful in our evaluation scenario, due to the label-based abstraction level in our data. However, if data are used at other granularity levels, the linking feature could be beneficial.

\paragraph{Improving Relevancy Ranking} A limitation that our evaluation of retrieval performance in Section~\ref{sec:evaluating_relevance_and_robustness_of_retrieved_contexts} reveals is the observed limitation of HubLink to rank relevant context higher than irrelevant context. Consequently, future work could focus on improving this limitation, possibly through refined prompt strategies for relevance assessment or the integration of dedicated reranking steps following the retrieval phase.

\paragraph{Extension of Hub Content} One interesting research direction for enhancing HubLink involves the extension of the hubs with further information. Future work could explore expanding the index with information such as summaries or precomputations of numerical aggregations derived from associated papers. Such extensions possess the potential to enrich the contextual information available during retrieval and synthesis, possibly leading to more comprehensive and accurate answers.

\paragraph{Numerical Constraints} The processing of numerical constraints, such as dates, ranges of values, or specific metric comparisons, presents a potential challenge for embedding-based retrieval approaches like HubLink, as highlighted in previous research \cite{jin_floating-point_2024}. Although we do have temporal constraints in our evaluation and the retriever demonstrates that it is able to consider them during inference, these do not involve complex time spans. Future work should explicitly evaluate the performance of HubLink on questions requiring complex numerical reasoning or filtering.

\paragraph{Document-based Setting} In our work, we primarily focused on the retrieval of knowledge stored in the graph. Future work could evaluate HubLink in a document retrieval setting. In such a setting, HubLink uses the \gls{kg} as a supporting structure similar to other prominent approaches in the literature \cite{edge_local_2024,guo_lightrag_2024}. The use of the linking capability of HubLink could be utilized in such a scenario to retrieve relevant document text chunks and return them. In this case, the knowledge information from the graph serves as a preliminary step to allow source-aware retrieval. We have already conducted preliminary work towards this evaluation, including the implementation of LightRag \cite{guo_lightrag_2024} and Microsoft GraphRAG \cite{edge_local_2024} as baselines, the preparation of the evaluation code of the \gls{sqa} system to work with document-based approaches, and the implementation of an algorithm for extracting structured knowledge from scientific papers. However, the evaluation could not be completed in the time frame of this thesis.

\paragraph{Other Graphs} The evaluation in this thesis focused on the \gls{orkg}. Future studies could investigate the applicability and performance of HubLink on other \glspl{rkg} to assess its robustness and utility across different knowledge representations within the scholarly domain. Furthermore, as argued in Section~\ref{sec:hublink_generality_and_chances}, the design principles of HubLink suggest potential applicability beyond the scholarly domain, particularly for large and heterogeneous \glspl{kg}. Investigating the scalability and effectiveness of HubLink on various \glspl{kg} in other domains represents another important research direction to determine the generalization of the approach.


\subsection{Taxonomy Construction and Application in KGQA}

Regarding the proposed taxonomy and its construction process, future work could explore the following directions:

\paragraph{Taxonomy Application in Diverse Domains} The taxonomy developed in this work proved useful in constructing a \gls{kgqa} dataset tailored to scholarly literature searches. However, its potential applicability extends beyond this specific context. Future work could explore the utility of this taxonomy for characterizing questions or guiding dataset creation in different \gls{kgqa} domains. This may necessitate adaptations to accommodate domain-specific question types or information needs but would test the broader relevance of the classification scheme.

\paragraph{Continuation of Literature Survey} The construction of the \gls{kgqa} retrieval taxonomy involves a comprehensive literature survey that was stopped prematurely after the second iteration. Although we argue that most interesting approaches that stem from the seed papers are considered, future work could extend the search by incorporating more sources from different domains. This process can be easily extended using the artifacts provided in our replication package \cite{schneider_replication_2025}.

\paragraph{Construction of new Taxonomies} This thesis introduced a systematic process for constructing taxonomies. To validate its broader utility and identify potential limitations, further application by researchers in different contexts would be valuable. Future work could involve employing this construction methodology to develop taxonomies in other subject areas or for different purposes. Such effort would help assess the robustness, adaptability, and generalizability of the proposed taxonomy construction process itself.



% 

\section{Final Remarks}
\label{sec:final_remarks}


% Es wurde ein Research Gap identifiziert von Methoden die schema-agnostic und trainings frei sind. Dies liegt daran, dass die Baseline Methoden ganz offensichtlich große Probleme mit der Aufgabe haben. Mit HubLink konnten wir dem Lösen dieses Problems ein großes Stück näher kommen. Man muss dazu sagen, dass es offen bleibt, wie gut Trainingsbasierte Methoden auf dieser Aufgabe abschneiden. Allerdings bleibt bei Trainingsbasierten Methoden wie auch bei Semantik Parsing Methoden die Frage offen, wie gut diese mit ungesehen Daten, also Evolving Schema umgehen können. Dabei gibt es hinzuzufügen, dass Semantic Parsing Methoden bereits exzellente Performance auf dem SciQA Benchmark gezeigt haben. Dieser ist sehr ähnlich zu unserem Benchmark d.h. es ist zu erwarten das Semantic Parsing Methoden auch in diesem Fall exzellent abschneiden würden. Allerdings würden diese bei Experiment 2 Probleme bekommen, da sich hier das Schema ändert.

% Taxonomy konnte erfolgreich in SWA context angewandt werden. Es wurde gezeigt das sie helfen dabei einzuschätzen, welche Komplexität und Anforderungen Research Fragen haben wenn sie in einem KGQA system angewandt werden. Die Taxonomy kann also effektiv dabei helfen KGQA systeme für den SWA Bereich zu entwickeln. 

% Nur weil Retriever in der open domain gut funktionieren bedeutet dies noch nicht das diese auch im scholarly task gut funktionieren


% Die ERstellung der QA Datasets war schwieriger als erwartet und hat dardurch deutlich mehr Zeit in Anspruch genommen als geplant. Für die Zukunft wäre es sinnvoll, die LLM gestütze Generierung von KGQA Datasets genauer zu untersuchen. Die Herausforderungen und Requirements sollten empirisch bestimmt werden.

