

\section{Final Remarks}
\label{sec:final_remarks}


% Es wurde ein Research Gap identifiziert von Methoden die schema-agnostic und trainings frei sind. Dies liegt daran, dass die Baseline Methoden ganz offensichtlich große Probleme mit der Aufgabe haben. Mit HubLink konnten wir dem Lösen dieses Problems ein großes Stück näher kommen. Man muss dazu sagen, dass es offen bleibt, wie gut Trainingsbasierte Methoden auf dieser Aufgabe abschneiden. Allerdings bleibt bei Trainingsbasierten Methoden wie auch bei Semantik Parsing Methoden die Frage offen, wie gut diese mit ungesehen Daten, also Evolving Schema umgehen können. Dabei gibt es hinzuzufügen, dass Semantic Parsing Methoden bereits exzellente Performance auf dem SciQA Benchmark gezeigt haben. Dieser ist sehr ähnlich zu unserem Benchmark d.h. es ist zu erwarten das Semantic Parsing Methoden auch in diesem Fall exzellent abschneiden würden. Allerdings würden diese bei Experiment 2 Probleme bekommen, da sich hier das Schema ändert.

% Taxonomy konnte erfolgreich in SWA context angewandt werden. Es wurde gezeigt das sie helfen dabei einzuschätzen, welche Komplexität und Anforderungen Research Fragen haben wenn sie in einem KGQA system angewandt werden. Die Taxonomy kann also effektiv dabei helfen KGQA systeme für den SWA Bereich zu entwickeln. 

% Nur weil Retriever in der open domain gut funktionieren bedeutet dies noch nicht das diese auch im scholarly task gut funktionieren